% Options for packages loaded elsewhere
% Options for packages loaded elsewhere
\PassOptionsToPackage{unicode}{hyperref}
\PassOptionsToPackage{hyphens}{url}
\PassOptionsToPackage{dvipsnames,svgnames,x11names}{xcolor}
%
\documentclass[
  12pt,
  letterpaper]{article}
\usepackage{xcolor}
\usepackage[margin=1in]{geometry}
\usepackage{amsmath,amssymb}
\setcounter{secnumdepth}{5}
\usepackage{iftex}
\ifPDFTeX
  \usepackage[T1]{fontenc}
  \usepackage[utf8]{inputenc}
  \usepackage{textcomp} % provide euro and other symbols
\else % if luatex or xetex
  \usepackage{unicode-math} % this also loads fontspec
  \defaultfontfeatures{Scale=MatchLowercase}
  \defaultfontfeatures[\rmfamily]{Ligatures=TeX,Scale=1}
\fi
\usepackage{lmodern}
\ifPDFTeX\else
  % xetex/luatex font selection
\fi
% Use upquote if available, for straight quotes in verbatim environments
\IfFileExists{upquote.sty}{\usepackage{upquote}}{}
\IfFileExists{microtype.sty}{% use microtype if available
  \usepackage[]{microtype}
  \UseMicrotypeSet[protrusion]{basicmath} % disable protrusion for tt fonts
}{}
\usepackage{setspace}
\makeatletter
\@ifundefined{KOMAClassName}{% if non-KOMA class
  \IfFileExists{parskip.sty}{%
    \usepackage{parskip}
  }{% else
    \setlength{\parindent}{0pt}
    \setlength{\parskip}{6pt plus 2pt minus 1pt}}
}{% if KOMA class
  \KOMAoptions{parskip=half}}
\makeatother
% Make \paragraph and \subparagraph free-standing
\makeatletter
\ifx\paragraph\undefined\else
  \let\oldparagraph\paragraph
  \renewcommand{\paragraph}{
    \@ifstar
      \xxxParagraphStar
      \xxxParagraphNoStar
  }
  \newcommand{\xxxParagraphStar}[1]{\oldparagraph*{#1}\mbox{}}
  \newcommand{\xxxParagraphNoStar}[1]{\oldparagraph{#1}\mbox{}}
\fi
\ifx\subparagraph\undefined\else
  \let\oldsubparagraph\subparagraph
  \renewcommand{\subparagraph}{
    \@ifstar
      \xxxSubParagraphStar
      \xxxSubParagraphNoStar
  }
  \newcommand{\xxxSubParagraphStar}[1]{\oldsubparagraph*{#1}\mbox{}}
  \newcommand{\xxxSubParagraphNoStar}[1]{\oldsubparagraph{#1}\mbox{}}
\fi
\makeatother


\usepackage{longtable,booktabs,array}
\usepackage{calc} % for calculating minipage widths
% Correct order of tables after \paragraph or \subparagraph
\usepackage{etoolbox}
\makeatletter
\patchcmd\longtable{\par}{\if@noskipsec\mbox{}\fi\par}{}{}
\makeatother
% Allow footnotes in longtable head/foot
\IfFileExists{footnotehyper.sty}{\usepackage{footnotehyper}}{\usepackage{footnote}}
\makesavenoteenv{longtable}
\usepackage{graphicx}
\makeatletter
\newsavebox\pandoc@box
\newcommand*\pandocbounded[1]{% scales image to fit in text height/width
  \sbox\pandoc@box{#1}%
  \Gscale@div\@tempa{\textheight}{\dimexpr\ht\pandoc@box+\dp\pandoc@box\relax}%
  \Gscale@div\@tempb{\linewidth}{\wd\pandoc@box}%
  \ifdim\@tempb\p@<\@tempa\p@\let\@tempa\@tempb\fi% select the smaller of both
  \ifdim\@tempa\p@<\p@\scalebox{\@tempa}{\usebox\pandoc@box}%
  \else\usebox{\pandoc@box}%
  \fi%
}
% Set default figure placement to htbp
\def\fps@figure{htbp}
\makeatother


% definitions for citeproc citations
\NewDocumentCommand\citeproctext{}{}
\NewDocumentCommand\citeproc{mm}{%
  \begingroup\def\citeproctext{#2}\cite{#1}\endgroup}
\makeatletter
 % allow citations to break across lines
 \let\@cite@ofmt\@firstofone
 % avoid brackets around text for \cite:
 \def\@biblabel#1{}
 \def\@cite#1#2{{#1\if@tempswa , #2\fi}}
\makeatother
\newlength{\cslhangindent}
\setlength{\cslhangindent}{1.5em}
\newlength{\csllabelwidth}
\setlength{\csllabelwidth}{3em}
\newenvironment{CSLReferences}[2] % #1 hanging-indent, #2 entry-spacing
 {\begin{list}{}{%
  \setlength{\itemindent}{0pt}
  \setlength{\leftmargin}{0pt}
  \setlength{\parsep}{0pt}
  % turn on hanging indent if param 1 is 1
  \ifodd #1
   \setlength{\leftmargin}{\cslhangindent}
   \setlength{\itemindent}{-1\cslhangindent}
  \fi
  % set entry spacing
  \setlength{\itemsep}{#2\baselineskip}}}
 {\end{list}}
\usepackage{calc}
\newcommand{\CSLBlock}[1]{\hfill\break\parbox[t]{\linewidth}{\strut\ignorespaces#1\strut}}
\newcommand{\CSLLeftMargin}[1]{\parbox[t]{\csllabelwidth}{\strut#1\strut}}
\newcommand{\CSLRightInline}[1]{\parbox[t]{\linewidth - \csllabelwidth}{\strut#1\strut}}
\newcommand{\CSLIndent}[1]{\hspace{\cslhangindent}#1}



\setlength{\emergencystretch}{3em} % prevent overfull lines

\providecommand{\tightlist}{%
  \setlength{\itemsep}{0pt}\setlength{\parskip}{0pt}}



 


\usepackage{booktabs}
\usepackage{longtable}
\usepackage{array}
\usepackage{multirow}
\usepackage{wrapfig}
\usepackage{float}
\usepackage{colortbl}
\usepackage{pdflscape}
\usepackage{tabu}
\usepackage{threeparttable}
\usepackage{threeparttablex}
\usepackage[normalem]{ulem}
\usepackage{makecell}
\usepackage{xcolor}
\usepackage{setspace}
\usepackage{booktabs}
\usepackage{caption}
\usepackage{longtable}
\usepackage{array}
\usepackage{multirow}
\usepackage{float}
\usepackage{graphicx}
\usepackage{threeparttable}
\usepackage{threeparttablex}
\usepackage{colortbl}
\usepackage{xcolor}
\captionsetup{width=\textwidth}
\setlength\LTcapwidth{\textwidth}
\floatplacement{figure}{H}
\floatplacement{table}{H}
\raggedright
\makeatletter
\@ifpackageloaded{caption}{}{\usepackage{caption}}
\AtBeginDocument{%
\ifdefined\contentsname
  \renewcommand*\contentsname{Table of contents}
\else
  \newcommand\contentsname{Table of contents}
\fi
\ifdefined\listfigurename
  \renewcommand*\listfigurename{List of Figures}
\else
  \newcommand\listfigurename{List of Figures}
\fi
\ifdefined\listtablename
  \renewcommand*\listtablename{List of Tables}
\else
  \newcommand\listtablename{List of Tables}
\fi
\ifdefined\figurename
  \renewcommand*\figurename{Figure}
\else
  \newcommand\figurename{Figure}
\fi
\ifdefined\tablename
  \renewcommand*\tablename{Table}
\else
  \newcommand\tablename{Table}
\fi
}
\@ifpackageloaded{float}{}{\usepackage{float}}
\floatstyle{ruled}
\@ifundefined{c@chapter}{\newfloat{codelisting}{h}{lop}}{\newfloat{codelisting}{h}{lop}[chapter]}
\floatname{codelisting}{Listing}
\newcommand*\listoflistings{\listof{codelisting}{List of Listings}}
\makeatother
\makeatletter
\makeatother
\makeatletter
\@ifpackageloaded{caption}{}{\usepackage{caption}}
\@ifpackageloaded{subcaption}{}{\usepackage{subcaption}}
\makeatother
\usepackage{bookmark}
\IfFileExists{xurl.sty}{\usepackage{xurl}}{} % add URL line breaks if available
\urlstyle{same}
\hypersetup{
  pdftitle={The Education of a Citizenry: Hope, Demobilization, and Authoritarian Normalization in Cambodia, 2008--2021},
  pdfauthor={Jeffrey Stark},
  pdfkeywords={Cambodia, Asian Barometer Survey, authoritarian
normalization, demobilization by subtraction, competitive
authoritarianism, democratic aspirations},
  colorlinks=true,
  linkcolor={blue},
  filecolor={Maroon},
  citecolor={blue},
  urlcolor={blue},
  pdfcreator={LaTeX via pandoc}}


\title{The Education of a Citizenry: Hope, Demobilization, and
Authoritarian Normalization in Cambodia, 2008--2021}
\author{Jeffrey Stark}
\date{February 24, 2026 at 7:22 AM}
\begin{document}
\maketitle
\begin{abstract}
Between 2008 and 2021, Cambodian citizens underwent one of the most
dramatic transformations in political orientation recorded in
contemporary survey research. Drawing on four waves of Asian Barometer
Survey data (Waves 2, 3, 4, and 6; N ≈ 1,000--1,242 per wave), this
article traces the arc of that transformation as a sequential process:
from post-conflict acquiescence, through a remarkable peak of democratic
aspiration and civic mobilization, to a final condition of political
withdrawal and regime normalization following the elimination of
organized opposition. The Cambodian case offers an unusually clean
empirical window into how dominant-party authoritarian regimes produce
political quiescence---not through the cultivation of enthusiastic
support, but through a process the article terms \emph{demobilization by
subtraction}, in which the removal of credible democratic alternatives
restructures citizen orientations across multiple attitudinal and
behavioral dimensions simultaneously. The findings challenge accounts
that treat authoritarian legitimacy as a product of performance or
ideology, suggesting instead that the most effective mechanism of regime
stabilization may be the systematic closure of the political
imagination.
\end{abstract}


\setstretch{2}
\section{Introduction}\label{introduction}

In 2012, Cambodians rated the democratic future of their country at 9.6
out of 10.

It is worth pausing on that number. On a scale designed to capture the
full range of democratic expectation, Cambodian respondents placed
themselves at the extreme ceiling---more optimistic about their
democratic trajectory than citizens of virtually any other country
surveyed by the Asian Barometer (Hu et al., 2023). This was not a
country with free and fair elections, an independent judiciary, or a
vibrant free press. It was a dominant-party authoritarian state under
the continuous rule of Hun Sen and the Cambodian People's Party (CPP), a
regime that had held power since 1985 and showed no signs of voluntary
departure. And yet Cambodians believed, with remarkable consistency,
that democracy was coming.

Nine years later, that same measure had fallen to 6.7 out of 10. In the
same period, independent civic engagement---contacting influential
figures outside the state, discussing politics, attending to political
news---had collapsed to historic lows, even as participation through
officially sanctioned channels quietly rose. Acceptance of authoritarian
governance alternatives, including strongman rule, single-party rule,
expert rule, and military rule, had risen sharply. Reported encounters
with corruption had plummeted from 49 percent to 15 percent. And voter
turnout in national elections, now conducted without any viable
opposition, had paradoxically climbed to 88 percent.

What happened in between is the subject of this article. The answer,
stated plainly, is that the Cambodian People's Party dissolved the only
credible opposition party, the Cambodia National Rescue Party (CNRP), in
November 2017, and in doing so restructured the political field so
thoroughly that Cambodian citizens adjusted their attitudes, behaviors,
and expectations in a remarkably coherent pattern. This article argues
that this adjustment is best understood not as a story of authoritarian
consolidation in the conventional sense---not as the triumph of
ideology, coercion, or performance legitimacy---but as something quieter
and, in its way, more profound: the education of a citizenry in the
limits of the politically possible.

The phenomenon under investigation---the systematic elimination of
organized opposition within an otherwise functioning electoral
system---has become increasingly common across consolidating
authoritarian regimes, from Turkey's suppression of the Peoples'
Democratic Party to Russia's dismantling of Navalny's organizational
network. Yet existing scholarship has focused primarily on the
institutional mechanics of opposition elimination rather than its
downstream effects on mass political orientations. This article proposes
that opposition elimination produces a distinctive observable signature
in public opinion: synchronized, multi-domain shifts in which democratic
expectations, independent civic engagement, political attention, and
willingness to report corruption all decline in parallel, while
acceptance of authoritarian governance alternatives and participation
through regime-managed channels rise. The critical diagnostic indicator
is a divergence between electoral and civic participation---rising voter
turnout amid collapsing independent engagement---that signals the
transformation of elections from mechanisms of political choice into
rituals of regime compliance. If this signature proves generalizable
beyond the Cambodian case, it offers both a theoretical framework for
understanding how authoritarian regimes stabilize through subtraction
rather than through active legitimation, and a practical diagnostic tool
for identifying democratic closure in real time using existing
cross-national survey infrastructure.

The article proceeds as a structured analytic narrative, organized
around the four survey waves as acts in a political drama. This is a
deliberate methodological choice. Cambodia's trajectory between 2008 and
2021 has the unusual property of being both analytically tractable---the
survey data capture discrete moments in a well-documented political
sequence---and narratively coherent, with each wave mapping onto a
distinct phase of political life. The narrative structure allows the
article to preserve the temporal dynamics that conventional
cross-sectional analysis tends to flatten, while the survey data anchor
the story in systematic evidence rather than impressionistic accounts.

The question of how dominant-party regimes endure has generated a rich
architectural vocabulary in contemporary scholarship. We are accustomed
to analyzing the structural mechanics of survival: the skewed playing
fields of competitive authoritarianism, the meticulous menu of
manipulation required to manage electoral autocracies, and the
stabilizing pillars of legitimation, repression, and
co-optation.\footnote{For the foundational typologies of these
  institutional survival mechanics, see Levitsky \& Way (2010); Schedler
  (2013); and Gerschewski (2013). While these frameworks masterfully
  explain elite-level regime consolidation, they are less equipped to
  explain the simultaneous, multi-dimensional shifts in mass public
  opinion observed in the Asian Barometer Survey data during periods of
  opposition collapse.} Yet, this institutional focus, while vital for
understanding the resilience of the state, often obscures the quiet,
downstream effects on the citizenry. Theories of regime survival tend to
treat the public as a variable to be managed---either repressed into
submission or bought off through patronage. This article shifts the
analytical lens. By tracing the Cambodian electorate's trajectory from
2008 to 2021, it focuses not on how the regime maintained power, but on
what happens to citizen orientations when the democratic alternative is
systematically removed. The resulting quiescence, this article argues,
is not a product of active authoritarian legitimation, but of
demobilization by subtraction.

The concept advanced here---demobilization by subtraction---is distinct
from the two mechanisms most commonly invoked to explain political
quiescence under authoritarianism. The first, repression-driven
demobilization, holds that citizens withdraw from political life because
the costs of participation have been raised through coercion,
surveillance, or the threat of punishment.\footnote{On repression as a
  mechanism of demobilization, the foundational treatment is Davenport
  (2007). For a comprehensive typology of repressive strategies, see
  also the essays collected in Davenport et al. (2005). On the specific
  mechanisms through which states demobilize protest, see Earl (2011).}
The second, co-optation, holds that citizens acquiesce because they have
been incorporated into the regime's distributional networks and
calculate that loyalty pays better than resistance.\footnote{The
  co-optation framework is developed most fully in Gandhi (2008). On the
  equilibrium logic of authoritarian power-sharing, see Przeworski et
  al. (2000).} Both mechanisms presuppose an active regime strategy
directed at the population. Demobilization by subtraction operates
differently. Its proximate cause is not an action taken against citizens
but an action taken against the political field itself: the removal of
the credible alternative around which civic mobilization had organized.
The result is not fear-based withdrawal or interest-based compliance but
something closer to what Gaventa termed ``quiescence''---the
internalization of powerlessness that follows from the sustained absence
of viable channels for political expression.\footnote{Gaventa's central
  insight---that sustained powerlessness produces not resistance but the
  internalization of quiescence---provides the closest existing
  theoretical analogue to the pattern observed in the Cambodian data,
  though his analysis is situated in a very different political context
  (Gaventa, 1980).} Where Lukes' third dimension of power operates by
shaping preferences so that grievances never form, demobilization by
subtraction operates one step later: grievances may persist, but the
perceived vehicle for their expression has been eliminated, producing a
cascade of behavioral and attitudinal adjustment that is visible across
multiple survey dimensions simultaneously.\footnote{The distinction
  matters: Lukes' third dimension operates \emph{before} grievances
  form; demobilization by subtraction operates \emph{after} grievances
  have formed and been politically activated, removing the vehicle for
  their expression rather than the grievances themselves (Lukes, 2005).}

The remainder of this article proceeds as follows. A data and
measurement section describes the Asian Barometer Survey waves, sampling
procedures, variable operationalization, and cross-wave comparability.
The analytic narrative is then organized as four acts corresponding to
the four available survey waves. Each section opens with a brief account
of the political conditions prevailing at the time of the survey, then
turns to the ABS data to trace how those conditions registered in
citizen attitudes and behaviors. The analysis tracks five thematic
domains across waves: political participation, authoritarian governance
preferences, democratic expectations, corruption experience and
perception, and media engagement and political interest. Following the
narrative, a section on alternative mechanisms adjudicates the
demobilization-by-subtraction interpretation against competing
explanations, drawing on subgroup decompositions, response-style
diagnostics, and placebo tests. A concluding section draws out the
theoretical implications of the Cambodian sequence for the study of
authoritarian durability and democratic aspiration.

\subsection{Data, Measures, and
Comparability}\label{data-measures-and-comparability}

This article draws on the Asian Barometer Survey (ABS), a repeated
cross-sectional survey program administered across East and Southeast
Asia since 2001. Cambodia has been included in four of the six completed
waves: Wave 2 (fielded 2008, N = 1,000), Wave 3 (2012, N = 1,200), Wave
4 (2015, N = 1,200), and Wave 6 (2021, N = 1,242). Cambodia was not
included in Wave 1 (2001--2003) or Wave 5 (2018--2019); the latter
omission is particularly consequential, as it means no ABS data exist
from the immediate aftermath of the CNRP dissolution and the 2018
one-party election. The survey's Wave 5 fieldwork period coincided with
the immediate aftermath of the forced dissolution of the CNRP in
November 2017 and the July 2018 national election in which the CPP ran
virtually unopposed---conditions under which credible public opinion
research could not be conducted. The absence of Wave 5 is thus not an
artifact of research design but a reflection of the political closure
this study documents: the very phenomenon under investigation rendered
standard survey fieldwork untenable. This gap means the analysis cannot
observe the immediate post-dissolution period directly, a limitation
addressed below through adjudication tests designed to distinguish
between an acute fear response and the longer-term normalization pattern
proposed here. The four available waves nevertheless bracket the
critical political sequence: pre-mobilization (2008), peak mobilization
(2012), post-crackdown selective retreat (2015), and post-dissolution
normalization (2021).

The ABS employs face-to-face interviews with nationally representative
probability samples drawn through multistage stratified random sampling.
The Cambodian samples are stratified by province and urban/rural
classification, with primary sampling units selected proportional to
population size and households selected through random walk procedures
within enumeration areas. The survey instrument is administered in
Khmer. Response rates across the four waves are consistently high,
ranging from approximately 80 to 90 percent, which is typical for
face-to-face surveys in Southeast Asia. No post-stratification weights
are applied in the present analysis; the samples are treated as
self-weighting, following the ABS's standard practice for single-country
analyses.\footnote{On ABS methodology and sampling design, see Chu et
  al. (2008). The ABS Technical Report for each wave provides
  country-level details on sampling frames, fieldwork dates, and
  response rates.}

The analysis tracks five thematic domains across waves. For each domain,
the specific indicators, scale constructions, and cross-wave
comparability considerations are as follows.

\emph{Political participation.} The ABS participation battery employs a
two-part structure: a gate question (``Have you ever done X?'') asked of
all respondents, followed by a frequency question asked only of those
who answer affirmatively. This article reports the gate
proportions---the share of the full sample reporting ever having engaged
in each activity---rather than the frequency means, which are
conditional on participation and based on much smaller subsamples. Six
gate items are tracked: contacting elected officials, contacting civil
servants, contacting influential persons, signing petitions, attending
demonstrations, and contacting the media. Of these, contacting
influential persons and attending demonstrations employ consistent
wording across all four waves and are treated as fully comparable. The
elected official and civil servant contact items use a narrower,
problem-motivated framing in Wave 3 (``to complain about or seek help
with a problem'') that differs from the broader framing in Waves 2, 4,
and 6; cross-wave comparisons involving Wave 3 for these items are
flagged accordingly. Petitions, demonstrations, and media contact were
not fielded in Wave 2. Community leader contact frequency (1--5 scale)
and self-reported voter turnout (binary) supplement the gate items. A
full comparability matrix documenting exact wording by wave is provided
in the appendix (Table A1).

\emph{Authoritarian governance preferences.} Four items assess
respondent evaluations of non-democratic governance alternatives:
single-party rule, strongman rule, military rule, and expert rule. Each
is measured on a consistent 1--4 scale (1 = ``very bad,'' 4 = ``very
good'') across all waves in which the item appears. Expert rule was not
fielded in Wave 2. These items are among the most stable in the ABS
instrument, with identical wording and response options across waves.

\emph{Democratic expectations.} Three items measure perceived democratic
quality on 0--10 scales: democratic past (how democratic was the country
ten years ago?), democratic present (how democratic is the country
today?), and democratic future (how democratic will the country be in
ten years?). These items were not fielded in Wave 2 and carry elevated
nonresponse rates across all waves---a point addressed in the
alternative mechanisms section below. The 0--10 scales are treated as
quasi-continuous, which is standard practice in the ABS literature and
consistent with the ordinal-as-interval convention widely adopted for
survey scales of this length.

\emph{Corruption experience and perception.} Witnessed corruption is
measured as a binary indicator (ever personally witnessed or experienced
government corruption). The Wave 2 item uses narrower wording focused
specifically on bribe solicitation, while Waves 3--6 use a broader
``witnessed or experienced'' framing; Wave 2 estimates for this item
should therefore be interpreted with caution. Perceived corruption in
national and local government is measured on consistent 1--4 scales
across all waves.

\emph{Media engagement and political interest.} Political interest,
political news following, and political discussion are measured on
consistent scales (1--4 or 1--3) across all waves. Internet news
frequency underwent substantial changes in response scale across waves
(6-point in Waves 2 and 3, 8-point in Wave 4, 9-point in Wave 6) and has
been harmonized to a common 1--6 frequency scale (1 = never, 6 = daily
or more); details of the harmonization procedure are documented in a
footnote at first use.

\emph{Analytic strategy.} The analysis is primarily descriptive,
reporting wave-level means, proportions, and cross-wave differences for
each indicator. All estimates are accompanied by 95 percent confidence
intervals computed from standard errors (binomial/Wilson intervals for
proportions; normal-approximation intervals for means). Cross-wave
differences for headline statistics are evaluated using two-sample
t-tests (for means) or two-sample proportion tests (for proportions).
The article does not employ regression modeling; the analytic leverage
derives from within-case temporal variation across a well-documented
political sequence rather than from cross-sectional covariate
adjustment. This design cannot establish causality in the formal sense,
but it can identify whether the timing, direction, and
domain-specificity of observed shifts are consistent with the
demobilization-by-subtraction mechanism and inconsistent with plausible
alternatives. The alternative mechanisms section addresses this
adjudication directly.

\subsubsection{The Quiet Kingdom: Cambodia in
2008}\label{the-quiet-kingdom-cambodia-in-2008}

The first act of this story is a silence, and it is important to
understand what kind of silence it was.\footnote{Unless otherwise noted,
  the political chronology and electoral data in this section draw on
  Hughes (2010); Strangio (2014); Un (2013a); and Un (2013b).}

By 2008, Cambodia had been under continuous CPP rule for over two
decades. The Paris Peace Accords of 1991 and the UNTAC-supervised
elections of 1993 had produced a brief, chaotic experiment in multiparty
democracy, but the CPP had never genuinely relinquished power, and after
a violent coup in 1997 and disputed elections in 1998 and 2003, the
party had consolidated its dominance through a combination of patronage,
institutional control, and selective coercion. The 2008 National
Assembly elections, which the CPP won with 58 of 123 seats and
subsequently expanded through coalition politics, were competitive in
form but not seriously contested in outcome.

The CPP's dominance by 2008 rested on institutional foundations that
went far beyond electoral mechanics. The party had constructed a dense
patronage network linking the central state to village-level
administration, with commune chiefs serving as the capillary system
through which resources, information, and political loyalty
flowed.\footnote{On the CPP's patronage architecture, see Caroline
  Hughes and Kheang Un, \emph{Cambodia's Economic Transformation}
  (Copenhagen: NIAS Press, 2011); Un (2012).} The military and security
services were thoroughly integrated into the party apparatus, and the
judiciary operated as an instrument of executive power rather than an
independent check upon it.\footnote{Blake (2019); Morgenbesser (2016).}
Economic growth, driven by the garment export sector, construction, and
an expanding tourism industry, provided a material foundation for the
regime's implicit social contract: stability and modest prosperity in
exchange for political acquiescence.\footnote{Hughes \& Un (2011)}

The opposition landscape was fragmented and demoralized. The Sam Rainsy
Party (SRP), the most prominent non-CPP formation, commanded a loyal but
limited urban constituency and suffered from periodic leadership crises
as Sam Rainsy himself cycled between exile and return. The Human Rights
Party, led by Kem Sokha, drew support from a distinct base but lacked
the organizational reach to challenge the CPP's rural dominance. Neither
party, operating independently, presented a credible threat to CPP
hegemony---a structural condition that would change dramatically with
their merger four years later.\footnote{Noren-Nilsson (2015) provides
  the most detailed account of the opposition landscape in this period
  and the narratives that would eventually fuel the CNRP's formation.}

The Wave 2 Asian Barometer data from this period paint a portrait of a
population that was neither enthusiastic nor resistant, but simply
quiet.\footnote{Unless otherwise noted, the following scale conventions
  apply throughout: gate participation items are reported as the
  percentage of respondents who ever engaged in each activity (binary:
  ever vs.~never); community leader contact is reported as a mean on a
  1--5 scale (1 = never, 5 = often); authoritarian governance
  preferences on a 1--4 scale (1 = very bad, 4 = very good); democratic
  assessments on a 1--10 scale; corruption perceptions on a 1--4 scale
  (1 = hardly anyone, 4 = almost everyone); and media and political
  interest items on a 1--4 scale (1 = none, 4 = a great deal). Internet
  news frequency is harmonized to a 1--6 scale (1 = never, 6 = daily or
  more) across waves with differing original response categories; see
  the dedicated footnote in the Wave 3 discussion for details. Witnessed
  corruption is reported as a binary proportion. Voter turnout is
  self-reported and expressed as a percentage.} Political participation
was present but low in intensity: 43 percent of respondents reported
ever having contacted a person of influence, yet community leader
contact averaged just 1.53 on a five-point frequency scale---figures
that capture a population with prior exposure to civic channels but with
limited ongoing engagement.\footnote{Direct contact with elected
  officials and civil servants uses a broadly-worded `at any level'
  framing in Wave 2 that differs from the narrower, problem-motivated
  framing in Wave 3 and from the Wave 4 and 6 formulations; these items
  are not used for cross-wave comparisons involving Wave 2 but are
  reported in the four-wave tables with appropriate documentation.}

\begin{table}

\caption{\label{tbl-table1}Wave 2 (2008) Baseline: Political
Orientations in Cambodia}

\centering{

\centering\begingroup\fontsize{10}{12}\selectfont

\begin{threeparttable}
\begin{tabular}[t]{lc}
\toprule
Variable & W2 (2008)\\
\midrule
\addlinespace[0.3em]
\multicolumn{2}{l}{\textbf{Political Participation}}\\
\hspace{1em}Contacted elected official & 32.6\% (29.7--35.6)\\
\hspace{1em}Contacted civil servant & 31.8\% (29.0--34.8)\\
\hspace{1em}Contacted influential person & 43.0\% (39.9--46.2)\\
\hspace{1em}Signed petition & ---\\
\hspace{1em}Attended demonstration & ---\\
\hspace{1em}Contacted media & ---\\
\hspace{1em}Contacted community leader (mean, 1–5) & 1.53 (1.49--1.57)\\
\hspace{1em}Voted in last election & ---\\
\addlinespace[0.3em]
\multicolumn{2}{l}{\textbf{Authoritarian Preferences (1–4: very bad to very good)}}\\
\hspace{1em}Expert rule & —\\
\hspace{1em}Single-party rule & 1.79 (1.72--1.87)\\
\hspace{1em}Strongman rule & 1.60 (1.54--1.67)\\
\hspace{1em}Military rule & 2.19 (2.11--2.27)\\
\addlinespace[0.3em]
\multicolumn{2}{l}{\textbf{Democratic Expectations (0–10 scale)}}\\
\hspace{1em}Democratic future (10pt) & —\\
\hspace{1em}Democratic past (10pt) & —\\
\hspace{1em}Democratic present (10pt) & 6.72 (6.47--6.96)\\
\addlinespace[0.3em]
\multicolumn{2}{l}{\textbf{Corruption}}\\
\hspace{1em}Witnessed corruption (binary) & 27.7\% (24.9--30.6)\\
\hspace{1em}National govt corruption (1–4) & 2.86 (2.81--2.92)\\
\hspace{1em}Local govt corruption (1–4) & 2.56 (2.51--2.61)\\
\addlinespace[0.3em]
\multicolumn{2}{l}{\textbf{Media \& Political Interest}}\\
\hspace{1em}Follows political news & —\\
\hspace{1em}Internet news (1–6) & —\\
\hspace{1em}Political interest & 2.56 (2.50--2.62)\\
\hspace{1em}Discusses politics & 1.65 (1.61--1.69)\\
\bottomrule
\end{tabular}
\begin{tablenotes}[para]
\item \textit{Note:} 
\item Wave 2 N = 1,000. Participation rows 1–6 report the percentage of respondents who ever engaged in each activity (gate question); 95\textbackslash{}\textbackslash{}\% CI (Wilson interval) in parentheses. Rows 7–8 report means (community leader contact: 1–5 scale) or proportions (voted); 95\textbackslash{}\textbackslash{}\% CI in parentheses. '—' indicates item not fielded in this wave. Authoritarian preferences: 1 = very bad, 4 = very good. Democratic expectations: 0–10. Corruption witnessed: binary proportion. National/local corruption: 1 = not at all, 4 = extremely.
\end{tablenotes}
\end{threeparttable}
\endgroup{}

}

\end{table}%

The authoritarian attitude measures are perhaps the most telling.
Single-party rule was rated at 1.79 on a four-point scale (where 4
represents ``very good''), strongman rule at 1.6, and military rule at
2.19. These are not the numbers of a population that embraces
authoritarianism as an ideal; they are the numbers of a population that
has learned to live within it. The slightly elevated tolerance for
military rule is consistent with the CPP's own origins in
Vietnamese-backed military intervention and the residual security
anxieties of a post-genocide society, but even this measure sits below
the scale midpoint.

What is most notable about the 2008 data, viewed in retrospect, is how
stable everything appears. There are no dramatic spikes, no anomalous
values, no signs of the upheaval to come. Cambodia in 2008 was a country
in political equilibrium---not a democratic equilibrium, but not one
characterized by active authoritarian mobilization either. It was, to
extend the metaphor, a kingdom in which the subjects had learned that
the king was permanent and had arranged their lives accordingly. The
question that the next four years would pose, with startling force, was
what would happen when that arrangement was briefly and dramatically
disrupted.

\subsubsection{The Awakening: Cambodia in
2012}\label{the-awakening-cambodia-in-2012}

Something extraordinary happened in Cambodia between 2008 and 2012, and
the Wave 3 Asian Barometer data capture its imprint with unusual
clarity.\footnote{Unless otherwise noted, the political chronology and
  electoral data in this section draw on Un (2013b); Um (2014); Strangio
  (2014); and Noren-Nilsson (2015). On the 2013 election results
  specifically, see COMFREL (The Committee for Free and Fair Elections
  in Cambodia) (2013).}

Across nearly every measure of civic engagement and political
aspiration, Cambodian respondents in 2012 registered at or near the
highest values in the dataset's history. Contact with influential
persons---the gate item with consistent framing across waves---rose from
43 percent in 2008 to 52.6 percent. Community leader contact averaged
1.9 on a five-point frequency scale, up from 1.53. And the political
activities newly tracked beginning in Wave 3---attending demonstrations
(2.6 percent ever) and signing petitions (10.5 percent ever)---debuted
at notable levels consistent with a population in active political
motion.

These are not the numbers of a quiescent population. They are the
numbers of a population that had been politically activated.

The catalyst was the formation of the Cambodia National Rescue Party in
2012, a merger of the Sam Rainsy Party and the Human Rights Party under
the joint leadership of Sam Rainsy and Kem Sokha. The CNRP did something
that no previous Cambodian opposition had managed: it unified the
fragmented non-CPP political space into a single, credible alternative
with a coherent message of democratic reform, anti-corruption, and
economic justice. The party drew its energy from an emerging urban youth
demographic---the first generation of Cambodians born after the Khmer
Rouge, educated in the expanding university system, connected to the
wider world through social media, and impatient with the political
settlement their parents had accepted.

The CNRP's formation in July 2012 was itself a product of political
learning. Previous opposition efforts had foundered on precisely the
fragmentation that the CPP exploited: multiple parties splitting the
non-CPP vote, personal rivalries preventing coordination, and the
absence of a unified organizational infrastructure capable of matching
the CPP's commune-level reach. The merger of the SRP and the Human
Rights Party, brokered through extended negotiations between Sam Rainsy
and Kem Sokha, resolved the coordination problem at a stroke.\footnote{Noren-Nilsson
  (2015); Un (2013a).} The resulting party offered voters something
genuinely new in Cambodian politics: a single, unified alternative to
the CPP with recognizable leaders, a coherent platform,
and---critically---the organizational capacity to mobilize supporters in
both urban and peri-urban constituencies.

The mobilization dynamics visible in the 2012 survey data intensified
dramatically in the months that followed. The July 2013 national
election became the most competitive in Cambodia's post-UNTAC history,
driven by a potent combination of youth demographics, social media
penetration, and economic grievance. Facebook, which had reached an
estimated 1.5 million Cambodian users by 2013, served as both an
organizational tool and an alternative information ecosystem that
partially circumvented CPP control of broadcast media.\footnote{Norén-Nilsson
  (2021a) on Fresh News and Cambodia's digital media landscape. On the
  broader role of social media in authoritarian contexts, see
  Morgenbesser (2020).} The CNRP's campaign messaging---centered on
anti-corruption, land rights, wages for garment workers, and a promise
to bring genuine democratic accountability---resonated with a generation
of voters who had no memory of the Khmer Rouge and little patience for
the stability-first bargain their parents had accepted.\footnote{Norén-Nilsson
  (2021b) on youth mobilization and power reproduction in Cambodia's
  authoritarian system.}

The election results shocked the political establishment. The CNRP won
55 of 123 National Assembly seats, reducing the CPP's margin to its
narrowest since 1993. Allegations of systematic electoral fraud
triggered months of mass protest in Phnom Penh, with demonstrations
regularly drawing crowds estimated in the tens of thousands. For
observers and participants alike, the political settlement that had
governed Cambodia since the late 1990s appeared to be
unraveling.\footnote{Strangio (2014); Morgenbesser (2017).}

The democratic expectation measures tell the same story from a different
angle. When asked to rate the democratic future of their country on a
ten-point scale, Cambodian respondents in 2012 produced a mean of 9.58.
This is, by any standard, a remarkably high figure, indicating
near-universal optimism. It suggests a population that had collectively
decided that democratic change was not merely desirable but
imminent---that the political order they had lived under for decades was
on the verge of fundamental transformation.

The paradox, of course, is that this expectation was not based on any
structural change in the regime. The CPP still controlled the state
apparatus, the military, the courts, the media, and the vast patronage
networks that sustained its rural base. Hun Sen showed no inclination to
reform or retire. What had changed was not the regime but the perception
of the regime's permanence---and that perceptual shift, as the
subsequent waves would demonstrate, proved far more fragile than the
political moment suggested.

The specific configuration of the democratic assessment triad in 2012 is
theoretically revealing. The past-present-future structure---low past
(3.97), middling present (5.85), exceptionally high future (9.58)---is
not the signature of a population evaluating its democratic institutions
on the basis of lived experience. A population grounded in experiential
assessment would be expected to rate the future as a modest
extrapolation from the present, not as a near-perfect score that exceeds
the present by nearly four points. What the 2012 configuration captures
instead is a hope-driven democratic orientation: the conviction that the
current political order is transitional and that the democratic
destination is both certain and imminent. This orientation, as
subsequent waves would reveal, proved extraordinarily sensitive to
political events---precisely because it was grounded in expectation
rather than experience.

Even the corruption data reflect the mobilization moment. Witnessed
corruption stood at 49 percent in Wave 3, up from 28 percent in Wave 2.
This increase is almost certainly not an increase in actual
corruption---the CPP's patronage system was well-established long before
2012---but rather reflects a population that had become more willing to
identify and report corruption as a political problem, emboldened by the
opposition's anti-corruption messaging and the expanded civic space that
accompanied the CNRP's rise.

There are also subtler patterns in the Wave 3 data that resist easy
interpretation. Traditional values measures---obedience to parents,
deference to teacher authority---also peaked in this wave, an unexpected
finding given that the broader mobilization was associated with youth
activism and challenges to established authority. One possibility is
that the CNRP's own messaging, which blended democratic aspiration with
Khmer cultural conservatism and Buddhist moral frameworks, activated
traditional values alongside rather than in opposition to democratic
ones. Another is that the elevated values reflect a general
intensification of political and social engagement in which all
expressed attitudes, whether progressive or traditional, registered at
higher levels. A third possibility, which cannot be ruled out, is survey
context effects: the charged political environment of 2012 may have
produced a general social desirability bias toward expressing strong
opinions of any kind.

The media and political engagement measures complete the portrait of a
mobilized citizenry. Political interest stood at 2.57 on a four-point
scale, and following political news registered at 3.07---both the
highest values in the Cambodian ABS series. Political discussion, at
1.45, remained modest, suggesting that the activation captured in 2012
was more a matter of individual attention and institutional engagement
than of deliberative civic culture. Internet news consumption,
harmonized to a common six-point frequency scale across waves, stood at
1.22---in the ``never to hardly ever'' range, consistent with the still
very limited internet penetration of Cambodia in 2012.\footnote{The ABS
  internet news item underwent substantial changes in response scale
  across waves (6-point in W2/W3, 8-point in W4, 9-point in W5/W6). The
  values reported here reflect harmonization to a common 1--6 frequency
  scale (1 = never, 6 = daily or more). The coarser W2/W3 scale cannot
  distinguish within-daily usage patterns available in later waves, but
  the harmonized values are comparable at the level of weekly, monthly,
  and annual frequency.}

The 2012 data, taken together, present a portrait of a political
community at its moment of maximum activation. Every dimension the Asian
Barometer measures---participation, aspiration, attention, even the
willingness to name corruption---points in the same direction. It is
tempting, with the knowledge of what came after, to read this as the
high-water mark of Cambodian democracy, the moment before the tide
turned. But that reading, while emotionally satisfying, may be too
simple. What the Wave 3 data actually capture is something rarer and
more theoretically interesting: the approximate moment at which a
population collectively updated its beliefs about what was politically
possible, right before the regime demonstrated, with devastating
effectiveness, that those beliefs were wrong.

\subsubsection{The Reckoning: Cambodia in
2015}\label{the-reckoning-cambodia-in-2015}

That demonstration began at the ballot box and ended in the
streets.\footnote{Unless otherwise noted, the political chronology in
  this section draws on Chheang (2015); Morgenbesser (2017); and
  Norén-Nilsson (2019).} The 2013 national election, held just months
after the Wave 3 survey captured such exceptional optimism, confirmed
the CNRP's mobilization power: the opposition won 55 seats to the CPP's
68, the narrowest margin in the country's modern history. Yet, the
structural ceiling of the dominant-party state held firm. Amid
widespread allegations of electoral fraud, the CNRP boycotted
parliament, triggering months of mass demonstrations that soon merged
with massive garment worker strikes. For a brief window, genuine
political transformation appeared imminent. It did not arrive. Following
a lethal state crackdown on protesters in January 2014, and a subsequent
political compromise that coaxed the opposition into the National
Assembly under a fragile and temporary ``Culture of Dialogue,'' the
momentum of the streets fractured.

The ``Culture of Dialogue'' that emerged from the July 2014 agreement
between Sam Rainsy and Hun Sen was, in retrospect, less a genuine
political opening than a controlled decompression. The CNRP entered
parliament and gained access to committee chairmanships, lending the
arrangement a veneer of power-sharing. But the underlying dynamics had
not changed: the CPP retained control of the security forces, the
judiciary, and the National Election Committee, while the CNRP's
capacity for mass mobilization---its primary source of leverage---was
progressively constrained through targeted legal harassment of its
leaders and selective restrictions on public assembly.\footnote{Morgenbesser
  (2017); Norén-Nilsson (2019). On the broader pattern of managed
  political openings in dominant-party systems, see Schedler (2013).} By
2015, the dialogue had largely stalled, and the political atmosphere had
shifted from the hopeful if chaotic energy of 2013 to something more
wary and uncertain. It was in this atmosphere that the Wave 4 Asian
Barometer survey entered the field.

The Wave 4 data from 2015 capture a population in the midst of this
reckoning---no longer at the peak of mobilization but not yet in the
trough of withdrawal. The behavioral participation measures tell a more
ambiguous story than might be expected. Contact with influential persons
held essentially steady at 55.9 percent (the 3.3 percentage-point
increase from Wave 3 is not statistically significant; \emph{p} = 0.11),
and demonstration attendance, petition-signing, and civil servant
contact all edged slightly upward from their Wave 3 levels. Community
leader contact averaged 2 on a five-point frequency scale, essentially
unchanged from 2012. The ``selective retreat'' of 2015 is not, in other
words, a story written primarily in the behavioral data. It is written
instead in the attitudinal record.

\begin{table}

\caption{\label{tbl-table2}Selective Retreat, 2012--2015: Wave 3 to Wave
4 Comparison}

\centering{

\centering\begingroup\fontsize{10}{12}\selectfont

\begin{threeparttable}
\begin{tabular}[t]{lccc}
\toprule
Variable & W3 (2012) & W4 (2015) & \Delta (W4−W3)\\
\midrule
\addlinespace[0.3em]
\multicolumn{4}{l}{\textbf{Political Participation}}\\
\hspace{1em}Contacted elected official & 4.1\% (3.1--5.4) & 4.5\% (3.4--5.9) & +0.4 pp\\
\hspace{1em}Contacted civil servant & 6.6\% (5.3--8.2) & 11.0\% (9.3--12.9) & +4.4 pp\\
\hspace{1em}Contacted influential person & 52.6\% (49.7--55.5) & 55.9\% (53.1--58.7) & +3.3 pp\\
\hspace{1em}Signed petition & 10.5\% (8.9--12.4) & 14.5\% (12.6--16.6) & +4.0 pp\\
\hspace{1em}Attended demonstration & 2.6\% (1.8--3.7) & 4.5\% (3.5--5.9) & +2.0 pp\\
\hspace{1em}Contacted media & 3.2\% (2.3--4.4) & 3.4\% (2.4--4.6) & +0.2 pp\\
\hspace{1em}Contacted community leader (mean, 1–5) & 1.90 (1.85--1.96) & 2.00 (1.95--2.06) & +0.10\\
\hspace{1em}Voted in last election & 78.7\% (76.2--80.9) & 83.2\% (80.9--85.2) & +4.5 pp\\
\addlinespace[0.3em]
\multicolumn{4}{l}{\textbf{Authoritarian Preferences (1–4)}}\\
\hspace{1em}Expert rule & 1.58 (1.53--1.63) & 1.66 (1.62--1.70) & +0.08\\
\hspace{1em}Single-party rule & 1.97 (1.91--2.03) & 1.78 (1.73--1.83) & -0.19\\
\hspace{1em}Strongman rule & 1.73 (1.67--1.78) & 1.76 (1.71--1.81) & +0.03\\
\hspace{1em}Military rule & 2.19 (2.12--2.25) & 1.98 (1.93--2.04) & -0.21\\
\addlinespace[0.3em]
\multicolumn{4}{l}{\textbf{Democratic Expectations (0–10)}}\\
\hspace{1em}Democratic future (10pt) & 9.58 (9.48--9.68) & 7.72 (7.48--7.96) & -1.86\\
\hspace{1em}Democratic past (10pt) & 3.97 (3.85--4.09) & 3.87 (3.75--3.98) & -0.10\\
\hspace{1em}Democratic present (10pt) & 5.85 (5.66--6.04) & 5.06 (4.89--5.23) & -0.80\\
\addlinespace[0.3em]
\multicolumn{4}{l}{\textbf{Corruption}}\\
\hspace{1em}Witnessed corruption (binary) & 49.4\% (46.5--52.3) & 62.8\% (59.9--65.7) & +13.4 pp\\
\hspace{1em}National govt corruption (1–4) & 2.67 (2.62--2.72) & 2.90 (2.86--2.94) & +0.23\\
\hspace{1em}Local govt corruption (1–4) & 2.47 (2.42--2.52) & 2.56 (2.52--2.61) & +0.09\\
\addlinespace[0.3em]
\multicolumn{4}{l}{\textbf{Media \& Political Interest}}\\
\hspace{1em}Follows political news & 3.07 (2.99--3.14) & 2.86 (2.78--2.94) & -0.21\\
\hspace{1em}Internet news (1–6) & 1.22 (1.16--1.27) & 1.82 (1.72--1.92) & +0.61\\
\hspace{1em}Political interest & 2.57 (2.52--2.62) & 2.29 (2.24--2.34) & -0.28\\
\hspace{1em}Discusses politics & 1.45 (1.42--1.49) & 1.47 (1.44--1.50) & +0.02\\
\bottomrule
\end{tabular}
\begin{tablenotes}[para]
\item \textit{Note:} 
\item W3 N = 1,200; W4 N = 1,200. Participation rows 1–6 report percentages (gate question); 95\% CI (Wilson interval) in parentheses. All other rows report means or binary proportions with 95\% CI (t-based) in parentheses. $\Delta$ for participation rows is in percentage points (pp); for other rows, in scale units. '—' indicates item not fielded. Internet news harmonized to 1–6 frequency scale (1 = never, 6 = daily or more).
\end{tablenotes}
\end{threeparttable}
\endgroup{}

}

\end{table}%

This attitudinal disengagement was not yet accompanied by a
corresponding rise in authoritarian acceptance. Single-party rule
actually dipped slightly from 1.97 to 1.78, and military rule fell from
2.19 to 1.98. Expert rule and strongman rule edged up marginally but
remained below 2.0. This pattern is analytically important: it suggests
that the demobilization observed between 2012 and 2015 was driven more
by disillusionment and tactical withdrawal than by a genuine
reorientation toward authoritarian preferences. Cambodians were pulling
back from political action, but they had not yet accepted the
authoritarian order as normatively appropriate.

The democratic expectation measures, however, had already begun their
descent. The democratic future rating fell from 9.58 to 7.72---a drop of
nearly two full points on the ten-point scale (\emph{p} \textless{}
0.001). The democratic present, as assessed through government
performance, declined from 5.85 to 5.06. Only the democratic past held
relatively steady at 3.87. What this configuration suggests is that the
hope that had animated the 2012 moment was already eroding, not because
Cambodians had embraced an alternative vision, but because the events of
2013-2014---the election that changed nothing, the protests that were
suppressed, the deal that left the CPP in power---had begun to teach a
lesson about the limits of popular mobilization within a dominant-party
system.

The corruption data from Wave 4 introduce a striking counterpoint.
Witnessed corruption surged to its highest level in the dataset: 63
percent of respondents reported having personally encountered
corruption. This is unlikely to reflect merely increased willingness to
report; the political environment of 2015 was, if anything, less
permissive of anti-government speech than 2012. The more likely
explanation is that the post-2013 period, with its political bargaining,
institutional jockeying, and expanded CNRP presence in parliament,
created more visible sites of corruption---more interactions between
citizens and competing political actors, more opportunities to observe
the transactional nature of Cambodian governance up close.

The informational dimensions of political life registered the earliest
signs of the withdrawal that would accelerate dramatically by 2021.
Political interest declined from 2.57 to 2.29, and following political
news fell from 3.07 to 2.86---modest drops in absolute terms, but
significant as the leading edge of a trend that would deepen sharply.
Political discussion held essentially flat at 1.47, suggesting that
Cambodians were beginning to disengage from formal political information
channels while maintaining informal conversational patterns. Internet
news consumption, meanwhile, continued its modest climb from 1.22 to
1.82, reflecting the early stages of broadening internet access even as
political engagement with that access was beginning to wane.

The 2015 wave, viewed within the larger narrative, represents the hinge
point---the moment when the trajectory could still have gone either way.
The CNRP was weakened but present, its leaders alternately accommodated
and harassed, its supporters disillusioned but not yet defeated. The
data register this ambiguity: democratic hope was falling but had not
crashed. Behavioral participation remained largely intact while
attitudinal engagement had already begun its descent. Authoritarian
acceptance had not yet risen. Corruption was more visible than ever. It
was a moment of unstable equilibrium, poised between the democratic
aspiration of 2012 and the authoritarian normalization that was about to
descend.

\subsubsection{The Silence: Cambodia in
2021}\label{the-silence-cambodia-in-2021}

On November 16, 2017, the Supreme Court of Cambodia dissolved the
Cambodia National Rescue Party.\footnote{Unless otherwise noted, the
  political chronology in this section draws on Un \& Luo (2020);
  Chheang \& President of the Asian Vision Institute (AVI), an
  independent think-tank based in Phnom Penh, Cambodia. He is a public
  policy analyst specializing in geopolitics and political economy of
  Southeast Asia. (2021); and Morgenbesser \& Pepinsky (2019).} The
ruling, which cited a vaguely defined conspiracy to overthrow the
government, banned 118 CNRP officials from political activity for five
years and transferred the party's parliamentary seats to smaller,
CPP-aligned parties. Kem Sokha had already been arrested in September;
Sam Rainsy remained in exile. In the July 2018 national election, the
CPP won all 125 seats in the National Assembly. Cambodia became, for all
practical purposes, a one-party state.

The CNRP's dissolution did not occur in isolation. It was the
centerpiece of a broader campaign of political closure that
systematically dismantled the infrastructure of independent civic life.
The \emph{Cambodia Daily}, the country's most prominent English-language
newspaper and a persistent thorn in the CPP's side, was forced to close
in September 2017 after receiving a disputed tax bill of \$6.3
million.\footnote{The Cambodia Daily's closure was widely interpreted as
  politically motivated. See Norén-Nilsson (2021a) on the restructuring
  of Cambodia's media landscape during this period.} Radio Free Asia's
Cambodian bureau was shuttered, and several of its journalists faced
criminal charges. Independent radio stations that had broadcast
opposition content were pressured to switch to pro-government
programming or cease operations entirely. Civil society organizations
faced intensified regulatory scrutiny under the 2015 Law on Associations
and Non-Governmental Organizations (LANGO), which gave the government
broad discretionary power to dissolve organizations deemed to threaten
public order.\footnote{On LANGO and its effects on civil society space,
  see Norén-Nilsson \& Eng (2020).}

The combined effect was the elimination not merely of the political
opposition but of the broader ecosystem---media, civil society, informal
networks---through which citizens had accessed alternative political
information and organized collective action. The transition, in Levitsky
and Way's typology, was from competitive authoritarianism, in which
opposition parties exist and can occasionally win, to hegemonic
authoritarianism, in which the electoral arena is formally maintained
but substantively emptied.\footnote{Levitsky \& Way (2010); Schedler
  (2002). Morgenbesser \& Pepinsky (2019) applies this framework
  specifically to Cambodia's post-2017 trajectory.} The 2018 national
election, conducted without the CNRP on the ballot, produced a CPP sweep
of all 125 National Assembly seats---a result that formalized what the
dissolution had already accomplished.

The Wave 6 data from 2021 record the consequences.

The participation data from 2021 present a more differentiated picture
than a simple collapse narrative would suggest. Contact with influential
persons---the item most directly measuring independent civic
engagement---fell sharply, from 55.9 percent in 2015 to 23.3 percent
(\emph{p} \textless{} 0.001), a 33 percentage-point decline that
represents one of the steepest single-item shifts in the dataset.
Community leader contact fell to a mean of 1.39 on the frequency scale,
and political discussion dropped to 1.26 on a four-point scale---a
population that had largely stopped talking about politics with one
another. Demonstration attendance held near its 2015 level at 3.3
percent.

What complicates any uniform ``participation collapse'' characterization
is that several formal channel indicators actually rose substantially by
2021: petition-signing reached 30.5 percent, civil servant contact
climbed to 37.2 percent, and media contact rose to 12.2 percent. These
increases likely reflect two overlapping dynamics: the rapid expansion
of internet access enabling digital petitions and online contact, and
the transformation of formal institutional channels into vehicles for
routine patronage interaction under a now-uncontested regime.
Participation through officially sanctioned channels rises precisely as
independent civic action---contacting influential figures outside the
state apparatus, organizing demonstrations, discussing
politics---contracts. The divergence between these two trajectories is
itself evidence of demobilization by subtraction: what citizens can
still do expands; what citizens choose to do in pursuit of genuine
political alternatives collapses.

\begin{table}

\caption{\label{tbl-table3}Four-Wave Trajectory of Political
Orientations in Cambodia, 2008--2021}

\centering{

\centering\begingroup\fontsize{10}{12}\selectfont

\resizebox{\ifdim\width>\linewidth\linewidth\else\width\fi}{!}{
\begin{threeparttable}
\begin{tabular}[t]{lccc>{}cc}
\toprule
Variable & W2 (2008) & W3 (2012) & W4 (2015) & \textbf{W6 (2021)} & \Delta W3{\textrightarrow}W6\\
\midrule
\addlinespace[0.3em]
\multicolumn{6}{l}{\textbf{Political Participation}}\\
\hspace{1em}Contacted elected official & 32.6\% (29.7--35.6) & 4.1\% (3.1--5.4) & 4.5\% (3.4--5.9) & \textbf{13.3\% (11.4--15.5)} & +9.2 pp\\
\hspace{1em}Contacted civil servant & 31.8\% (29.0--34.8) & 6.6\% (5.3--8.2) & 11.0\% (9.3--12.9) & \textbf{37.2\% (34.4--40.1)} & +30.6 pp\\
\hspace{1em}Contacted influential person & 43.0\% (39.9--46.2) & 52.6\% (49.7--55.5) & 55.9\% (53.1--58.7) & \textbf{23.3\% (20.9--25.9)} & -29.3 pp\\
\hspace{1em}Signed petition & --- & 10.5\% (8.9--12.4) & 14.5\% (12.6--16.6) & \textbf{30.5\% (27.9--33.4)} & +20.0 pp\\
\hspace{1em}Attended demonstration & --- & 2.6\% (1.8--3.7) & 4.5\% (3.5--5.9) & \textbf{3.3\% (2.3--4.6)} & +0.7 pp\\
\hspace{1em}Contacted media & --- & 3.2\% (2.3--4.4) & 3.4\% (2.4--4.6) & \textbf{12.2\% (10.3--14.2)} & +9.0 pp\\
\hspace{1em}Contacted community leader (mean, 1–5) & 1.53 (1.49--1.57) & 1.90 (1.85--1.96) & 2.00 (1.95--2.06) & \textbf{1.39 (1.34--1.43)} & -0.52\\
\hspace{1em}Voted in last election & --- & 78.7\% (76.2--80.9) & 83.2\% (80.9--85.2) & \textbf{88.1\% (86.1--89.9)} & +9.4 pp\\
\addlinespace[0.3em]
\multicolumn{6}{l}{\textbf{Authoritarian Preferences (1–4: very bad to very good)}}\\
\hspace{1em}Expert rule & — & 1.58 (1.53--1.63) & 1.66 (1.62--1.70) & \textbf{2.11 (2.06--2.16)} & +0.53\\
\hspace{1em}Single-party rule & 1.79 (1.72--1.87) & 1.97 (1.91--2.03) & 1.78 (1.73--1.83) & \textbf{2.21 (2.15--2.26)} & +0.24\\
\hspace{1em}Strongman rule & 1.60 (1.54--1.67) & 1.73 (1.67--1.78) & 1.76 (1.71--1.81) & \textbf{2.17 (2.12--2.23)} & +0.45\\
\hspace{1em}Military rule & 2.19 (2.11--2.27) & 2.19 (2.12--2.25) & 1.98 (1.93--2.04) & \textbf{2.21 (2.15--2.27)} & +0.02\\
\addlinespace[0.3em]
\multicolumn{6}{l}{\textbf{Democratic Expectations (0–10 scale)}}\\
\hspace{1em}Democratic future (10pt) & — & 9.58 (9.48--9.68) & 7.72 (7.48--7.96) & \textbf{6.67 (6.45--6.89)} & -2.91\\
\hspace{1em}Democratic past (10pt) & — & 3.97 (3.85--4.09) & 3.87 (3.75--3.98) & \textbf{4.78 (4.65--4.92)} & +0.82\\
\hspace{1em}Democratic present (10pt) & 6.72 (6.47--6.96) & 5.85 (5.66--6.04) & 5.06 (4.89--5.23) & \textbf{5.77 (5.62--5.92)} & -0.08\\
\addlinespace[0.3em]
\multicolumn{6}{l}{\textbf{Corruption}}\\
\hspace{1em}Witnessed corruption (binary) & 27.7\% (24.9--30.6) & 49.4\% (46.5--52.3) & 62.8\% (59.9--65.7) & \textbf{14.9\% (12.9--17.1)} & -34.5 pp\\
\hspace{1em}National govt corruption (1–4) & 2.86 (2.81--2.92) & 2.67 (2.62--2.72) & 2.90 (2.86--2.94) & \textbf{2.33 (2.29--2.37)} & -0.34\\
\hspace{1em}Local govt corruption (1–4) & 2.56 (2.51--2.61) & 2.47 (2.42--2.52) & 2.56 (2.52--2.61) & \textbf{2.36 (2.32--2.41)} & -0.11\\
\addlinespace[0.3em]
\multicolumn{6}{l}{\textbf{Media \& Political Interest}}\\
\hspace{1em}Follows political news & — & 3.07 (2.99--3.14) & 2.86 (2.78--2.94) & \textbf{2.04 (1.97--2.11)} & -1.03\\
\hspace{1em}Internet news (1–6) & — & 1.22 (1.16--1.27) & 1.82 (1.72--1.92) & \textbf{4.66 (4.54--4.78)} & +3.45\\
\hspace{1em}Political interest & 2.56 (2.50--2.62) & 2.57 (2.52--2.62) & 2.29 (2.24--2.34) & \textbf{2.06 (2.01--2.11)} & -0.51\\
\hspace{1em}Discusses politics & 1.65 (1.61--1.69) & 1.45 (1.42--1.49) & 1.47 (1.44--1.50) & \textbf{1.26 (1.23--1.28)} & -0.20\\
\bottomrule
\end{tabular}
\begin{tablenotes}[para]
\item \textit{Note:} 
\item N per wave: W2 = 1,000; W3 = 1,200; W4 = 1,200; W6 = 1,242. Rows 1–6 (Political Participation) report the percentage of respondents who ever engaged in each activity (gate proportion), with 95\% CI (Wilson interval) in parentheses; $\Delta$ for these rows is in percentage points. Rows 7–8 and all remaining rows report unweighted wave means with 95\% CI (t-based) in parentheses; $\Delta$ is the arithmetic difference. '---' indicates variable not fielded in that wave. $\Delta$ W3$\rightarrow$W6 = W6 value minus W3 value (peak-to-trough change). W6 column in bold marks the endpoint of the trajectory. Voted last election and corruption witnessed are binary proportions.
\end{tablenotes}
\end{threeparttable}}
\endgroup{}

}

\end{table}%

\begin{figure}[H]

{\centering \includegraphics[width=1\linewidth,height=\textheight,keepaspectratio]{../analysis/reviewer_response/fig1_with_uncertainty.pdf}

}

\caption{Political Orientations in Cambodia by Domain, 2008--2021. Point
estimates with 95\% confidence intervals (Wilson intervals for
proportions; t-based for means) across five thematic domains. Dashed
vertical line marks 2017 CNRP dissolution. Y-axes are free-scaled; each
panel uses its own metric range. Participation panel reports gate
proportions (\% ever engaged). Source: Asian Barometer Survey, Waves 2,
3, 4, 6.}

\end{figure}%

The authoritarian acceptance measures moved in mirror image. All four
alternative governance indicators rose to their highest levels: expert
rule to 2.11, single-party rule to 2.21, strongman rule to 2.17, and
military rule to 2.21. The jump was concentrated entirely in the
W4-to-W6 period---the six years that encompassed the CNRP's destruction
and the subsequent one-party election. What had been stable or declining
through 2015 now surged.

There is an important interpretive question here, and the article's
argument depends on how it is answered. Does the rise in authoritarian
acceptance represent genuine preference change---Cambodians coming to
believe that authoritarian governance is normatively desirable---or does
it represent something else? Three considerations suggest the latter.
First, the magnitude of the shift, while statistically meaningful, is
modest in absolute terms: all four measures remain around 2.2 on a
four-point scale, below the midpoint. Cambodians are not rating
authoritarianism as ``good''; they are rating it as slightly less bad
than before. Second, the timing---concentrated after the removal of the
democratic alternative---suggests a mechanism closer to adaptive
preference formation than genuine conversion. Third, and most
importantly, the shift co-occurs with the collapse of democratic future
expectations (from 7.72 to 6.67), rising perceptions of government
information withholding, and declining political interest. This is not
the attitudinal profile of a population that has found a new political
faith; it is the profile of a population that has stopped believing the
old one was achievable.

The article proposes the term authoritarian normalization to describe
this process, distinguishing it from authoritarian legitimation.
Legitimation implies that citizens have come to view the regime as
rightfully entitled to rule, whether on the basis of performance,
ideology, or tradition. Normalization implies something weaker but
potentially more durable: citizens have come to view the regime as the
only available reality and have adjusted their expressed preferences to
align with that reality. The distinction matters theoretically because
it suggests different mechanisms of regime stability---and different
vulnerabilities. A regime sustained by legitimation is threatened when
performance falters or ideology loses appeal. A regime sustained by
normalization is threatened when alternatives re-enter the political
imagination.

The normalization framework finds a useful analogue in Wedeen's analysis
of political compliance in Hafez al-Assad's Syria, where citizens
participated in elaborate public rituals of regime praise not because
they believed the regime's claims but because acting as if they believed
was the condition of ordinary life.\footnote{Wedeen (1998); Wedeen
  (2000). Wedeen's concept of ``acting as if'' captures the performative
  dimension of political compliance under conditions where genuine
  belief is neither required nor expected by the regime.} The Cambodian
case differs in important respects---the ABS data capture private survey
responses rather than public performances, and the CPP's demands on
citizens are less theatrically extravagant than the Assad cult of
personality. But the underlying logic is similar: when the political
field is restructured so that only one arrangement appears viable,
citizens adjust their expressed orientations to accommodate that
arrangement, not out of conviction but out of a practical reckoning with
the available reality. The survey data cannot definitively distinguish
between genuine preference change and adaptive accommodation, but the
pattern of evidence---modest rather than dramatic shifts in
authoritarian acceptance, coinciding with collapsing democratic
expectations and civic withdrawal---is more consistent with
normalization than with conversion.

The corruption data offer perhaps the most unsettling evidence of
normalization. Witnessed corruption collapsed from 63 percent in 2015 to
15 percent in 2021. Perceived national government corruption also
declined, from 2.9 to 2.33 on the four-point scale. On the surface,
these numbers might suggest genuine anti-corruption progress. The CPP
did conduct selective anti-corruption campaigns during this period,
targeting rivals and politically inconvenient officials. But the
magnitude of the witnessed-corruption decline---from nearly two-thirds
to one-seventh of the population---is too large to explain through
policy alone.

The more parsimonious explanation is that the corruption decline
reflects the same demobilization process visible in the participation
data. When citizens withdraw from political life---when they stop
contacting officials, stop attending meetings, stop engaging with
governance institutions---they also reduce their exposure to the sites
where corruption is experienced and observed. A population that does not
petition, does not demonstrate, and does not contact civil servants is
also a population that has fewer occasions to witness a bribe demanded
or a favor exchanged. Demobilization, in this reading, does not merely
reduce political action; it reduces the perceptual evidence that
political action might address. The regime becomes less corrupt not
because it has reformed but because citizens have stopped looking.

An alternative, and not mutually exclusive, explanation for the
corruption decline invokes the survey environment itself. In a political
climate where the opposition has been criminalized and independent media
shuttered, respondents may reasonably calculate that acknowledging
witnessed corruption---even in a nominally confidential survey---carries
risks of self-incrimination or unwanted official attention. The social
desirability bias literature has long recognized that survey responses
on sensitive topics shift in predictable directions under conditions of
political constraint, with respondents offering answers that align with
perceived official expectations.\footnote{On the reliability of survey
  responses under electoral authoritarianism, see Schedler \& Sarsfield
  (2007). Frye et al. (2016) demonstrate that even in semi-competitive
  authoritarian contexts, survey-measured approval can closely track
  genuine opinion---but their findings also imply that as political
  space closes further, the gap between expressed and private attitudes
  is likely to widen.} The post-2017 Cambodian environment, in which
public criticism of the government carried tangible legal consequences,
represents precisely the conditions under which such bias would be
expected to intensify. The two explanations---reduced exposure through
demobilization and increased reticence through social desirability
pressure---likely operate in tandem, each reinforcing the other to
produce the dramatic decline observed in the data.

The voter turnout paradox completes the portrait. Despite the collapse
of all other forms of political engagement, self-reported voting in the
last election rose from 79 percent in 2012 to 83 percent in 2015 to 88
percent in 2021. This pattern---rising electoral participation amid
collapsing civic engagement---is a well-documented feature of hegemonic
authoritarian systems, where elections serve not as mechanisms of
political choice but as rituals of regime affirmation. The 2018 election
made the ritual character of Cambodian voting unusually visible. The
CNRP-in-exile's ``clean finger'' campaign---urging voters to boycott and
display uninked fingers as proof of refusal---transformed the indelible
ink applied at polling stations from an anti-fraud measure into a tool
of social surveillance. Factory workers were warned that returning to
work without inked fingers would result in wage deductions; village
heads distributed polling slips with the expectation that residents
would comply; government employees were required to
participate.\footnote{McCargo (2018) provides a detailed firsthand
  account of the turnout-maximization dynamics surrounding the 2018
  election.} Voting in a one-party election is not a political act in
the same sense as voting in a competitive one; it is closer to a
demonstration of compliance, a public performance of belonging to the
political community as the regime defines it. The fact that turnout
rises precisely as meaningful engagement collapses suggests that
Cambodians understand the distinction even as they perform the ritual.

The international orientation data add a final dimension to the
normalization portrait. Cambodian assessments of Chinese influence in
the region grew more positive between Waves 4 and 6, rising from 2.69 to
2.88 on a four-point scale measuring perceived benefit versus harm. More
striking still, expectations of future Asian---implicitly
Chinese---regional influence climbed from 2.74 in Wave 4 to 3.3 in Wave
6, a trajectory that places Cambodia among the most China-positive
publics in the ABS dataset. This is consistent with Cambodia's deepening
alignment with Beijing during this period, visible in Belt and Road
infrastructure investment, the Ream Naval Base controversy, and
Cambodia's consistent support for Chinese positions within ASEAN. But it
also fits the normalization logic: as democratic futures recede and
Western-aligned political alternatives are eliminated, alignment with
the ascendant authoritarian power in the region becomes the natural
orientation of a population adjusting to the available political
reality. A paired comparison with Vietnam or the Philippines---where
territorial disputes drive China perceptions in sharply negative
directions despite similar economic interdependence---would sharpen this
point considerably.\footnote{Strangio (2020) provides the most
  comprehensive account of Cambodia's positioning within China's
  Southeast Asian strategy. On the divergent China perception
  trajectories across ASEAN, the ABS cross-national data offer direct
  comparative evidence.}

The informational withdrawal that began tentatively in 2015 reached its
full expression by 2021. Following political news fell to 2.04,
political interest declined to 2.06, and political discussion dropped to
1.26---the lowest values recorded across all four waves. These figures
describe a population that has not merely withdrawn from political
action but has largely ceased to attend to political information at all.
The internet news variable throws this withdrawal into particularly
sharp relief: harmonized internet news consumption rose to 4.66 by Wave
6---approaching ``at least once a week''---reflecting the rapid
expansion of internet infrastructure and smartphone penetration across
Cambodia. Cambodians in 2021 had far greater access to online
information than at any previous survey wave, yet their political
interest, news following, and willingness to discuss politics all fell
to historic lows. The infrastructure for engagement had expanded; the
will to use it for political purposes had collapsed. The
self-reinforcing logic is straightforward: citizens who do not
participate have less reason to follow political developments; citizens
who do not follow political developments have less basis on which to
participate. Demobilization, once initiated through the removal of the
political alternative, generates its own sustaining momentum through
this informational feedback loop---a dynamic that suggests the effects
of the CNRP's dissolution may prove more durable than the act itself.

The Cambodia of 2021 is, in a sense, the inverse of the Cambodia of
2012. Where the earlier moment was characterized by activation across
all dimensions---participation, aspiration, attention, even the
willingness to name corruption---the later moment is characterized by
withdrawal across all the same dimensions. The symmetry is striking and,
for the theoretical argument, essential. It suggests that what changed
between the two moments was not any single attitude or behavior but the
underlying orientation from which attitudes and behaviors derive---what
might be called the perceived horizon of political possibility. When
that horizon expanded, everything expanded with it. When it contracted,
everything contracted.

\subsubsection{Alternative Mechanisms and
Adjudication}\label{alternative-mechanisms-and-adjudication}

The preceding narrative has interpreted the synchronized attitudinal and
behavioral shifts across ABS waves as evidence of demobilization by
subtraction. Before drawing theoretical implications, it is necessary to
adjudicate this interpretation against four competing explanations:
preference falsification under intensified repression, pandemic-era
behavioral disruption, technological channel substitution, and
generational replacement. Each generates distinct observable predictions
that can be evaluated against the available data.

\emph{Preference falsification.} The most direct challenge to the
normalization interpretation holds that the observed shifts do not
reflect genuine attitude change but rather increased survey reticence
under harsher post-2017 authoritarian conditions. If respondents in 2021
were concealing pro-democratic views and understating civic
participation out of fear, the measured declines would overstate actual
demobilization. The response-style diagnostics offer partial support for
this concern: item nonresponse on democratic expectations items rose
substantially between Wave 3 and Wave 6, from 24 percent to 63 percent
on the democratic future item, and from 21 to 31 percent on the
democratic past item.\footnote{Mean nonresponse across all political
  items also rose, from approximately 5 percent in Wave 3 to 14 percent
  in Wave 6. This general pattern is consistent with either increased
  reluctance to venture political opinions or decreased political
  engagement reducing the salience of these questions.} This elevated
nonresponse is consistent with a political environment in which some
respondents feel uncomfortable offering evaluations of their country's
democratic trajectory.

Three considerations, however, limit the scope of this alternative.
First, if survey reticence were the primary driver, one would expect it
to manifest broadly---through increased straightlining, elevated ``don't
know'' responses across all political batteries, or systematic
acquiescence. The data show a more selective pattern: straightlining
rates on the authoritarian preferences battery actually declined from
Wave 4 (38 percent) to Wave 6 (35 percent), and acquiescence rates
(proportion selecting the most positive response) remained low and
stable across waves. Respondents who did answer appear to have engaged
substantively with the questions rather than defaulting to safe or
formulaic responses. Second, the demobilization pattern began before the
most intensive period of political closure: democratic future
expectations fell by nearly two points between Wave 3 and Wave 4, and
political interest declined between the same waves, well before the CNRP
dissolution and its associated repressive apparatus. If fear were the
primary mechanism, the timing should concentrate in the post-2017 period
rather than emerging incrementally from 2012 onward. Third, the placebo
battery provides the strongest evidence of domain specificity.
Non-political items that should not be affected by opposition
elimination---national pride (3.66 in Wave 3, 3.63 in Wave 4, 3.79 in
Wave 6), family economic situation (3.03, 3.05, 2.86), and perceived
economic change (3.1, 3.04, 3.27)---show no systematic decline across
waves. If generalized survey reticence were suppressing responses, these
items would be expected to drift downward as well; they do not.

The most direct empirical test of the preference-falsification
hypothesis draws on two ABS items asking respondents to rate, on a
four-point scale from strongly disagree to strongly agree, whether
``people are free to speak what they think without fear''
(\emph{dem\_free\_speech}) and ``people can join any organization they
like without fear'' (\emph{gov\_free\_to\_organize}). Perceived freedom
to organize declined steadily across waves---from a mean of 3.41 in Wave
2 to 3.21 in Wave 3, 3.1 in Wave 4, and 2.96 in Wave 6---confirming that
the perceived civic environment tightened substantially over the study
period, consistent with the documentary record. Perceived freedom of
speech was comparatively flat (W2: 2.08; W3: 2.13; W4: 2.22; W6: 2.12).
If preference falsification were the primary driver of the observed
attitudinal shifts, controlling for these freedom perceptions should
attenuate the wave-level coefficients. Adding both variables as
covariates in OLS models predicting democratic future expectations,
democracy satisfaction, and democratic preference produces virtually no
attenuation: the Wave 6 coefficient on democratic future expectations
changes by 0.4 percent; the Wave 4 coefficient by 3.3 percent.
Attenuation on democracy satisfaction (11.9 percent at Wave 4) and
democratic preference (6.7 percent at Wave 4) is similarly negligible.
Freedom perceptions absorb essentially none of the temporal variance in
democratic orientations. Furthermore, the moderation pattern runs
opposite to the preference-falsification prediction: adding an
interaction between wave and perceived freedom of speech reveals that
the declines in democratic future expectations are concentrated among
respondents who perceive \emph{more} freedom of speech---not less (Wave
4 interaction: β = -1.88, \emph{p} \textless{} .001; Wave 6: β = -1.18,
\emph{p} \textless{} .001). If survey reticence were suppressing
pro-democratic responses, the effect should concentrate among those who
feel least free to speak; instead, the deepest disillusionment runs
through precisely those who feel most free to do so---a pattern
consistent with genuine preference revision rather than strategic
concealment.

The most parsimonious reading is that the elevated nonresponse on
democratic expectation items reflects a combination of genuine
disengagement (respondents who no longer attend to politics and
therefore have no opinion to offer) and some degree of reluctance, while
the responses that are provided reflect real attitudinal shifts. This
interpretation is consistent with Frye et al.'s finding that
survey-measured approval in semi-competitive authoritarian contexts can
closely track genuine opinion, though the gap between expressed and
private attitudes likely widens as political space narrows
further.\footnote{Frye et al. (2016).}

\emph{Pandemic-era behavioral disruption.} The Wave 6 data were
collected in 2021, during the COVID-19 pandemic. Public health
restrictions on gatherings, movement, and face-to-face interaction could
independently depress participation measures, particularly those
involving physical contact such as community leader visits and
demonstration attendance. This concern is legitimate but bounded. The
key attitudinal shifts---the decline in democratic future expectations,
the rise in authoritarian governance acceptance---are not plausibly
explained by pandemic restrictions, which constrain behavior but have no
direct mechanism for altering governance preferences. The decline in
democratic expectations began in Wave 4 (2015), six years before the
pandemic, establishing a trajectory that the 2021 data continue rather
than initiate. Moreover, the participation items that most directly
involve physical contact do not show the expected pandemic signature:
demonstration attendance held essentially constant between Wave 4 and
Wave 6 (4.5 to 3.3 percent, a non-significant difference), while the
sharpest participation decline occurred in contacting influential
persons, an activity that need not involve face-to-face interaction. The
pandemic may have amplified certain behavioral measures, but it cannot
account for the breadth, direction, or temporal onset of the observed
pattern.

One placebo item does register a notable decline: interpersonal trust
fell from 14 percent (Wave 4) to 5 percent (Wave 6) on the binary ``most
people can be trusted'' item. This drop is consistent with the social
atomization documented in pandemic-era survey research across multiple
countries and may also reflect broader erosion of social cohesion under
intensified authoritarianism.\footnote{On the cross-national effects of
  COVID-19 on generalized trust, see Aassve et al. (2021).} However, the
fact that this is the only non-political item showing a substantial
decline---while national pride, family economic assessment, and
perceived economic change all remain stable---suggests a localized
effect rather than a general contamination of the survey environment.

\emph{Technological channel substitution.} The finding that
participation through formal channels rose between Wave 4 and Wave
6---petition-signing, civil servant contact, and media contact all
increased---while contact with influential persons outside the state
apparatus declined, could reflect a shift in participation modalities
driven by smartphone and internet penetration rather than political
demobilization. If citizens simply migrated from informal, face-to-face
civic engagement to digitally mediated formal channels, the resulting
pattern would resemble demobilization even if total civic engagement
were unchanged or increasing.

This interpretation is difficult to sustain for two reasons. First, it
cannot explain the concurrent attitudinal shifts: channel substitution
might redistribute participation across modes but provides no mechanism
for the simultaneous decline in democratic expectations, rise in
authoritarian acceptance, and collapse of political interest. If
citizens were merely engaging through different channels, their
attitudes toward democracy and authoritarianism should remain stable.
They did not. Second, the formal channels that expanded---petitioning,
contacting civil servants, contacting media---are precisely those most
compatible with regime-managed participation under hegemonic
authoritarianism. The rise of these channels alongside the collapse of
independent engagement is more consistent with the substitution of
participatory form for participatory substance than with genuine channel
migration.

\emph{Generational replacement.} Cambodia's young population and rapid
demographic change raise the possibility that apparent attitudinal
shifts across waves reflect cohort replacement rather than
within-individual attitude change: the politically activated youth of
2012 aged into their thirties by 2021, while a new cohort with no memory
of the CNRP mobilization entered the survey frame. If the post-2017
shifts are concentrated among younger respondents, compositional change
rather than demobilization could be driving the aggregate pattern.

The subgroup decompositions do not support this interpretation. The key
shifts between Wave 4 and Wave 6 are broad-based across age groups,
education levels, and urban/rural residence. Contact with influential
persons declined among under-30s (51.5 to 22.5 percent), 30--49
year-olds (55.8 to 24.8 percent), and over-50s (59.9 to 21.5 percent)
alike. Single-party rule acceptance rose across all three age groups.
Political interest fell uniformly regardless of age, education, or
urbanicity. If generational replacement were the primary mechanism, the
shifts should concentrate in the youngest cohort or attenuate among
older respondents who experienced the CNRP period directly; instead, the
pattern is strikingly uniform. This breadth is consistent with a
population-wide reorientation rather than a compositional artifact.

Taken together, the adjudication tests support the
demobilization-by-subtraction interpretation while acknowledging that
the 2021 data likely reflect some degree of pandemic amplification and
elevated survey reticence on the most politically sensitive items. The
core finding---synchronized, domain-specific shifts in political
orientations that began before the pandemic, cut across all demographic
subgroups, and leave non-political attitudes largely untouched---is more
consistent with genuine attitudinal adjustment to political closure than
with any single alternative mechanism.

\subsubsection{The Education}\label{the-education}

The story told in the preceding pages is, in one sense, particular to
Cambodia---a product of specific historical circumstances, a specific
regime, a specific opposition movement, and a specific act of political
closure. In another sense, it is a story with much wider application,
because the mechanism it illustrates---the restructuring of citizen
orientations through the removal of political alternatives---operates
wherever dominant-party regimes consolidate power. This concluding
section draws out four theoretical implications of the Cambodian
sequence and considers their relevance beyond the single case.

The first concerns the mechanism itself. Demobilization by subtraction,
as the Cambodian data illustrate it, is distinct from the two processes
most commonly invoked to explain political quiescence. It is not
repression-driven demobilization: while the CPP certainly employed
coercion, the attitudinal shifts observed in the ABS data---rising
authoritarian acceptance, collapsing democratic expectations, declining
political interest---are not the signature of a frightened population. A
population demobilized primarily through fear would be expected to
maintain private democratic commitments even as public participation
declined; the Cambodian data show both dimensions moving in the same
direction.\footnote{This pattern is consistent with the distinction
  between ``preference falsification,'' in which private attitudes
  diverge from public behavior, and genuine preference adjustment. The
  Cambodian case appears closer to the latter, though survey data alone
  cannot fully resolve the question. On preference falsification under
  authoritarianism, see Kuran (1995).} Nor is it co-optation: the CPP's
patronage networks were already well-established before 2012, and there
is no evidence of a dramatic expansion of distributive benefits between
2015 and 2021 that could account for the attitudinal shifts. What
changed was not the cost of participation or the rewards of compliance
but the perceived availability of an alternative---the object around
which civic mobilization had organized. When that object was removed,
the entire architecture of engagement collapsed with it.

The second implication concerns the relationship between hope and
democratic attitudes. The Wave 3 data suggest that democratic support in
authoritarian contexts may be driven less by abstract normative
commitments than by perceived feasibility. When democracy appeared
possible in 2012, every democratic indicator in the ABS surged---not
only future expectations but participation, political interest, and even
the willingness to name corruption. When democracy appeared impossible
by 2021, those same indicators collapsed. This pattern challenges a
foundational assumption in the democratic values literature: that
survey-measured democratic attitudes reflect stable, deeply held
orientations that exist independently of political
circumstances.\footnote{The canonical formulation of democratic values
  as a stable cultural attribute is found in the political culture
  tradition from Inglehart (1997) onward. The Cambodian data suggest a
  more situationally contingent model, closer to what Bratton and Mattes
  have termed ``demand for democracy'' as a function of perceived supply
  (Bratton et al., 2004; Mattes \& Bratton, 2007).} The Cambodian
evidence suggests instead that measured democratic attitudes may be
substantially a function of perceived democratic possibility---that hope
is not merely the emotional accompaniment of democratic aspiration but
its primary cognitive driver. If this is correct, cross-national surveys
that measure democratic support without accounting for the structure of
political alternatives available to respondents may be capturing
something quite different from what they intend.

The third implication concerns corruption measurement. The finding that
witnessed corruption declined precipitously alongside civic
participation suggests a troubling methodological artifact: survey-based
corruption indicators may systematically understate corruption in
demobilized populations. When citizens withdraw from engagement with
state institutions, they also withdraw from the sites where corruption
is experienced and observed. A regime that successfully demobilizes its
population will, by this logic, appear less corrupt on standard survey
measures even if actual corruption is unchanged or increasing. This has
practical consequences for the evaluation of anti-corruption campaigns
in authoritarian contexts. The CPP's selective anti-corruption efforts
during this period may well have had some genuine effect, but the
dramatic scale of the decline---from 49 percent to 15 percent (\emph{p}
\textless{} 0.001)---is far more parsimoniously explained by the
withdrawal of citizens from the institutional encounters where
corruption becomes visible. Transparency International's Corruption
Perception Index and similar instruments may be particularly vulnerable
to this dynamic in consolidating authoritarian regimes.

The fourth implication is diagnostic. The divergence between electoral
participation and civic participation---turnout rising to 88 percent
even as independent civic engagement collapsed---may serve as a
generalizable indicator of authoritarian normalization. Where elections
are maintained as regime rituals rather than competitive mechanisms,
rising turnout signals not democratic health but its opposite: the
transformation of voting from an act of political choice into an act of
political compliance.\footnote{On the ritual function of elections in
  hegemonic authoritarian systems, see Schedler (2002); Magaloni \&
  Kricheli (2010).} The Cambodian case suggests that this
turnout-participation divergence emerges specifically in the wake of
opposition elimination, and that its appearance in other
contexts---wherever rising turnout coincides with collapsing civic
engagement---should be read as evidence that the electoral arena has
been substantively emptied even if it remains formally intact.

These findings carry inherent limitations. Four cross-sectional waves
across thirteen years provide suggestive but not definitive evidence;
the absence of Waves 1 and 5 from the Cambodian ABS series limits the
precision with which structural breaks can be identified, and the
six-year gap between Wave 4 (2015) and Wave 6 (2021) means the analysis
cannot pinpoint whether the sharpest attitudinal shifts occurred
immediately following the CNRP dissolution in 2017 or accumulated
gradually. The elevated nonresponse on democratic expectation items in
Wave 6---reaching 63 percent on the democratic future measure---means
the attitudinal estimates from 2021 are based on the subset of
respondents willing to venture an opinion, which may disproportionately
exclude those with the strongest pro-democratic views or, alternatively,
those least engaged with politics. A direct test of whether fear of
speech drives this missingness is available from the ABS
freedom-of-speech and freedom-of-assembly items: a logistic regression
predicting nonresponse on the Wave 6 democratic future item, controlling
for education and urbanicity, finds that neither perceived freedom of
speech (OR = 1.01, \emph{p} = 0.95) nor perceived freedom to organize
(OR = 0.9, \emph{p} = 0.36) predicts missingness. Respondents who feel
less free to speak are no more likely to skip the democratic future
question than those who feel more free, indicating that the elevated
nonresponse rate is not the signature of a population censoring itself
out of fear. The preceding section on alternative mechanisms has
addressed the pandemic, generational, and preference-falsification
concerns in detail; the conclusion here is that these factors may
amplify specific estimates but cannot account for the overall pattern.
Finally, any single-country study faces the inherent limitation of
generalizability: the mechanisms identified here may be specific to
Cambodia's particular configuration of post-conflict politics,
dominant-party rule, and opposition dynamics.

The most productive response to this limitation is comparative
extension. The demobilization sequence observed in Cambodia---declining
democratic expectations, followed by collapsing independent civic
engagement, followed by rising authoritarian acceptance---generates a
testable temporal ordering that can be examined in other cases of
opposition elimination. Turkey after the suppression of the Peoples'
Democratic Party (HDP), Russia following the dismantling of Navalny's
organizational network, and Thailand in the years following the 2014
coup all represent contexts where credible political alternatives were
systematically removed from the political field. The Asian Barometer
dataset itself offers the most direct comparative leverage: Thailand's
post-coup trajectory and the Philippines under Duterte provide cases
where similar dynamics may be observable using identical survey
instruments. The turnout-participation divergence identified
here---rising electoral compliance amid collapsing civic
engagement---offers a particularly tractable diagnostic, as it requires
only two commonly available survey measures and generates a clear
empirical prediction. Future research should test whether the Cambodian
sequence represents a general pattern of authoritarian normalization or
a configuration specific to the conditions of this particular case.

This article began with a number: 9.6 out of 10. It is worth ending with
the question of what that number meant. In 2012, Cambodian citizens
looked at their country---a dominant-party authoritarian state with
controlled media, a compromised judiciary, and a history of political
violence---and saw a democratic future so bright they could barely
imagine anything brighter. Nine years later, they had learned otherwise.
The democratic future had not arrived. The party that promised it had
been dissolved. The leaders who championed it were in prison or exile.
And the population, having briefly imagined a different political life,
had returned to the quiet arrangement their parents had
known---participating in elections that offered no choice, accepting
governance structures they had once rejected, and gradually, measurably,
ceasing to pay attention.

Any reliance on longitudinal survey data carries inherent
vulnerabilities, and this analysis of Cambodia's trajectory is no
exception. The dramatic peaks of 2012 and the stark valleys of 2021 are
undeniably colored by the unique atmospheric conditions of those exact
moments.\footnote{Two contextual anomalies warrant specific
  acknowledgment. First, the Wave 3 (2012) survey was fielded during an
  unprecedented window of opposition mobilization just prior to the 2013
  elections, meaning the data likely capture a temporary zenith of
  democratic optimism rather than a stable baseline. Second, the Wave 6
  (2021) data were collected during the COVID-19 pandemic, a period
  where public health restrictions inherently depressed physical civic
  participation (e.g., attending demonstrations) and may have
  temporarily inflated reliance on state authority. However, the fact
  that ideological demobilization (such as the drop in democratic future
  expectations) began in Wave 4, well before the pandemic, suggests the
  2021 results reflect deeper authoritarian normalization rather than
  mere epidemiological artifacts. On the methodological challenges of
  surveying under authoritarian conditions more generally, see Schedler
  \& Sarsfield (2007). Furthermore, the single-country focus of this
  study naturally bounds its immediate generalizability, though the
  mechanisms observed here offer a framework for testing similar
  dynamics in other consolidating hegemonic regimes.} Yet, the
theoretical weight of this study does not rest on the absolute value of
any single data point, but on the architecture of the sequence. By
anchoring this narrative in within-case temporal variation---observing
the same population across four distinct acts of a remarkably
well-documented political closure---the noise of individual survey waves
gives way to a clear, structural signal. It provides an unusually clean
empirical window into how a population recalibrates its political
reality when the horizon of possibility collapses.

The fairy tale has no happy ending. But it has a lesson, and the lesson
is not about Cambodia alone. It is about the fragility of democratic
aspiration in the absence of democratic infrastructure, and about the
remarkable efficiency with which authoritarian regimes can reshape
citizen orientations simply by removing the object around which those
orientations had formed. The education of Cambodia's citizenry between
2008 and 2021 was not accomplished through indoctrination or ideology.
It was accomplished through subtraction---the removal of an alternative,
and the long silence that followed.

\newpage

\section*{References}\label{references}
\addcontentsline{toc}{section}{References}

\phantomsection\label{refs}
\begin{CSLReferences}{1}{0}
\bibitem[\citeproctext]{ref-Aassve2021-dn}
Aassve, Arnstein, Alfani, Guido, Gandolfi, Francesco, \& Le Moglie,
Marco. (2021). {Epidemics and trust: The case of the Spanish Flu}.
\emph{Health Economics}, \emph{30}, 840--857.
\url{https://doi.org/10.1002/hec.4218}

\bibitem[\citeproctext]{ref-Blake2019-ji}
Blake, David J H. (2019). {Recalling hydraulic despotism: Hun Sen's
Cambodia and the return of strict authoritarianism}. \emph{Austrian
Journal of South-East Asian Studies}, \emph{12}, 69--89.
\url{https://doi.org/10.14764/10.ASEAS-0014}

\bibitem[\citeproctext]{ref-Bratton2004-ke}
Bratton, Michael, Mattes, Robert, \& Gyimah-Boadi, E. (2004).
\emph{{Cambridge studies in comparative politics: Public opinion,
democracy, and market reform in Africa}}. Cambridge University Press.

\bibitem[\citeproctext]{ref-Chheang2015-ud}
Chheang, Vannarith. (2015). {CAMBODIA IN 2014: The Beginning of Concrete
Reforms}. \emph{Nan Yang Wen Ti Yan Qiu = Southeast Asian Affairs},
89--101.

\bibitem[\citeproctext]{ref-Chheang2021-td}
Chheang, Vannarith, \& President of the Asian Vision Institute (AVI), an
independent think-tank based in Phnom Penh, Cambodia. He is a public
policy analyst specializing in geopolitics and political economy of
Southeast Asia. (2021). {Cambodia in 2020: Regime legitimacy tested}.
\emph{Southeast Asian Affairs}, \emph{SEAA21}, 73--91.
\url{https://doi.org/10.1355/aa21-1e}

\bibitem[\citeproctext]{ref-Chu2008-ld}
Chu, Yun-Han, Choffnes, Andrew J, Diamond, Larry, Cao, Liqun, Nathan,
Andrew J, \& Shin, Doh Chull (Eds.). (2008). \emph{{How east Asians view
democracy}}. Columbia University Press.

\bibitem[\citeproctext]{ref-COMFREL-The-Committee-for-Free-and-Fair-Elections-in-Cambodia-2013-kt}
COMFREL (The Committee for Free and Fair Elections in Cambodia). (2013).
\emph{{2013 National Assembly Elections Final Assessment and Report}}
{[}Research report{]}.

\bibitem[\citeproctext]{ref-Davenport2007-ww}
Davenport, Christian. (2007). {State Repression and Political Order}.
\emph{Annual Review of Political Science}, \emph{10}, 1--23.
\url{https://doi.org/10.1146/annurev.polisci.10.101405.143216}

\bibitem[\citeproctext]{ref-Davenport2005-mk}
Davenport, Christian, Johnston, Hank, \& Mueller, Carol Mcclurg. (2005).
\emph{{Repression and Mobilization}}. University of Minnesota Press.

\bibitem[\citeproctext]{ref-Earl2011-et}
Earl, Jennifer. (2011). {Political Repression: Iron Fists, Velvet
Gloves, and Diffuse Control}. \emph{Annual Review of Sociology},
\emph{37}, 261--284.
\url{https://doi.org/10.1146/annurev.soc.012809.102609}

\bibitem[\citeproctext]{ref-Frye2016-zm}
Frye, Timothy, Gehlbach, Scott, Marquardt, Kyle L, \& Reuter, Ora John.
(2016). {Is Putin's popularity real?} \emph{Post-Soviet Affairs},
\emph{33}, 1--15. \url{https://doi.org/10.1080/1060586X.2016.1144334}

\bibitem[\citeproctext]{ref-Gandhi2008-dk}
Gandhi, Jennifer. (2008). \emph{{Political institutions under
dictatorship}}. Cambridge University Press.
\url{https://doi.org/10.1017/cbo9780511510090}

\bibitem[\citeproctext]{ref-Gaventa1980-ob}
Gaventa, John. (1980). \emph{{Power and powerlessness: Quiescence and
rebellion in an Appalachian valley}}. Oxford University Press.

\bibitem[\citeproctext]{ref-Gerschewski2013-en}
Gerschewski, Johannes. (2013). {The three pillars of stability:
legitimation, repression, and co-optation in autocratic regimes}.
\emph{Democratization}, \emph{20}, 13--38.
\url{https://doi.org/10.1080/13510347.2013.738860}

\bibitem[\citeproctext]{ref-Hu2023-ib}
Hu, Fu, Chu, Yun-Han, \& Asian Barometer Survey. (2023). \emph{{Asian
Barometer Survey, Waves 2-4, 6}}.

\bibitem[\citeproctext]{ref-Hughes2010-ns}
Hughes, Caroline. (2010). {Cambodia in 2009: The Party's Not Over Yet}.
\emph{Nan Yang Wen Ti Yan Qiu = Southeast Asian Affairs}, 85--99.

\bibitem[\citeproctext]{ref-Hughes2011-zf}
Hughes, Caroline, \& Un, Kheang (Eds.). (2011). \emph{{Cambodia's
Economic Transformation}}. NIAS Press.

\bibitem[\citeproctext]{ref-Inglehart1997-gw}
Inglehart, Ronald. (1997). \emph{{Modernization and postmodernization:
Cultural, economic, and political change in 43 societies}}. Princeton
University Press.

\bibitem[\citeproctext]{ref-Kuran1995-up}
Kuran, Timur. (1995). \emph{{Private truths, public lies: The social
consequences of preference falsification}} (2nd ed.). Harvard University
Press.

\bibitem[\citeproctext]{ref-Levitsky2010-bc}
Levitsky, Steven, \& Way, Lucan A. (2010). \emph{{Problems of
international politics: Competitive authoritarianism: Hybrid regimes
after the cold war}}. Cambridge University Press.

\bibitem[\citeproctext]{ref-Lukes2005-bc}
Lukes, Steven. (2005). \emph{{Power: A Radical View}} (2nd ed.).
Palgrave MacMillan.

\bibitem[\citeproctext]{ref-Magaloni2010-wl}
Magaloni, Beatriz, \& Kricheli, Ruth. (2010). {Political order and
one-party rule}. \emph{Annual Review of Political Science (Palo Alto,
Calif.)}, \emph{13}, 123--143.
\url{https://doi.org/10.1146/annurev.polisci.031908.220529}

\bibitem[\citeproctext]{ref-Mattes2007-ej}
Mattes, Robert, \& Bratton, Michael. (2007). {Learning about {Democracy}
in {Africa}: Awareness, {Performance}, and {Experience}}. \emph{American
Journal of Political Science}, \emph{51}, 192--217.

\bibitem[\citeproctext]{ref-McCargo2018-ms}
McCargo, Duncan. (2018, July 31). \emph{{The trouble with turnout at
Cambodia's election}}.

\bibitem[\citeproctext]{ref-Morgenbesser2017-cu}
Morgenbesser, L. (2017). {The failure of democratisation by elections in
Cambodia}. \emph{Contemporary Politics}.

\bibitem[\citeproctext]{ref-Morgenbesser2016-vf}
Morgenbesser, Lee. (2016). \emph{{Behind the Facade: Elections under
Authoritarianism in Southeast Asia}}. State University of New York
Press.

\bibitem[\citeproctext]{ref-Morgenbesser2020-kz}
Morgenbesser, Lee. (2020). \emph{{The rise of sophisticated
authoritarianism in Southeast Asia}}. Cambridge University Press.

\bibitem[\citeproctext]{ref-Morgenbesser2019-mb}
Morgenbesser, L, \& Pepinsky, T B. (2019). {Elections as causes of
democratization: Southeast Asia in comparative perspective}.
\emph{Comparative Political Studies}.

\bibitem[\citeproctext]{ref-Noren-Nilsson2015-yi}
Noren-Nilsson, Astrid. (2015). {Cambodia at a Crossroads. The Narratives
of Cambodia National Rescue Party Supporters after the 2013 Elections}.
\emph{Internationales Asienforum}, \emph{46}, 261--278.
\url{https://doi.org/10.11588/iaf.2015.46.3726}

\bibitem[\citeproctext]{ref-Noren-Nilsson2019-ck}
Norén-Nilsson, Astrid. (2019). {Kem Ley and Cambodian Citizenship Today:
Grass-Roots Mobilisation, Electoral Politics and Individuals}.
\emph{Journal of Current Southeast Asian Affairs}, \emph{38}, 77--97.
\url{https://doi.org/10.1177/1868103419846009}

\bibitem[\citeproctext]{ref-Noren-Nilsson2021-is}
Norén-Nilsson, Astrid. (2021a). {Fresh News, innovative news:
popularizing Cambodia's authoritarian turn}. \emph{Critical Asian
Studies}, \emph{53}, 89--108.
\url{https://doi.org/10.1080/14672715.2020.1837637}

\bibitem[\citeproctext]{ref-Noren-Nilsson2021-sq}
Norén-Nilsson, Astrid. (2021b). {Youth Mobilization, Power Reproduction
and Cambodia's Authoritarian Turn}. \emph{Contemporary Southeast Asia: A
Journal of International and Strategic Affairs}, \emph{43}, 265--292.

\bibitem[\citeproctext]{ref-Noren-Nilsson2020-ev}
Norén-Nilsson, Astrid, \& Eng, Netra. (2020). {Pathways to leadership
within and beyond Cambodian civil society: Elite status and
boundary-crossing}. \emph{Politics and Governance}, \emph{8}, 109--119.
\url{https://doi.org/10.17645/pag.v8i3.3020}

\bibitem[\citeproctext]{ref-Przeworski2000-ed}
Przeworski, Adam, Alvarez, Michael E, Cheibub, Jose Antonio, \& Limongi,
Fernando. (2000). \emph{{Democracy and Development: Political
Institutions and Well-Being in the World, 1950-{1990}}}. Cambridge
University Press.

\bibitem[\citeproctext]{ref-Schedler2002-eg}
Schedler, Andreas. (2002). {Elections Without Democracy: The Menu of
Manipulation}. \emph{Journal of Democracy}, \emph{13}, 36--50.
\url{https://doi.org/10.1353/jod.2002.0031}

\bibitem[\citeproctext]{ref-Schedler2013-qn}
Schedler, Andreas. (2013). \emph{{The politics of uncertainty:
Sustaining and subverting electoral authoritarianism}}. Oxford
University Press.

\bibitem[\citeproctext]{ref-Schedler2007-pv}
Schedler, A, \& Sarsfield, R. (2007). {Democrats with Adjectives:
Linking Direct and Indirect Measures of Democratic Support}.
\emph{European Journal of Political Research}, \emph{46}, 637--659.

\bibitem[\citeproctext]{ref-Strangio2014-vi}
Strangio, Sebastian. (2014). \emph{{Hun Sen's Cambodia}}. Yale
University Press.

\bibitem[\citeproctext]{ref-Strangio2020-gw}
Strangio, Sebastian. (2020). \emph{{In the Dragon's Shadow: Southeast
Asia in the Chinese Century}}. Yale University Press.

\bibitem[\citeproctext]{ref-Um2014-ft}
Um, Khatharya. (2014). {Cambodia in 2013: The Winds of Change}.
\emph{Nan Yang Wen Ti Yan Qiu = Southeast Asian Affairs}, 99--116.

\bibitem[\citeproctext]{ref-Un2012-im}
Un, Kheang. (2012). {A Thin Veneer of Change}. \emph{Asian Survey},
\emph{52}, 202--209. \url{https://doi.org/10.1525/as.2012.52.1.202}

\bibitem[\citeproctext]{ref-Un2013-ss}
Un, Kheang. (2013a). {Cambodia in 2012: Beyond the Crossroads}.
\emph{Asian Survey}, \emph{53}, 142--149.
\url{https://doi.org/10.1525/as.2013.53.1.142}

\bibitem[\citeproctext]{ref-Un2013-zu}
Un, Kheang. (2013b). {{Cambodia in 2012}: Towards developmental
authoritarianism?} In Daljit Singh (Ed.), \emph{{Southeast Asian Affairs
2013}} (pp. 73--86). ISEAS Publishing.
\url{https://doi.org/10.1355/9789814459563-009}

\bibitem[\citeproctext]{ref-Un2020-dq}
Un, Kheang, \& Luo, Jing Jing. (2020). {Cambodia in 2019: Entrenching
one-party rule and asserting national sovereignty in the era of shifting
global geopolitics}. \emph{Southeast Asian Affairs}, \emph{SEAA20},
119--136. \url{https://doi.org/10.1355/aa20-1g}

\bibitem[\citeproctext]{ref-Wedeen1998-qz}
Wedeen, Lisa. (1998). {Acting {``As If''}: Symbolic Politics and Social
Control in Syria}. \emph{Comparative Studies in Society and History},
\emph{40}, 503--523. \url{https://doi.org/10.1017/S0010417598001388}

\bibitem[\citeproctext]{ref-Wedeen2000-bg}
Wedeen, Lisa. (2000). \emph{{Ambiguities of domination: Politics,
rhetoric, and symbols in contemporary Syria}} (2nd ed.). University of
Chicago Press.

\end{CSLReferences}

\newpage

\section*{Appendix}\label{appendix}
\addcontentsline{toc}{section}{Appendix}

\subsection*{Table A1: Cross-Wave Comparability of Survey
Items}\label{table-a1-cross-wave-comparability-of-survey-items}
\addcontentsline{toc}{subsection}{Table A1: Cross-Wave Comparability of
Survey Items}

\begin{table}

\caption{\label{tbl-tableA1}Cross-Wave Comparability of Survey Items}

\centering{

\begingroup\fontsize{7}{9}\selectfont

\begin{longtabu} to \linewidth {>{\raggedright\arraybackslash}p{2.2cm}>{\raggedright\arraybackslash}p{2.8cm}>{\raggedright\arraybackslash}p{2.8cm}>{\raggedright\arraybackslash}p{2.8cm}>{\raggedright\arraybackslash}p{2.8cm}>{\raggedright\arraybackslash}p{1.4cm}}
\toprule
Variable & W2 (2008) & W3 (2012) & W4 (2015) & W6 (2021) & Status\\
\midrule
\endfirsthead
\multicolumn{6}{@{}l}{\textit{(continued)}}\\
\toprule
Variable & W2 (2008) & W3 (2012) & W4 (2015) & W6 (2021) & Status\\
\midrule
\endhead

\endfoot
\bottomrule
\multicolumn{6}{l}{\rule{0pt}{1em}\textit{Note:} Status: Full = identical or functionally equivalent wording across all waves used; Partial = minor wording shift or item not fielded in all waves (cross-wave comparison valid with caveats noted in text). No items are classified as Excluded from all comparisons; wave-pair restrictions are noted in the main text.}\\
\endlastfoot
Contacted elected official & Have you ever contacted an elected official to express your views? (broad) & Have you ever contacted an elected official to complain about or seek help with a problem? (narrow) & Have you ever contacted an elected official to express your views? (broad) & Have you ever contacted an elected official to express your views? (broad) & Partial\\
Contacted civil servant & Have you ever contacted a civil servant to express your views? (broad) & Have you ever contacted a government official to complain about or seek help with a problem? (narrow) & Have you ever contacted a civil servant to express your views? (broad) & Have you ever contacted a civil servant to express your views? (broad) & Partial\\
Contacted influential person & Have you ever contacted a person of influence? (broad) & Have you ever contacted a person of influence? (broad) & Have you ever contacted a person of influence? (broad) & Have you ever contacted a person of influence? (broad) & Full\\
Signed petition & Not collected in W2 & Have you ever signed a petition? (consistent) & Have you ever signed a petition? (consistent) & Have you ever signed a petition? (consistent) & Partial\\
Attended demonstration & Have you ever attended a demonstration or protest march? (consistent) & Have you ever attended a demonstration or protest march? (consistent) & Have you ever attended a demonstration or protest march? (consistent) & Have you ever attended a demonstration or protest march? (consistent) & Full\\
Contacted media & Have you ever contacted the media to express your views? (consistent) & Have you ever contacted the media to express your views? (consistent) & Have you ever contacted the media to express your views? (consistent) & Have you ever contacted the media to express your views? (consistent) & Full\\
Community leader contact & How often do you contact community leaders? (1–5 frequency) & How often do you contact community leaders? (1–5 frequency) & How often do you contact community leaders? (1–5 frequency) & How often do you contact community leaders? (1–5 frequency) & Full\\
Voted in last election & Did you vote in the last national election? (binary) & Did you vote in the last national election? (binary) & Did you vote in the last national election? (binary) & Did you vote in the last national election? (binary) & Full\\
Expert rule & Not collected in W2 & Should experts rather than government make decisions? (1–4) & Should experts rather than government make decisions? (1–4) & Should experts rather than government make decisions? (1–4) & Partial\\
Single-party rule & Should only one political party be allowed to contest elections? (1–4) & Should only one political party be allowed to contest elections? (1–4) & Should only one political party be allowed to contest elections? (1–4) & Should only one political party be allowed to contest elections? (1–4) & Full\\
Strongman rule & Should we get rid of parliament and elections for a strong leader? (1–4) & Should we get rid of parliament and elections for a strong leader? (1–4) & Should we get rid of parliament and elections for a strong leader? (1–4) & Should we get rid of parliament and elections for a strong leader? (1–4) & Full\\
Military rule & Should the military rule the country? (1–4) & Should the military rule the country? (1–4) & Should the military rule the country? (1–4) & Should the military rule the country? (1–4) & Full\\
Democratic future (0-10) & Not collected in W2 & How democratic will this country be in 10 years? (0–10) & How democratic will this country be in 10 years? (0–10) & How democratic will this country be in 10 years? (0–10) & Partial\\
Democratic past (0-10) & Not collected in W2 & How democratic was this country 10 years ago? (0–10) & How democratic was this country 10 years ago? (0–10) & How democratic was this country 10 years ago? (0–10) & Partial\\
Democratic present (0-10) & How democratic is this country today? (partial coverage, W2) & How democratic is this country today? (0–10 scale, government performance framing) & How democratic is this country today? (0–10 scale, government performance framing) & How democratic is this country today? (0–10 scale, government performance framing) & Partial\\
Witnessed corruption (binary) & Has any government official asked you for a bribe? (binary W2) & Have you or anyone in your family personally experienced or witnessed government corruption? (broader) & Have you or anyone in your family personally experienced or witnessed government corruption? & Have you or anyone in your family personally experienced or witnessed government corruption? & Partial\\
National govt corruption & How widespread is corruption in national government? (1–4) & How widespread is corruption in national government? (1–4) & How widespread is corruption in national government? (1–4) & How widespread is corruption in national government? (1–4) & Full\\
Local govt corruption & How widespread is corruption in local government? (1–4) & How widespread is corruption in local government? (1–4) & How widespread is corruption in local government? (1–4) & How widespread is corruption in local government? (1–4) & Full\\
Follows political news & How often do you follow political news? (1–5) & How often do you follow political news? (1–5) & How often do you follow political news? (1–5) & How often do you follow political news? (1–5) & Full\\
Internet news (harmonized) & How often do you get news from internet? (6-point scale) & How often do you get news from internet? (6-point scale) & How often do you get news from internet? (8-point scale — harmonized to 6pt) & How often do you get news from internet? (9-point scale — harmonized to 6pt) & Partial\\
Political interest & How interested are you in politics? (1–4) & How interested are you in politics? (1–4) & How interested are you in politics? (1–4) & How interested are you in politics? (1–4) & Full\\
Discusses politics & How often do you discuss politics with friends or family? (1–3) & How often do you discuss politics with friends or family? (1–3) & How often do you discuss politics with friends or family? (1–3) & How often do you discuss politics with friends or family? (1–3) & Full\\*
\end{longtabu}
\endgroup{}

}

\end{table}%

\newpage

\subsection*{Table A2: Subgroup Analysis of Key
Variables}\label{table-a2-subgroup-analysis-of-key-variables}
\addcontentsline{toc}{subsection}{Table A2: Subgroup Analysis of Key
Variables}

\begin{table}

\caption{\label{tbl-tableA2}Subgroup Analysis by Urban/Rural Residence}

\centering{

\centering\begingroup\fontsize{8}{10}\selectfont

\resizebox{\ifdim\width>\linewidth\linewidth\else\width\fi}{!}{
\begin{tabular}[t]{lll}
\toprule
Wave & Urban & Rural\\
\midrule
\addlinespace[0.3em]
\multicolumn{3}{l}{\textbf{Gate: contacted influential person}}\\
\hspace{1em}Wave 2 (2008) & 38.6\% (32.2--45.3) & 44.3\% (40.8--47.9)\\
\hspace{1em}Wave 3 (2012) & 58.3\% (51.2--65.0) & 51.4\% (48.3--54.6)\\
\hspace{1em}Wave 4 (2015) & 65.5\% (58.4--72.0) & 54.0\% (50.9--57.1)\\
\hspace{1em}Wave 6 (2021) & 29.2\% (24.1--34.8) & 21.3\% (18.6--24.2)\\
\addlinespace[0.3em]
\multicolumn{3}{l}{\textbf{Democratic future (0-10 mean)}}\\
\hspace{1em}Wave 3 (2012) & 9.48 (9.22--9.74) & 9.60 (9.49--9.71)\\
\hspace{1em}Wave 4 (2015) & 7.32 (6.74--7.89) & 7.79 (7.53--8.05)\\
\hspace{1em}Wave 6 (2021) & 6.94 (6.53--7.36) & 6.57 (6.30--6.83)\\
\addlinespace[0.3em]
\multicolumn{3}{l}{\textbf{Witnessed corruption (\textbackslash{}\%)}}\\
\hspace{1em}Wave 2 (2008) & 34.7\% (28.5--41.5) & 25.7\% (22.6--28.9)\\
\hspace{1em}Wave 3 (2012) & 56.0\% (49.0--62.9) & 48.0\% (44.9--51.2)\\
\hspace{1em}Wave 4 (2015) & 65.6\% (58.3--72.3) & 62.3\% (59.0--65.4)\\
\hspace{1em}Wave 6 (2021) & 17.6\% (13.6--22.6) & 14.0\% (11.8--16.5)\\
\addlinespace[0.3em]
\multicolumn{3}{l}{\textbf{Single-party rule (1-4 mean)}}\\
\hspace{1em}Wave 2 (2008) & 1.53 (1.40--1.67) & 1.87 (1.78--1.95)\\
\hspace{1em}Wave 3 (2012) & 1.77 (1.64--1.91) & 2.01 (1.94--2.07)\\
\hspace{1em}Wave 4 (2015) & 1.64 (1.53--1.75) & 1.81 (1.75--1.86)\\
\hspace{1em}Wave 6 (2021) & 2.09 (1.98--2.21) & 2.24 (2.18--2.31)\\
\addlinespace[0.3em]
\multicolumn{3}{l}{\textbf{Voted in last election (\textbackslash{}\%)}}\\
\hspace{1em}Wave 3 (2012) & 70.9\% (64.1--76.9) & 80.3\% (77.7--82.7)\\
\hspace{1em}Wave 4 (2015) & 80.9\% (74.6--86.0) & 83.6\% (81.2--85.8)\\
\hspace{1em}Wave 6 (2021) & 92.1\% (88.4--94.8) & 86.8\% (84.3--88.9)\\
\addlinespace[0.3em]
\multicolumn{3}{l}{\textbf{Political interest (1-4 mean)}}\\
\hspace{1em}Wave 2 (2008) & 2.35 (2.21--2.48) & 2.62 (2.55--2.68)\\
\hspace{1em}Wave 3 (2012) & 2.44 (2.32--2.56) & 2.60 (2.54--2.65)\\
\hspace{1em}Wave 4 (2015) & 2.31 (2.17--2.45) & 2.29 (2.23--2.35)\\
\hspace{1em}Wave 6 (2021) & 1.94 (1.83--2.04) & 2.10 (2.04--2.16)\\
\bottomrule
\multicolumn{3}{l}{\rule{0pt}{1em}\textit{Note:} Estimates shown with 95\% CI in parentheses (Wilson interval for proportions; t-based for means). Gate proportions and voted/corruption displayed as percentages. '---' = item not fielded in wave. Wave 2 (2008), Wave 3 (2012), Wave 4 (2015), Wave 6 (2021).}\\
\end{tabular}}
\endgroup{}

}

\end{table}%

\begin{table}

\caption{\label{tbl-tableA2b}Subgroup Analysis by Age Group}

\centering{

\centering\begingroup\fontsize{8}{10}\selectfont

\resizebox{\ifdim\width>\linewidth\linewidth\else\width\fi}{!}{
\begin{tabular}[t]{llll}
\toprule
Wave & Under 30 & 30-49 & 50+\\
\midrule
\addlinespace[0.3em]
\multicolumn{4}{l}{\textbf{Gate: contacted influential person}}\\
\hspace{1em}Wave 2 (2008) & 43.5\% (38.0--49.2) & 40.9\% (36.4--45.5) & 47.1\% (40.3--54.1)\\
\hspace{1em}Wave 3 (2012) & 50.6\% (45.6--55.7) & 53.6\% (49.2--57.9) & 53.5\% (47.5--59.4)\\
\hspace{1em}Wave 4 (2015) & 51.5\% (45.7--57.2) & 55.8\% (51.5--60.1) & 59.9\% (54.6--65.0)\\
\hspace{1em}Wave 6 (2021) & 22.5\% (17.7--28.2) & 24.8\% (21.3--28.7) & 21.5\% (17.4--26.4)\\
\addlinespace[0.3em]
\multicolumn{4}{l}{\textbf{Democratic future (0-10 mean)}}\\
\hspace{1em}Wave 3 (2012) & 9.60 (9.43--9.77) & 9.53 (9.37--9.70) & 9.62 (9.43--9.82)\\
\hspace{1em}Wave 4 (2015) & 7.84 (7.38--8.30) & 7.42 (7.05--7.79) & 8.06 (7.63--8.49)\\
\hspace{1em}Wave 6 (2021) & 6.52 (6.07--6.97) & 6.93 (6.61--7.26) & 6.35 (5.94--6.77)\\
\addlinespace[0.3em]
\multicolumn{4}{l}{\textbf{Witnessed corruption (\textbackslash{}\%)}}\\
\hspace{1em}Wave 2 (2008) & 29.7\% (24.7--35.2) & 28.2\% (24.3--32.6) & 23.3\% (17.8--29.8)\\
\hspace{1em}Wave 3 (2012) & 52.4\% (47.4--57.5) & 49.1\% (44.8--53.5) & 45.8\% (39.9--51.8)\\
\hspace{1em}Wave 4 (2015) & 59.9\% (53.9--65.6) & 67.1\% (62.7--71.1) & 58.9\% (53.3--64.3)\\
\hspace{1em}Wave 6 (2021) & 16.4\% (12.3--21.5) & 15.0\% (12.1--18.3) & 13.6\% (10.3--17.8)\\
\addlinespace[0.3em]
\multicolumn{4}{l}{\textbf{Single-party rule (1-4 mean)}}\\
\hspace{1em}Wave 2 (2008) & 1.73 (1.61--1.86) & 1.85 (1.74--1.97) & 1.73 (1.56--1.90)\\
\hspace{1em}Wave 3 (2012) & 2.02 (1.92--2.13) & 1.95 (1.85--2.04) & 1.93 (1.80--2.06)\\
\hspace{1em}Wave 4 (2015) & 1.82 (1.72--1.92) & 1.69 (1.62--1.77) & 1.87 (1.77--1.96)\\
\hspace{1em}Wave 6 (2021) & 2.30 (2.18--2.42) & 2.19 (2.12--2.27) & 2.16 (2.06--2.26)\\
\addlinespace[0.3em]
\multicolumn{4}{l}{\textbf{Voted in last election (\textbackslash{}\%)}}\\
\hspace{1em}Wave 3 (2012) & 52.4\% (47.4--57.5) & 88.9\% (85.8--91.4) & 95.8\% (92.6--97.7)\\
\hspace{1em}Wave 4 (2015) & 60.1\% (54.3--65.6) & 90.2\% (87.3--92.6) & 92.2\% (88.8--94.7)\\
\hspace{1em}Wave 6 (2021) & 80.3\% (74.9--84.8) & 90.2\% (87.4--92.5) & 90.7\% (87.1--93.4)\\
\addlinespace[0.3em]
\multicolumn{4}{l}{\textbf{Political interest (1-4 mean)}}\\
\hspace{1em}Wave 2 (2008) & 2.54 (2.44--2.65) & 2.52 (2.44--2.61) & 2.65 (2.52--2.79)\\
\hspace{1em}Wave 3 (2012) & 2.55 (2.46--2.63) & 2.55 (2.47--2.62) & 2.65 (2.54--2.76)\\
\hspace{1em}Wave 4 (2015) & 2.14 (2.04--2.24) & 2.31 (2.23--2.39) & 2.40 (2.29--2.50)\\
\hspace{1em}Wave 6 (2021) & 2.08 (1.97--2.19) & 2.04 (1.96--2.11) & 2.08 (1.99--2.17)\\
\bottomrule
\multicolumn{4}{l}{\rule{0pt}{1em}\textit{Note:} Estimates shown with 95\% CI in parentheses (Wilson interval for proportions; t-based for means). Gate proportions and voted/corruption displayed as percentages. '---' = item not fielded in wave.}\\
\end{tabular}}
\endgroup{}

}

\end{table}%

\begin{table}

\caption{\label{tbl-tableA2c}Subgroup Analysis by Education Level}

\centering{

\centering\begingroup\fontsize{8}{10}\selectfont

\resizebox{\ifdim\width>\linewidth\linewidth\else\width\fi}{!}{
\begin{tabular}[t]{llll}
\toprule
Wave & Primary or below & Secondary & Tertiary\\
\midrule
\addlinespace[0.3em]
\multicolumn{4}{l}{\textbf{Gate: contacted influential person}}\\
\hspace{1em}Wave 2 (2008) & 43.1\% (39.3--46.9) & 44.2\% (38.5--50.1) & 46.2\% (27.1--66.3)\\
\hspace{1em}Wave 3 (2012) & 51.6\% (48.0--55.1) & 54.2\% (49.1--59.3) & 57.1\% (41.1--71.9)\\
\hspace{1em}Wave 4 (2015) & 55.0\% (51.3--58.5) & 57.9\% (52.9--62.8) & 54.5\% (39.0--69.3)\\
\hspace{1em}Wave 6 (2021) & 19.9\% (16.9--23.3) & 27.2\% (23.1--31.7) & 30.3\% (20.5--42.0)\\
\addlinespace[0.3em]
\multicolumn{4}{l}{\textbf{Democratic future (0-10 mean)}}\\
\hspace{1em}Wave 3 (2012) & 9.51 (9.38--9.65) & 9.73 (9.59--9.87) & 9.38 (8.66--10.11)\\
\hspace{1em}Wave 4 (2015) & 7.93 (7.65--8.22) & 7.29 (6.83--7.74) & 7.47 (6.05--8.88)\\
\hspace{1em}Wave 6 (2021) & 6.77 (6.47--7.07) & 6.39 (6.04--6.74) & 7.34 (6.38--8.30)\\
\addlinespace[0.3em]
\multicolumn{4}{l}{\textbf{Witnessed corruption (\textbackslash{}\%)}}\\
\hspace{1em}Wave 2 (2008) & 25.2\% (22.0--28.8) & 33.2\% (27.9--39.0) & 26.9\% (12.4--48.1)\\
\hspace{1em}Wave 3 (2012) & 46.5\% (42.9--50.1) & 51.8\% (46.7--57.0) & 81.0\% (65.4--90.9)\\
\hspace{1em}Wave 4 (2015) & 59.1\% (55.3--62.7) & 69.1\% (64.1--73.7) & 70.7\% (54.3--83.4)\\
\hspace{1em}Wave 6 (2021) & 13.9\% (11.5--16.9) & 16.3\% (13.0--20.2) & 15.6\% (8.7--26.0)\\
\addlinespace[0.3em]
\multicolumn{4}{l}{\textbf{Single-party rule (1-4 mean)}}\\
\hspace{1em}Wave 2 (2008) & 1.94 (1.84--2.04) & 1.53 (1.42--1.65) & 1.19 (0.94--1.45)\\
\hspace{1em}Wave 3 (2012) & 2.12 (2.04--2.20) & 1.73 (1.62--1.83) & 1.36 (1.16--1.55)\\
\hspace{1em}Wave 4 (2015) & 1.91 (1.85--1.97) & 1.56 (1.48--1.64) & 1.53 (1.30--1.77)\\
\hspace{1em}Wave 6 (2021) & 2.22 (2.15--2.29) & 2.17 (2.08--2.25) & 2.34 (2.09--2.58)\\
\addlinespace[0.3em]
\multicolumn{4}{l}{\textbf{Voted in last election (\textbackslash{}\%)}}\\
\hspace{1em}Wave 3 (2012) & 84.9\% (82.1--87.3) & 67.7\% (62.7--72.3) & 62.5\% (45.8--76.8)\\
\hspace{1em}Wave 4 (2015) & 84.8\% (82.0--87.3) & 81.0\% (76.7--84.7) & 74.4\% (58.5--86.0)\\
\hspace{1em}Wave 6 (2021) & 87.9\% (85.2--90.2) & 87.7\% (84.1--90.5) & 92.3\% (83.4--96.8)\\
\addlinespace[0.3em]
\multicolumn{4}{l}{\textbf{Political interest (1-4 mean)}}\\
\hspace{1em}Wave 2 (2008) & 2.49 (2.42--2.57) & 2.72 (2.61--2.82) & 2.62 (2.22--3.01)\\
\hspace{1em}Wave 3 (2012) & 2.52 (2.46--2.59) & 2.65 (2.56--2.73) & 2.71 (2.44--2.99)\\
\hspace{1em}Wave 4 (2015) & 2.24 (2.18--2.31) & 2.37 (2.28--2.46) & 2.41 (2.07--2.75)\\
\hspace{1em}Wave 6 (2021) & 2.06 (1.99--2.13) & 2.07 (1.98--2.15) & 2.01 (1.79--2.23)\\
\bottomrule
\multicolumn{4}{l}{\rule{0pt}{1em}\textit{Note:} Estimates shown with 95\% CI in parentheses (Wilson interval for proportions; t-based for means). Gate proportions and voted/corruption displayed as percentages. '---' = item not fielded in wave.}\\
\end{tabular}}
\endgroup{}

}

\end{table}%

\newpage

\subsection*{Table A3: Placebo Battery --- Non-Political
Variables}\label{table-a3-placebo-battery-non-political-variables}
\addcontentsline{toc}{subsection}{Table A3: Placebo Battery ---
Non-Political Variables}

\begin{table}

\caption{\label{tbl-tableA3}Placebo Battery: Non-Political Variables by
Wave}

\centering{

\begingroup\fontsize{9}{11}\selectfont

\resizebox{\ifdim\width>\linewidth\linewidth\else\width\fi}{!}{
\begin{tabu} to \linewidth {>{\raggedright\arraybackslash}p{4.5cm}>{\raggedright}X>{\raggedright}X>{\raggedright}X>{\raggedright}X}
\toprule
label & W2 (2008) & W3 (2012) & W4 (2015) & W6 (2021)\\
\midrule
Interpersonal trust (binary) & 7.4\% (0.83) & 11.6\% (0.92) & 14.1\% (1.01) & 5.0\% (0.62)\\
Interpersonal trust (ordinal) & --- & 2.30 (0.026) & 2.44 (0.027) & 2.23 (0.022)\\
National pride & 3.60 (0.020) & 3.66 (0.017) & 3.63 (0.017) & 3.79 (0.015)\\
Pride in political system & --- & 3.03 (0.022) & 2.75 (0.024) & 2.95 (0.021)\\
HH income satisfaction & --- & 2.67 (0.029) & 1.91 (0.024) & 2.96 (0.028)\\
Family econ situation (now) & 2.92 (0.019) & 3.03 (0.020) & 3.05 (0.020) & 2.86 (0.020)\\
Family econ change (1yr) & 3.30 (0.026) & 3.10 (0.027) & 3.04 (0.024) & 3.27 (0.026)\\
National econ situation & 3.18 (0.026) & 3.33 (0.024) & 3.12 (0.023) & 3.05 (0.024)\\
Democracy satisfaction & 2.97 (0.024) & 2.98 (0.020) & 2.71 (0.023) & 3.10 (0.019)\\
\bottomrule
\multicolumn{5}{l}{\rule{0pt}{1em}\textit{Note:} Cell format: estimate (SE). Proportions (interpersonal trust) shown as percentage (SE in percentage points). Means shown as raw scale values (SE). '---' = item not fielded in wave. Items are non-political and should not be affected by opposition elimination; stable values support the domain-specificity of observed political attitude shifts.}\\
\end{tabu}}
\endgroup{}

}

\end{table}%




\end{document}
