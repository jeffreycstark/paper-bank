% Options for packages loaded elsewhere
% Options for packages loaded elsewhere
\PassOptionsToPackage{unicode}{hyperref}
\PassOptionsToPackage{hyphens}{url}
\PassOptionsToPackage{dvipsnames,svgnames,x11names}{xcolor}
%
\documentclass[
  12pt,
  letterpaper]{article}
\usepackage{xcolor}
\usepackage[margin=1in]{geometry}
\usepackage{amsmath,amssymb}
\setcounter{secnumdepth}{5}
\usepackage{iftex}
\ifPDFTeX
  \usepackage[T1]{fontenc}
  \usepackage[utf8]{inputenc}
  \usepackage{textcomp} % provide euro and other symbols
\else % if luatex or xetex
  \usepackage{unicode-math} % this also loads fontspec
  \defaultfontfeatures{Scale=MatchLowercase}
  \defaultfontfeatures[\rmfamily]{Ligatures=TeX,Scale=1}
\fi
\usepackage{lmodern}
\ifPDFTeX\else
  % xetex/luatex font selection
\fi
% Use upquote if available, for straight quotes in verbatim environments
\IfFileExists{upquote.sty}{\usepackage{upquote}}{}
\IfFileExists{microtype.sty}{% use microtype if available
  \usepackage[]{microtype}
  \UseMicrotypeSet[protrusion]{basicmath} % disable protrusion for tt fonts
}{}
\usepackage{setspace}
\makeatletter
\@ifundefined{KOMAClassName}{% if non-KOMA class
  \IfFileExists{parskip.sty}{%
    \usepackage{parskip}
  }{% else
    \setlength{\parindent}{0pt}
    \setlength{\parskip}{6pt plus 2pt minus 1pt}}
}{% if KOMA class
  \KOMAoptions{parskip=half}}
\makeatother
% Make \paragraph and \subparagraph free-standing
\makeatletter
\ifx\paragraph\undefined\else
  \let\oldparagraph\paragraph
  \renewcommand{\paragraph}{
    \@ifstar
      \xxxParagraphStar
      \xxxParagraphNoStar
  }
  \newcommand{\xxxParagraphStar}[1]{\oldparagraph*{#1}\mbox{}}
  \newcommand{\xxxParagraphNoStar}[1]{\oldparagraph{#1}\mbox{}}
\fi
\ifx\subparagraph\undefined\else
  \let\oldsubparagraph\subparagraph
  \renewcommand{\subparagraph}{
    \@ifstar
      \xxxSubParagraphStar
      \xxxSubParagraphNoStar
  }
  \newcommand{\xxxSubParagraphStar}[1]{\oldsubparagraph*{#1}\mbox{}}
  \newcommand{\xxxSubParagraphNoStar}[1]{\oldsubparagraph{#1}\mbox{}}
\fi
\makeatother


\usepackage{longtable,booktabs,array}
\usepackage{calc} % for calculating minipage widths
% Correct order of tables after \paragraph or \subparagraph
\usepackage{etoolbox}
\makeatletter
\patchcmd\longtable{\par}{\if@noskipsec\mbox{}\fi\par}{}{}
\makeatother
% Allow footnotes in longtable head/foot
\IfFileExists{footnotehyper.sty}{\usepackage{footnotehyper}}{\usepackage{footnote}}
\makesavenoteenv{longtable}
\usepackage{graphicx}
\makeatletter
\newsavebox\pandoc@box
\newcommand*\pandocbounded[1]{% scales image to fit in text height/width
  \sbox\pandoc@box{#1}%
  \Gscale@div\@tempa{\textheight}{\dimexpr\ht\pandoc@box+\dp\pandoc@box\relax}%
  \Gscale@div\@tempb{\linewidth}{\wd\pandoc@box}%
  \ifdim\@tempb\p@<\@tempa\p@\let\@tempa\@tempb\fi% select the smaller of both
  \ifdim\@tempa\p@<\p@\scalebox{\@tempa}{\usebox\pandoc@box}%
  \else\usebox{\pandoc@box}%
  \fi%
}
% Set default figure placement to htbp
\def\fps@figure{htbp}
\makeatother


% definitions for citeproc citations
\NewDocumentCommand\citeproctext{}{}
\NewDocumentCommand\citeproc{mm}{%
  \begingroup\def\citeproctext{#2}\cite{#1}\endgroup}
\makeatletter
 % allow citations to break across lines
 \let\@cite@ofmt\@firstofone
 % avoid brackets around text for \cite:
 \def\@biblabel#1{}
 \def\@cite#1#2{{#1\if@tempswa , #2\fi}}
\makeatother
\newlength{\cslhangindent}
\setlength{\cslhangindent}{1.5em}
\newlength{\csllabelwidth}
\setlength{\csllabelwidth}{3em}
\newenvironment{CSLReferences}[2] % #1 hanging-indent, #2 entry-spacing
 {\begin{list}{}{%
  \setlength{\itemindent}{0pt}
  \setlength{\leftmargin}{0pt}
  \setlength{\parsep}{0pt}
  % turn on hanging indent if param 1 is 1
  \ifodd #1
   \setlength{\leftmargin}{\cslhangindent}
   \setlength{\itemindent}{-1\cslhangindent}
  \fi
  % set entry spacing
  \setlength{\itemsep}{#2\baselineskip}}}
 {\end{list}}
\usepackage{calc}
\newcommand{\CSLBlock}[1]{\hfill\break\parbox[t]{\linewidth}{\strut\ignorespaces#1\strut}}
\newcommand{\CSLLeftMargin}[1]{\parbox[t]{\csllabelwidth}{\strut#1\strut}}
\newcommand{\CSLRightInline}[1]{\parbox[t]{\linewidth - \csllabelwidth}{\strut#1\strut}}
\newcommand{\CSLIndent}[1]{\hspace{\cslhangindent}#1}



\setlength{\emergencystretch}{3em} % prevent overfull lines

\providecommand{\tightlist}{%
  \setlength{\itemsep}{0pt}\setlength{\parskip}{0pt}}



 


\usepackage{setspace}
\usepackage{booktabs}
\usepackage{caption}
\usepackage{longtable}
\usepackage{array}
\usepackage{multirow}
\usepackage{float}
\usepackage{graphicx}
\usepackage{threeparttable}
\usepackage{threeparttablex}
\usepackage{colortbl}
\usepackage{xcolor}
\captionsetup{width=\textwidth}
\setlength\LTcapwidth{\textwidth}
\floatplacement{figure}{H}
\floatplacement{table}{H}
\raggedright
\makeatletter
\@ifpackageloaded{caption}{}{\usepackage{caption}}
\AtBeginDocument{%
\ifdefined\contentsname
  \renewcommand*\contentsname{Table of contents}
\else
  \newcommand\contentsname{Table of contents}
\fi
\ifdefined\listfigurename
  \renewcommand*\listfigurename{List of Figures}
\else
  \newcommand\listfigurename{List of Figures}
\fi
\ifdefined\listtablename
  \renewcommand*\listtablename{List of Tables}
\else
  \newcommand\listtablename{List of Tables}
\fi
\ifdefined\figurename
  \renewcommand*\figurename{Figure}
\else
  \newcommand\figurename{Figure}
\fi
\ifdefined\tablename
  \renewcommand*\tablename{Table}
\else
  \newcommand\tablename{Table}
\fi
}
\@ifpackageloaded{float}{}{\usepackage{float}}
\floatstyle{ruled}
\@ifundefined{c@chapter}{\newfloat{codelisting}{h}{lop}}{\newfloat{codelisting}{h}{lop}[chapter]}
\floatname{codelisting}{Listing}
\newcommand*\listoflistings{\listof{codelisting}{List of Listings}}
\makeatother
\makeatletter
\makeatother
\makeatletter
\@ifpackageloaded{caption}{}{\usepackage{caption}}
\@ifpackageloaded{subcaption}{}{\usepackage{subcaption}}
\makeatother
\usepackage{bookmark}
\IfFileExists{xurl.sty}{\usepackage{xurl}}{} % add URL line breaks if available
\urlstyle{same}
\hypersetup{
  pdftitle={The Education of a Citizenry: Hope, Demobilization, and Authoritarian Normalization in Cambodia, 2008--2021},
  pdfauthor={Jeffrey Stark},
  pdfkeywords={Cambodia, Asian Barometer Survey, authoritarian
normalization, demobilization by subtraction, competitive
authoritarianism, democratic aspirations},
  colorlinks=true,
  linkcolor={blue},
  filecolor={Maroon},
  citecolor={blue},
  urlcolor={blue},
  pdfcreator={LaTeX via pandoc}}


\title{The Education of a Citizenry: Hope, Demobilization, and
Authoritarian Normalization in Cambodia, 2008--2021}
\author{Jeffrey Stark}
\date{February 22, 2026 at 2:42 PM}
\begin{document}
\maketitle
\begin{abstract}
Between 2008 and 2021, Cambodian citizens underwent one of the most
dramatic transformations in political orientation recorded in
contemporary survey research. Drawing on four waves of Asian Barometer
Survey data (Waves 2, 3, 4, and 6; N ≈ 1,000--1,242 per wave), this
article traces the arc of that transformation as a sequential process:
from post-conflict acquiescence, through a remarkable peak of democratic
aspiration and civic mobilization, to a final condition of political
withdrawal and regime normalization following the elimination of
organized opposition. The Cambodian case offers an unusually clean
empirical window into how dominant-party authoritarian regimes produce
political quiescence---not through the cultivation of enthusiastic
support, but through a process the article terms \emph{demobilization by
subtraction}, in which the removal of credible democratic alternatives
restructures citizen orientations across multiple attitudinal and
behavioral dimensions simultaneously. The findings challenge accounts
that treat authoritarian legitimacy as a product of performance or
ideology, suggesting instead that the most effective mechanism of regime
stabilization may be the systematic closure of the political
imagination.
\end{abstract}


\setstretch{2}
\section{Introduction}\label{introduction}

In 2012, Cambodians rated the democratic future of their country at 9.6
out of 10.

It is worth pausing on that number. On a scale designed to capture the
full range of democratic expectation, Cambodian respondents placed
themselves at the extreme ceiling---more optimistic about their
democratic trajectory than citizens of virtually any other country
surveyed by the Asian Barometer. This was not a country with free and
fair elections, an independent judiciary, or a vibrant free press. It
was a dominant-party authoritarian state under the continuous rule of
Hun Sen and the Cambodian People's Party (CPP), a regime that had held
power since 1985 and showed no signs of voluntary departure. And yet
Cambodians believed, overwhelmingly and almost unanimously, that
democracy was coming.

Nine years later, that same measure had fallen to 6.7 out of 10. In the
same period, every indicator of political participation in the Asian
Barometer Survey---contacting elected officials, signing petitions,
attending demonstrations, discussing politics with others---had
collapsed to historic lows. Acceptance of authoritarian governance
alternatives, including strongman rule, single-party rule, expert rule,
and military rule, had risen sharply. Reported encounters with
corruption had plummeted from 63 percent to 15 percent. And voter
turnout in national elections, now conducted without any viable
opposition, had paradoxically climbed to nearly 90 percent.

What happened in between is the subject of this article. The answer,
stated plainly, is that the Cambodian People's Party dissolved the only
credible opposition party, the Cambodia National Rescue Party (CNRP), in
November 2017, and in doing so restructured the political field so
thoroughly that Cambodian citizens adjusted their attitudes, behaviors,
and expectations in a remarkably coherent pattern. This article argues
that this adjustment is best understood not as a story of authoritarian
consolidation in the conventional sense---not as the triumph of
ideology, coercion, or performance legitimacy---but as something quieter
and, in its way, more profound: the education of a citizenry in the
limits of the politically possible.

The article proceeds as a structured analytic narrative, organized
around the four survey waves as acts in a political drama. This is a
deliberate methodological choice. Cambodia's trajectory between 2008 and
2021 has the unusual property of being both analytically tractable---the
survey data capture discrete moments in a well-documented political
sequence---and narratively coherent, with each wave mapping onto a
distinct phase of political life. The narrative structure allows the
article to preserve the temporal dynamics that conventional
cross-sectional analysis tends to flatten, while the survey data anchor
the story in systematic evidence rather than impressionistic accounts.

The question of how dominant-party regimes endure has generated a rich
architectural vocabulary in contemporary scholarship. We are accustomed
to analyzing the structural mechanics of survival: the skewed playing
fields of competitive authoritarianism, the meticulous menu of
manipulation required to manage electoral autocracies, and the
stabilizing pillars of legitimation, repression, and
co-optation.\footnote{For the foundational typologies of these
  institutional survival mechanics, see Levitsky \& A. (2010); Schedler
  (2013); and Gerschewski (2013). While these frameworks masterfully
  explain elite-level regime consolidation, they are less equipped to
  explain the simultaneous, multi-dimensional shifts in mass public
  opinion observed in the Asian Barometer Survey data during periods of
  opposition collapse.} Yet, this institutional focus, while vital for
understanding the resilience of the state, often obscures the quiet,
downstream effects on the citizenry. Theories of regime survival tend to
treat the public as a variable to be managed---either repressed into
submission or bought off through patronage. This article shifts the
analytical lens. By tracing the Cambodian electorate's trajectory from
2008 to 2021, it focuses not on how the regime maintained power, but on
what happens to citizen orientations when the democratic alternative is
systematically removed. The resulting quiescence, I argue, is not a
product of active authoritarian legitimation, but of demobilization by
subtraction.

The concept advanced here---demobilization by subtraction---is distinct
from the two mechanisms most commonly invoked to explain political
quiescence under authoritarianism. The first, repression-driven
demobilization, holds that citizens withdraw from political life because
the costs of participation have been raised through coercion,
surveillance, or the threat of punishment.\footnote{On repression as a
  mechanism of demobilization, the foundational treatment is Davenport
  (2007). For a comprehensive typology of repressive strategies, see
  also the essays collected in Davenport et al. (2005). On the specific
  mechanisms through which states demobilize protest, see Earl (2011).}
The second, co-optation, holds that citizens acquiesce because they have
been incorporated into the regime's distributional networks and
calculate that loyalty pays better than resistance.\footnote{The
  co-optation framework is developed most fully in Gandhi (2008). On the
  equilibrium logic of authoritarian power-sharing, see Przeworski et
  al. (2000).} Both mechanisms presuppose an active regime strategy
directed at the population. Demobilization by subtraction operates
differently. Its proximate cause is not an action taken against citizens
but an action taken against the political field itself: the removal of
the credible alternative around which civic mobilization had organized.
The result is not fear-based withdrawal or interest-based compliance but
something closer to what Gaventa termed ``quiescence''---the
internalization of powerlessness that follows from the sustained absence
of viable channels for political expression.\footnote{Gaventa's central
  insight---that sustained powerlessness produces not resistance but the
  internalization of quiescence---provides the closest existing
  theoretical analogue to the pattern observed in the Cambodian data,
  though his analysis is situated in a very different political context
  (Gaventa, 1980).} Where Lukes' third dimension of power operates by
shaping preferences so that grievances never form, demobilization by
subtraction operates one step later: grievances may persist, but the
perceived vehicle for their expression has been eliminated, producing a
cascade of behavioral and attitudinal adjustment that is visible across
multiple survey dimensions simultaneously.\footnote{The distinction
  matters: Lukes' third dimension operates \emph{before} grievances
  form; demobilization by subtraction operates \emph{after} grievances
  have formed and been politically activated, removing the vehicle for
  their expression rather than the grievances themselves (Lukes, 2005).}

The remainder of this article is organized as four acts corresponding to
the four available survey waves. Each section opens with a brief account
of the political conditions prevailing at the time of the survey, then
turns to the Asian Barometer data to trace how those conditions
registered in citizen attitudes and behaviors. The analysis tracks five
thematic domains across waves: political participation, authoritarian
governance preferences, democratic expectations, corruption experience
and perception, and media engagement and political interest. A
concluding section draws out the theoretical implications of the
Cambodian sequence for the study of authoritarian durability and
democratic aspiration.

\subsubsection{The Quiet Kingdom: Cambodia in
2008}\label{the-quiet-kingdom-cambodia-in-2008}

The first act of this story is a silence, and it is important to
understand what kind of silence it was.

By 2008, Cambodia had been under continuous CPP rule for over two
decades. The Paris Peace Accords of 1991 and the UNTAC-supervised
elections of 1993 had produced a brief, chaotic experiment in multiparty
democracy, but the CPP had never genuinely relinquished power, and after
a violent coup in 1997 and disputed elections in 1998 and 2003, the
party had consolidated its dominance through a combination of patronage,
institutional control, and selective coercion. The 2008 National
Assembly elections, which the CPP won with 58 of 123 seats and
subsequently expanded through coalition politics, were competitive in
form but not seriously contested in outcome.

The CPP's dominance by 2008 rested on institutional foundations that
went far beyond electoral mechanics. The party had constructed a dense
patronage network linking the central state to village-level
administration, with commune chiefs serving as the capillary system
through which resources, information, and political loyalty
flowed.\footnote{On the CPP's patronage architecture, see Caroline
  Hughes and Kheang Un, \emph{Cambodia's Economic Transformation}
  (Copenhagen: NIAS Press, 2011); Un (2012).} The military and security
services were thoroughly integrated into the party apparatus, and the
judiciary operated as an instrument of executive power rather than an
independent check upon it.\footnote{Blake (2019); Morgenbesser (2016).}
Economic growth, driven by the garment export sector, construction, and
an expanding tourism industry, provided a material foundation for the
regime's implicit social contract: stability and modest prosperity in
exchange for political acquiescence.\footnote{{[}NOTE --- cite World
  Bank Cambodia economic assessments 2008--2010, or Hughes \& Un 2011{]}}

The opposition landscape was fragmented and demoralized. The Sam Rainsy
Party (SRP), the most prominent non-CPP formation, commanded a loyal but
limited urban constituency and suffered from periodic leadership crises
as Sam Rainsy himself cycled between exile and return. The Human Rights
Party, led by Kem Sokha, drew support from a distinct base but lacked
the organizational reach to challenge the CPP's rural dominance. Neither
party, operating independently, presented a credible threat to CPP
hegemony---a structural condition that would change dramatically with
their merger four years later.\footnote{Noren-Nilsson (2015) provides
  the most detailed account of the opposition landscape in this period
  and the narratives that would eventually fuel the CNRP's formation.}

The Wave 2 Asian Barometer data from this period paint a portrait of a
population that was neither enthusiastic nor resistant, but simply
quiet. Political participation measures hovered at modest levels:
contacting elected officials averaged 3.44 on a five-point scale (where
5 represents frequent contact), and contacting civil servants registered
at 3.43---neither disengaged nor actively engaged, but somewhere in the
unremarkable middle. Community leader contact stood at 1.53, suggesting
that local political networks, while present, were not sites of
intensive citizen activity.

{[}STUB: Table 1 --- Wave 2 baseline values across all five thematic
domains: participation, authoritarian attitudes, democratic
expectations, corruption, media/political interest.{]}

The authoritarian attitude measures are perhaps the most telling.
Single-party rule was rated at 1.79 on a four-point scale (where 4
represents ``very good''), strongman rule at 1.60, and military rule at
2.19. These are not the numbers of a population that embraces
authoritarianism as an ideal; they are the numbers of a population that
has learned to live within it. The slightly elevated tolerance for
military rule is consistent with the CPP's own origins in
Vietnamese-backed military intervention and the residual security
anxieties of a post-genocide society, but even this measure sits below
the scale midpoint.

What is most notable about the 2008 data, viewed in retrospect, is how
stable everything appears. There are no dramatic spikes, no anomalous
values, no signs of the upheaval to come. Cambodia in 2008 was a country
in political equilibrium---not a democratic equilibrium, but not one
characterized by active authoritarian mobilization either. It was, to
extend the metaphor, a kingdom in which the subjects had learned that
the king was permanent and had arranged their lives accordingly. The
question that the next four years would pose, with startling force, was
what would happen when that arrangement was briefly, thrillingly
disrupted.

\subsubsection{The Awakening: Cambodia in
2012}\label{the-awakening-cambodia-in-2012}

Something extraordinary happened in Cambodia between 2008 and 2012, and
the Wave 3 Asian Barometer data capture its imprint with unusual
clarity.

Across nearly every measure of civic engagement and political
aspiration, Cambodian respondents in 2012 registered at or near the
highest values in the dataset's history. Contacting elected officials
jumped from 3.44 to 4.31. Contacting civil servants rose from 3.43 to
4.33. Petition activity climbed to 4.34 (from no W2 baseline but, as
subsequent waves would reveal, far above the structural average).
Community leader contact increased from 1.53 to 1.90. And the political
participation measures that would only be tracked beginning in Wave
3---attending demonstrations (4.48) and signing petitions
(4.34)---debuted at remarkably high levels.

These are not the numbers of a quiescent population. They are the
numbers of a population that had been politically activated.

The catalyst was the formation of the Cambodia National Rescue Party in
2012, a merger of the Sam Rainsy Party and the Human Rights Party under
the joint leadership of Sam Rainsy and Kem Sokha. The CNRP did something
that no previous Cambodian opposition had managed: it unified the
fragmented non-CPP political space into a single, credible alternative
with a coherent message of democratic reform, anti-corruption, and
economic justice. The party drew its energy from an emerging urban youth
demographic---the first generation of Cambodians born after the Khmer
Rouge, educated in the expanding university system, connected to the
wider world through social media, and impatient with the political
settlement their parents had accepted.

The CNRP's formation in July 2012 was itself a product of political
learning. Previous opposition efforts had foundered on precisely the
fragmentation that the CPP exploited: multiple parties splitting the
non-CPP vote, personal rivalries preventing coordination, and the
absence of a unified organizational infrastructure capable of matching
the CPP's commune-level reach. The merger of the SRP and the Human
Rights Party, brokered through extended negotiations between Sam Rainsy
and Kem Sokha, resolved the coordination problem at a stroke.\footnote{Noren-Nilsson
  (2015); Kheang Un, ``Cambodia in 2012: Towards Developmental
  Authoritarianism?'' \emph{Asian Survey} 53, no. 1 (2013): 142--49.}
The resulting party offered voters something genuinely new in Cambodian
politics: a single, unified alternative to the CPP with recognizable
leaders, a coherent platform, and---critically---the organizational
capacity to mobilize supporters in both urban and peri-urban
constituencies.

The mobilization dynamics visible in the 2012 survey data intensified
dramatically in the months that followed. The July 2013 national
election became the most competitive in Cambodia's post-UNTAC history,
driven by a potent combination of youth demographics, social media
penetration, and economic grievance. Facebook, which had reached an
estimated 1.5 million Cambodian users by 2013, served as both an
organizational tool and an alternative information ecosystem that
partially circumvented CPP control of broadcast media.\footnote{Norén-Nilsson
  (2021a) on Fresh News and Cambodia's digital media landscape. On the
  broader role of social media in authoritarian contexts, see
  Morgenbesser (2020).} The CNRP's campaign messaging---centered on
anti-corruption, land rights, wages for garment workers, and a promise
to bring genuine democratic accountability---resonated with a generation
of voters who had no memory of the Khmer Rouge and little patience for
the stability-first bargain their parents had accepted.\footnote{Norén-Nilsson
  (2021b) on youth mobilization and power reproduction in Cambodia's
  authoritarian system.}

The election results shocked the political establishment. The CNRP won
55 of 123 National Assembly seats, reducing the CPP's margin to its
narrowest since 1993. Allegations of systematic electoral fraud
triggered months of mass protest in Phnom Penh, with demonstrations
regularly drawing crowds estimated in the tens of thousands. For
observers and participants alike, the political settlement that had
governed Cambodia since the late 1990s appeared to be
unraveling.\footnote{Strangio (2014); Morgenbesser (2017).}

The democratic expectation measures tell the same story from a different
angle. When asked to rate the democratic future of their country on a
ten-point scale, Cambodian respondents in 2012 produced a mean of 9.58.
This is, by any standard, an extraordinary number. It is not merely
optimistic; it approaches unanimity. It suggests a population that had
collectively decided that democratic change was not merely desirable but
imminent---that the political order they had lived under for decades was
on the verge of fundamental transformation.

The paradox, of course, is that this expectation was not based on any
structural change in the regime. The CPP still controlled the state
apparatus, the military, the courts, the media, and the vast patronage
networks that sustained its rural base. Hun Sen showed no inclination to
reform or retire. What had changed was not the regime but the perception
of the regime's permanence---and that perceptual shift, as the
subsequent waves would demonstrate, proved far more fragile than the
political moment suggested.

The specific configuration of the democratic assessment triad in 2012 is
theoretically revealing. The past-present-future structure---low past
(3.97), middling present (5.85), stratospheric future (9.58)---is not
the signature of a population evaluating its democratic institutions on
the basis of lived experience. A population grounded in experiential
assessment would be expected to rate the future as a modest
extrapolation from the present, not as a near-perfect score that exceeds
the present by nearly four points. What the 2012 configuration captures
instead is a hope-driven democratic orientation: the conviction that the
current political order is transitional and that the democratic
destination is both certain and imminent. This orientation, as
subsequent waves would reveal, proved extraordinarily sensitive to
political events---precisely because it was grounded in expectation
rather than experience.

Even the corruption data reflect the mobilization moment. Witnessed
corruption stood at 49.4 percent in Wave 3, up from 27.7 percent in Wave
2. This increase is almost certainly not an increase in actual
corruption---the CPP's patronage system was well-established long before
2012---but rather reflects a population that had become more willing to
identify and report corruption as a political problem, emboldened by the
opposition's anti-corruption messaging and the expanded civic space that
accompanied the CNRP's rise.

There are also subtler patterns in the Wave 3 data that resist easy
interpretation. Traditional values measures---obedience to parents,
deference to teacher authority---also peaked in this wave, an unexpected
finding given that the broader mobilization was associated with youth
activism and challenges to established authority. One possibility is
that the CNRP's own messaging, which blended democratic aspiration with
Khmer cultural conservatism and Buddhist moral frameworks, activated
traditional values alongside rather than in opposition to democratic
ones. Another is that the elevated values reflect a general
intensification of political and social engagement in which all
expressed attitudes, whether progressive or traditional, registered at
higher levels. A third possibility, which cannot be ruled out, is survey
context effects: the charged political environment of 2012 may have
produced a general social desirability bias toward expressing strong
opinions of any kind.

The media and political engagement measures complete the portrait of a
mobilized citizenry. Political interest stood at 2.57 on a four-point
scale, and following political news registered at 3.07---both the
highest values in the Cambodian ABS series. Political discussion, at
1.45, remained modest, suggesting that the activation captured in 2012
was more a matter of individual attention and institutional engagement
than of deliberative civic culture. The internet news measure presents a
methodological puzzle: at 5.78, it sits far above any subsequent wave
(W4 and W6 both register 2.17), a gap too large to reflect real-world
change alone. The most likely explanation is a shift in question wording
or response scale between Waves 3 and 4, and this variable should be
interpreted with caution in cross-wave comparisons.\footnote{The ABS
  questionnaire underwent revisions between Waves 3 and 4 that affected
  several media consumption items. The original questionnaires should be
  consulted to verify whether the internet news variable reflects a
  genuine scale change. {[}NOTE: Jeff, you flagged this in the
  prospecting memo --- worth a footnote to the ABS codebook here.{]}}

The 2012 data, taken together, present a portrait of a political
community at its moment of maximum activation. Every dimension the Asian
Barometer measures---participation, aspiration, attention, even the
willingness to name corruption---points in the same direction. It is
tempting, with the knowledge of what came after, to read this as the
high-water mark of Cambodian democracy, the moment before the tide
turned. But that reading, while emotionally satisfying, may be too
simple. What the Wave 3 data actually capture is something rarer and
more theoretically interesting: the exact moment a population
collectively updated its beliefs about what was politically possible,
right before the regime demonstrated, with devastating effectiveness,
that those beliefs were wrong.

\subsubsection{The Reckoning: Cambodia in
2015}\label{the-reckoning-cambodia-in-2015}

That demonstration began at the ballot box and ended in the streets. The
2013 national election, held just months after the Wave 3 survey
captured such stratospheric optimism, confirmed the CNRP's mobilization
power: the opposition won 55 seats to the CPP's 68, the narrowest margin
in the country's modern history. Yet, the structural ceiling of the
dominant-party state held firm. Amid widespread allegations of electoral
fraud, the CNRP boycotted parliament, triggering months of mass
demonstrations that soon merged with massive garment worker strikes. For
a brief window, genuine political transformation appeared imminent. It
did not arrive. Following a lethal state crackdown on protesters in
January 2014, and a subsequent political compromise that coaxed the
opposition into the National Assembly under a fragile and temporary
``Culture of Dialogue,'' the momentum of the streets fractured.

The ``Culture of Dialogue'' that emerged from the July 2014 agreement
between Sam Rainsy and Hun Sen was, in retrospect, less a genuine
political opening than a controlled decompression. The CNRP entered
parliament and gained access to committee chairmanships, lending the
arrangement a veneer of power-sharing. But the underlying dynamics had
not changed: the CPP retained control of the security forces, the
judiciary, and the National Election Committee, while the CNRP's
capacity for mass mobilization---its primary source of leverage---was
progressively constrained through targeted legal harassment of its
leaders and selective restrictions on public assembly.\footnote{Morgenbesser
  (2017); Norén-Nilsson (2019). On the broader pattern of managed
  political openings in dominant-party systems, see Schedler (2013).} By
2015, the dialogue had largely stalled, and the political atmosphere had
shifted from the hopeful if chaotic energy of 2013 to something more
wary and uncertain. It was in this atmosphere that the Wave 4 Asian
Barometer survey entered the field.

The Wave 4 data from 2015 capture a population in the midst of this
reckoning---no longer at the peak of mobilization but not yet in the
trough of withdrawal. The numbers tell a story of selective retreat.
Contacting elected officials fell from 4.31 to 3.08. Contacting civil
servants dropped from 4.33 to 3.19. Demonstration attendance declined
from 4.48 to 3.07, and petition activity from 4.34 to 3.23. Community
leader contact, however, held relatively steady at 2.00, suggesting that
local-level political networks remained more resilient than
national-level civic action.

{[}STUB: Table 2 --- Wave 3 to Wave 4 comparison across all domains,
with change values.{]}

The retreat from participation was not yet accompanied by a
corresponding rise in authoritarian acceptance. Single-party rule
actually dipped slightly from 1.97 to 1.78, and military rule fell from
2.19 to 1.98. Expert rule and strongman rule edged up marginally but
remained below 2.0. This pattern is analytically important: it suggests
that the demobilization observed between 2012 and 2015 was driven more
by disillusionment and tactical withdrawal than by a genuine
reorientation toward authoritarian preferences. Cambodians were pulling
back from political action, but they had not yet accepted the
authoritarian order as normatively appropriate.

The democratic expectation measures, however, had already begun their
descent. The democratic future rating fell from 9.58 to 7.72---a drop of
nearly two full points on the ten-point scale. The democratic present,
as assessed through government performance, declined from 5.85 to 5.06.
Only the democratic past held relatively steady at 3.87. What this
configuration suggests is that the hope that had animated the 2012
moment was already eroding, not because Cambodians had embraced an
alternative vision, but because the events of 2013-2014---the election
that changed nothing, the protests that were suppressed, the deal that
left the CPP in power---had begun to teach a lesson about the limits of
popular mobilization within a dominant-party system.

The corruption data from Wave 4 introduce a striking counterpoint.
Witnessed corruption surged to its highest level in the dataset: 62.8
percent of respondents reported having personally encountered
corruption. This was not an artifact of increased willingness to report;
the political environment of 2015 was, if anything, less permissive of
anti-government speech than 2012. The more likely explanation is that
the post-2013 period, with its political bargaining, institutional
jockeying, and expanded CNRP presence in parliament, created more
visible sites of corruption---more interactions between citizens and
competing political actors, more opportunities to observe the
transactional nature of Cambodian governance up close.

The informational dimensions of political life registered the earliest
signs of the withdrawal that would accelerate dramatically by 2021.
Political interest declined from 2.57 to 2.29, and following political
news fell from 3.07 to 2.86---modest drops in absolute terms, but
significant as the leading edge of a trend that would deepen sharply.
Political discussion held essentially flat at 1.47, suggesting that
Cambodians were beginning to disengage from formal political information
channels while maintaining informal conversational patterns. The
internet news measure, as noted, dropped precipitously from 5.78 to
2.17, but the probable methodological discontinuity between waves makes
this variable unreliable as an indicator of real-world change in this
specific transition.

The 2015 wave, viewed within the larger narrative, represents the hinge
point---the moment when the trajectory could still have gone either way.
The CNRP was weakened but present, its leaders alternately accommodated
and harassed, its supporters disillusioned but not yet defeated. The
data register this ambiguity: down from the peaks of 2012 but not yet at
the floor. Participation was retreating but authoritarian acceptance had
not yet risen. Democratic hope was falling but had not crashed.
Corruption was more visible than ever. It was a moment of unstable
equilibrium, poised between the democratic aspiration of 2012 and the
authoritarian normalization that was about to descend.

\subsubsection{The Silence: Cambodia in
2021}\label{the-silence-cambodia-in-2021}

On November 16, 2017, the Supreme Court of Cambodia dissolved the
Cambodia National Rescue Party. The ruling, which cited a vaguely
defined conspiracy to overthrow the government, banned 118 CNRP
officials from political activity for five years and transferred the
party's parliamentary seats to smaller, CPP-aligned parties. Kem Sokha
had already been arrested in September; Sam Rainsy remained in exile. In
the July 2018 national election, the CPP won all 125 seats in the
National Assembly. Cambodia became, for all practical purposes, a
one-party state.

The CNRP's dissolution did not occur in isolation. It was the
centerpiece of a broader campaign of political closure that
systematically dismantled the infrastructure of independent civic life.
The \emph{Cambodia Daily}, the country's most prominent English-language
newspaper and a persistent thorn in the CPP's side, was forced to close
in September 2017 after receiving a disputed tax bill of \$6.3
million.\footnote{The Cambodia Daily's closure was widely interpreted as
  politically motivated. See Norén-Nilsson (2021a) on the restructuring
  of Cambodia's media landscape during this period.} Radio Free Asia's
Cambodian bureau was shuttered, and several of its journalists faced
criminal charges. Independent radio stations that had broadcast
opposition content were pressured to switch to pro-government
programming or cease operations entirely. Civil society organizations
faced intensified regulatory scrutiny under the 2015 Law on Associations
and Non-Governmental Organizations (LANGO), which gave the government
broad discretionary power to dissolve organizations deemed to threaten
public order.\footnote{On LANGO and its effects on civil society space,
  see Norén-Nilsson \& Eng (2020).}

The combined effect was the elimination not merely of the political
opposition but of the broader ecosystem---media, civil society, informal
networks---through which citizens had accessed alternative political
information and organized collective action. The transition, in Levitsky
and Way's typology, was from competitive authoritarianism, in which
opposition parties exist and can occasionally win, to hegemonic
authoritarianism, in which the electoral arena is formally maintained
but substantively emptied.\footnote{Levitsky \& A. (2010); Schedler
  (2002). Morgenbesser \& Pepinsky (2019) applies this framework
  specifically to Cambodia's post-2017 trajectory.} The 2018 national
election, conducted without the CNRP on the ballot, produced a CPP sweep
of all 125 National Assembly seats---a result that formalized what the
dissolution had already accomplished.

The Wave 6 data from 2021 record the consequences.

The participation collapse is the most dramatic finding. Contacting
elected officials fell to 1.74---a drop from 4.31 in 2012 and 3.08 in
2015, representing one of the steepest declines in civic participation
recorded in the Asian Barometer dataset. Demonstration attendance
crashed to 1.29, barely above the theoretical floor. Petition activity
fell to 2.14. Contacting civil servants dropped to 2.47. Even community
leader contact, which had been the most resilient participation measure,
fell to 1.39. Political discussion, already low, declined to 1.26---a
population that had largely stopped talking about politics with one
another.

{[}STUB: Table 3 --- Full four-wave trajectory across all domains. This
is the key empirical table of the paper.{]}

The authoritarian acceptance measures moved in mirror image. All four
alternative governance indicators rose to their highest levels: expert
rule to 2.11, single-party rule to 2.21, strongman rule to 2.17, and
military rule to 2.21. The jump was concentrated entirely in the
W4-to-W6 period---the six years that encompassed the CNRP's destruction
and the subsequent one-party election. What had been stable or declining
through 2015 now surged.

There is an important interpretive question here, and the article's
argument depends on how it is answered. Does the rise in authoritarian
acceptance represent genuine preference change---Cambodians coming to
believe that authoritarian governance is normatively desirable---or does
it represent something else? Three considerations suggest the latter.
First, the magnitude of the shift, while statistically meaningful, is
modest in absolute terms: all four measures remain around 2.2 on a
four-point scale, below the midpoint. Cambodians are not rating
authoritarianism as ``good''; they are rating it as slightly less bad
than before. Second, the timing---concentrated after the removal of the
democratic alternative---suggests a mechanism closer to adaptive
preference formation than genuine conversion. Third, and most
importantly, the shift co-occurs with the collapse of democratic future
expectations (from 7.72 to 6.67), rising perceptions of government
information withholding, and declining political interest. This is not
the attitudinal profile of a population that has found a new political
faith; it is the profile of a population that has stopped believing the
old one was achievable.

The article proposes the term authoritarian normalization to describe
this process, distinguishing it from authoritarian legitimation.
Legitimation implies that citizens have come to view the regime as
rightfully entitled to rule, whether on the basis of performance,
ideology, or tradition. Normalization implies something weaker but
potentially more durable: citizens have come to view the regime as the
only available reality and have adjusted their expressed preferences to
align with that reality. The distinction matters theoretically because
it suggests different mechanisms of regime stability---and different
vulnerabilities. A regime sustained by legitimation is threatened when
performance falters or ideology loses appeal. A regime sustained by
normalization is threatened when alternatives re-enter the political
imagination.

The normalization framework finds a useful analogue in Wedeen's analysis
of political compliance in Hafez al-Assad's Syria, where citizens
participated in elaborate public rituals of regime praise not because
they believed the regime's claims but because acting as if they believed
was the condition of ordinary life.\footnote{Wedeen (1998); Wedeen
  (2000). Wedeen's concept of ``acting as if'' captures the performative
  dimension of political compliance under conditions where genuine
  belief is neither required nor expected by the regime.} The Cambodian
case differs in important respects---the ABS data capture private survey
responses rather than public performances, and the CPP's demands on
citizens are less theatrically extravagant than the Assad cult of
personality. But the underlying logic is similar: when the political
field is restructured so that only one arrangement appears viable,
citizens adjust their expressed orientations to accommodate that
arrangement, not out of conviction but out of a practical reckoning with
the available reality. The survey data cannot definitively distinguish
between genuine preference change and adaptive accommodation, but the
pattern of evidence---modest rather than dramatic shifts in
authoritarian acceptance, coinciding with collapsing democratic
expectations and civic withdrawal---is more consistent with
normalization than with conversion.

The corruption data offer perhaps the most unsettling evidence of
normalization. Witnessed corruption collapsed from 62.8 percent in 2015
to 14.9 percent in 2021. Perceived national government corruption also
declined, from 2.90 to 2.33 on the four-point scale. On the surface,
these numbers might suggest genuine anti-corruption progress. The CPP
did conduct selective anti-corruption campaigns during this period,
targeting rivals and politically inconvenient officials. But the
magnitude of the witnessed-corruption decline---from nearly two-thirds
to one-seventh of the population---is too large to explain through
policy alone.

The more parsimonious explanation is that the corruption decline
reflects the same demobilization process visible in the participation
data. When citizens withdraw from political life---when they stop
contacting officials, stop attending meetings, stop engaging with
governance institutions---they also reduce their exposure to the sites
where corruption is experienced and observed. A population that does not
petition, does not demonstrate, and does not contact civil servants is
also a population that has fewer occasions to witness a bribe demanded
or a favor exchanged. Demobilization, in this reading, does not merely
reduce political action; it reduces the perceptual evidence that
political action might address. The regime becomes less corrupt not
because it has reformed but because citizens have stopped looking.

An alternative, and not mutually exclusive, explanation for the
corruption decline invokes the survey environment itself. In a political
climate where the opposition has been criminalized and independent media
shuttered, respondents may reasonably calculate that acknowledging
witnessed corruption---even in a nominally confidential survey---carries
risks of self-incrimination or unwanted official attention. The social
desirability bias literature has long recognized that survey responses
on sensitive topics shift in predictable directions under conditions of
political constraint, with respondents offering answers that align with
perceived official expectations.\footnote{On social desirability bias in
  authoritarian survey contexts, see Schedler and Sarsfield's work on
  survey reliability under electoral authoritarianism (Schedler \&
  Sarsfield, 2007). {[}NOTE: Jeff, you might also cite Frye et
  al.~(2017) ``Is Putin's Popularity Real?'' or similar --- check your
  bib for survey methodology under authoritarianism references.{]}} The
post-2017 Cambodian environment, in which public criticism of the
government carried tangible legal consequences, represents precisely the
conditions under which such bias would be expected to intensify. The two
explanations---reduced exposure through demobilization and increased
reticence through social desirability pressure---likely operate in
tandem, each reinforcing the other to produce the dramatic decline
observed in the data.

The voter turnout paradox completes the portrait. Despite the collapse
of all other forms of political participation, self-reported voting in
the last election rose from 78.7 percent in 2012 to 83.2 percent in 2015
to 88.1 percent in 2021. This pattern---rising electoral participation
amid collapsing civic participation---is a well-documented feature of
hegemonic authoritarian systems, where elections serve not as mechanisms
of political choice but as rituals of regime affirmation. Voting in a
one-party election is not a political act in the same sense as voting in
a competitive one; it is closer to a demonstration of compliance, a
public performance of belonging to the political community as the regime
defines it. The fact that turnout rises precisely as meaningful
participation collapses suggests that Cambodians understand the
distinction even as they perform the ritual.

The international orientation data add a final dimension to the
normalization portrait. Cambodian assessments of Chinese influence in
the region grew more positive between Waves 4 and 6, rising from 2.69 to
2.88 on a four-point scale measuring perceived benefit versus harm. More
striking still, expectations of future Asian---implicitly
Chinese---regional influence climbed from 2.74 in Wave 4 to 3.30 in Wave
6, a trajectory that places Cambodia among the most China-positive
publics in the ABS dataset. This is consistent with Cambodia's deepening
alignment with Beijing during this period, visible in Belt and Road
infrastructure investment, the Ream Naval Base controversy, and
Cambodia's consistent support for Chinese positions within ASEAN. But it
also fits the normalization logic: as democratic futures recede and
Western-aligned political alternatives are eliminated, alignment with
the ascendant authoritarian power in the region becomes the natural
orientation of a population adjusting to the available political
reality. A paired comparison with Vietnam or the Philippines---where
territorial disputes drive China perceptions in sharply negative
directions despite similar economic interdependence---would sharpen this
point considerably.\footnote{Strangio (2020) provides the most
  comprehensive account of Cambodia's positioning within China's
  Southeast Asian strategy. On the divergent China perception
  trajectories across ASEAN, the ABS cross-national data offer direct
  comparative evidence. {[}NOTE: Jeff, this could connect to your
  Vietnam Paradox paper --- worth a forward reference if you're
  comfortable signaling the broader research agenda.{]}}

The informational withdrawal that began tentatively in 2015 reached its
full expression by 2021. Following political news fell to 2.04,
political interest declined to 2.06, and political discussion dropped to
1.26---the lowest values recorded across all four waves. These figures
describe a population that has not merely withdrawn from political
action but has largely ceased to attend to political information at all.
The self-reinforcing logic is straightforward: citizens who do not
participate have less reason to follow political developments; citizens
who do not follow political developments have less basis on which to
participate. Demobilization, once initiated through the removal of the
political alternative, generates its own sustaining momentum through
this informational feedback loop---a dynamic that suggests the effects
of the CNRP's dissolution may prove more durable than the act itself.

The Cambodia of 2021 is, in a sense, the inverse of the Cambodia of
2012. Where the earlier moment was characterized by activation across
all dimensions---participation, aspiration, attention, even the
willingness to name corruption---the later moment is characterized by
withdrawal across all the same dimensions. The symmetry is striking and,
for the theoretical argument, essential. It suggests that what changed
between the two moments was not any single attitude or behavior but the
underlying orientation from which attitudes and behaviors derive---what
might be called the perceived horizon of political possibility. When
that horizon expanded, everything expanded with it. When it contracted,
everything contracted.

\subsubsection{The Education}\label{the-education}

The story told in the preceding pages is, in one sense, particular to
Cambodia---a product of specific historical circumstances, a specific
regime, a specific opposition movement, and a specific act of political
closure. In another sense, it is a story with much wider application,
because the mechanism it illustrates---the restructuring of citizen
orientations through the removal of political alternatives---operates
wherever dominant-party regimes consolidate power. This concluding
section draws out four theoretical implications of the Cambodian
sequence and considers their relevance beyond the single case.

The first concerns the mechanism itself. Demobilization by subtraction,
as the Cambodian data illustrate it, is distinct from the two processes
most commonly invoked to explain political quiescence. It is not
repression-driven demobilization: while the CPP certainly employed
coercion, the attitudinal shifts observed in the ABS data---rising
authoritarian acceptance, collapsing democratic expectations, declining
political interest---are not the signature of a frightened population. A
population demobilized primarily through fear would be expected to
maintain private democratic commitments even as public participation
declined; the Cambodian data show both dimensions moving in the same
direction.\footnote{This pattern is consistent with the distinction
  between ``preference falsification,'' in which private attitudes
  diverge from public behavior, and genuine preference adjustment. The
  Cambodian case appears closer to the latter, though survey data alone
  cannot fully resolve the question. On preference falsification under
  authoritarianism, see Kuran (1995).} Nor is it co-optation: the CPP's
patronage networks were already well-established before 2012, and there
is no evidence of a dramatic expansion of distributive benefits between
2015 and 2021 that could account for the attitudinal shifts. What
changed was not the cost of participation or the rewards of compliance
but the perceived availability of an alternative---the object around
which civic mobilization had organized. When that object was removed,
the entire architecture of engagement collapsed with it.

The second implication concerns the relationship between hope and
democratic attitudes. The Wave 3 data suggest that democratic support in
authoritarian contexts may be driven less by abstract normative
commitments than by perceived feasibility. When democracy appeared
possible in 2012, every democratic indicator in the ABS surged---not
only future expectations but participation, political interest, and even
the willingness to name corruption. When democracy appeared impossible
by 2021, those same indicators collapsed. This pattern challenges a
foundational assumption in the democratic values literature: that
survey-measured democratic attitudes reflect stable, deeply held
orientations that exist independently of political
circumstances.\footnote{The canonical formulation of democratic values
  as a stable cultural attribute is found in the political culture
  tradition from Inglehart (1997) onward. The Cambodian data suggest a
  more situationally contingent model, closer to what Bratton and Mattes
  have termed ``demand for democracy'' as a function of perceived supply
  (Bratton et al., 2004; Mattes \& Bratton, 2007).} The Cambodian
evidence suggests instead that measured democratic attitudes may be
substantially a function of perceived democratic possibility---that hope
is not merely the emotional accompaniment of democratic aspiration but
its primary cognitive driver. If this is correct, cross-national surveys
that measure democratic support without accounting for the structure of
political alternatives available to respondents may be capturing
something quite different from what they intend.

The third implication concerns corruption measurement. The finding that
witnessed corruption declined precipitously alongside civic
participation suggests a troubling methodological artifact: survey-based
corruption indicators may systematically understate corruption in
demobilized populations. When citizens withdraw from engagement with
state institutions, they also withdraw from the sites where corruption
is experienced and observed. A regime that successfully demobilizes its
population will, by this logic, appear less corrupt on standard survey
measures even if actual corruption is unchanged or increasing. This has
practical consequences for the evaluation of anti-corruption campaigns
in authoritarian contexts. The CPP's selective anti-corruption efforts
during this period may well have had some genuine effect, but the
dramatic scale of the decline---from 63 percent to 15 percent---is far
more parsimoniously explained by the withdrawal of citizens from the
institutional encounters where corruption becomes visible. Transparency
International's Corruption Perception Index and similar instruments may
be particularly vulnerable to this dynamic in consolidating
authoritarian regimes.

The fourth implication is diagnostic. The divergence between electoral
participation and civic participation---turnout rising to 88 percent
even as every other form of engagement collapsed---may serve as a
generalizable indicator of authoritarian normalization. Where elections
are maintained as regime rituals rather than competitive mechanisms,
rising turnout signals not democratic health but its opposite: the
transformation of voting from an act of political choice into an act of
political compliance.\footnote{On the ritual function of elections in
  hegemonic authoritarian systems, see Schedler (2002); Magaloni \&
  Kricheli (2010).} The Cambodian case suggests that this
turnout-participation divergence emerges specifically in the wake of
opposition elimination, and that its appearance in other
contexts---wherever rising turnout coincides with collapsing civic
engagement---should be read as evidence that the electoral arena has
been substantively emptied even if it remains formally intact.

These findings carry inherent limitations. Four survey waves across
thirteen years provide a suggestive but not definitive empirical
foundation; the absence of Waves 1 and 5 from the Cambodian ABS series
limits the precision with which structural breaks can be identified. The
Wave 3 peak, while consistent with the CNRP mobilization narrative, may
partly reflect survey timing effects rather than durable attitudinal
change. The Wave 6 data, collected during the COVID-19 pandemic, are
subject to the concern that mobility restrictions and heightened state
authority may have independently depressed participation measures and
elevated compliance orientations. The analysis has addressed this
concern by noting that the key attitudinal shifts---particularly the
decline in democratic future expectations---began in Wave 4, well before
the pandemic. But the possibility of pandemic amplification effects in
the 2021 data cannot be entirely excluded. Finally, any single-country
study faces the inherent limitation of generalizability: the mechanisms
identified here may be specific to Cambodia's particular configuration
of post-conflict politics, dominant-party rule, and opposition dynamics.

The most productive response to this limitation is not defensive
qualification but comparative extension. The demobilization sequence
observed in Cambodia---civic participation collapse, followed by rising
authoritarian acceptance, followed by declining democratic
aspiration---generates a testable temporal ordering that can be examined
in other cases of opposition elimination. Turkey after the suppression
of the Peoples' Democratic Party (HDP), Russia following the banning of
Navalny's organizational network, and Thailand in the years following
the 2014 coup all represent contexts where credible political
alternatives were systematically removed from the political field. The
Asian Barometer dataset itself offers the most direct comparative
leverage: Thailand's post-coup trajectory and the Philippines under
Duterte provide cases where similar dynamics may be observable using
identical survey instruments. Future research could test whether the
Cambodian sequence represents a general pattern of authoritarian
normalization or a configuration specific to the conditions of this
particular case.

This article began with a number: 9.6 out of 10. It is worth ending with
the question of what that number meant. In 2012, Cambodian citizens
looked at their country---a dominant-party authoritarian state with
controlled media, a compromised judiciary, and a history of political
violence---and saw a democratic future so bright they could barely
imagine anything brighter. Nine years later, they had learned otherwise.
The democratic future had not arrived. The party that promised it had
been dissolved. The leaders who championed it were in prison or exile.
And the population, having briefly imagined a different political life,
had returned to the quiet arrangement their parents had
known---participating in elections that offered no choice, accepting
governance structures they had once rejected, and gradually, measurably,
ceasing to pay attention.

Any reliance on longitudinal survey data carries inherent
vulnerabilities, and this analysis of Cambodia's trajectory is no
exception. The dramatic peaks of 2012 and the stark valleys of 2021 are
undeniably colored by the unique atmospheric conditions of those exact
moments.\footnote{Two contextual anomalies warrant specific
  acknowledgment. First, the Wave 3 (2012) survey was fielded during an
  unprecedented window of opposition mobilization just prior to the 2013
  elections, meaning the data likely capture a temporary zenith of
  democratic optimism rather than a stable baseline. Second, the Wave 6
  (2021) data were collected during the COVID-19 pandemic, a period
  where public health restrictions inherently depressed physical civic
  participation (e.g., attending demonstrations) and may have
  temporarily inflated reliance on state authority. However, the fact
  that ideological demobilization (such as the drop in democratic future
  expectations) began in Wave 4, well before the pandemic, suggests the
  2021 results reflect deeper authoritarian normalization rather than
  mere epidemiological artifacts. On the methodological challenges of
  surveying under authoritarian conditions more generally, see Schedler
  \& Sarsfield (2007). Furthermore, the single-country focus of this
  study naturally bounds its immediate generalizability, though the
  mechanisms observed here offer a framework for testing similar
  dynamics in other consolidating hegemonic regimes.} Yet, the
theoretical weight of this study does not rest on the absolute value of
any single data point, but on the architecture of the sequence. By
anchoring this narrative in within-case temporal variation---observing
the same population across four distinct acts of a remarkably
well-documented political closure---the noise of individual survey waves
gives way to a clear, structural signal. It provides an unusually clean
empirical window into how a population recalibrates its political
reality when the horizon of possibility collapses.

The fairy tale has no happy ending. But it has a lesson, and the lesson
is not about Cambodia alone. It is about the fragility of democratic
aspiration in the absence of democratic infrastructure, and about the
remarkable efficiency with which authoritarian regimes can reshape
citizen orientations simply by removing the object around which those
orientations had formed. The education of Cambodia's citizenry between
2008 and 2021 was not accomplished through indoctrination or ideology.
It was accomplished through subtraction---the removal of an alternative,
and the long silence that followed.

\newpage

\section*{References}\label{references}
\addcontentsline{toc}{section}{References}

\phantomsection\label{refs}
\begin{CSLReferences}{1}{0}
\bibitem[\citeproctext]{ref-Blake2019-ji}
Blake, David J H. (2019). {Recalling hydraulic despotism: Hun Sen's
Cambodia and the return of strict authoritarianism}. \emph{Austrian
Journal of South-East Asian Studies}, \emph{12}, 69--89.
\url{https://doi.org/10.14764/10.ASEAS-0014}

\bibitem[\citeproctext]{ref-Bratton2004-ke}
Bratton, Michael, Mattes, Robert, \& Gyimah-Boadi, E. (2004).
\emph{{Cambridge studies in comparative politics: Public opinion,
democracy, and market reform in Africa}}. Cambridge University Press.

\bibitem[\citeproctext]{ref-Davenport2007-ww}
Davenport, Christian. (2007). {State Repression and Political Order}.
\emph{Annual Review of Political Science}, \emph{10}, 1--23.
\url{https://doi.org/10.1146/annurev.polisci.10.101405.143216}

\bibitem[\citeproctext]{ref-Davenport2005-mk}
Davenport, Christian, Johnston, Hank, \& Mueller, Carol Mcclurg. (2005).
\emph{{Repression and Mobilization}}. University of Minnesota Press.

\bibitem[\citeproctext]{ref-Earl2011-et}
Earl, Jennifer. (2011). {Political Repression: Iron Fists, Velvet
Gloves, and Diffuse Control}. \emph{Annual Review of Sociology},
\emph{37}, 261--284.
\url{https://doi.org/10.1146/annurev.soc.012809.102609}

\bibitem[\citeproctext]{ref-Gandhi2008-dk}
Gandhi, Jennifer. (2008). \emph{{Political institutions under
dictatorship}}. Cambridge University Press.
\url{https://doi.org/10.1017/cbo9780511510090}

\bibitem[\citeproctext]{ref-Gaventa1980-ob}
Gaventa, John. (1980). \emph{{Power and powerlessness: Quiescence and
rebellion in an Appalachian valley}}. Oxford University Press.

\bibitem[\citeproctext]{ref-Gerschewski2013-en}
Gerschewski, Johannes. (2013). {The three pillars of stability:
legitimation, repression, and co-optation in autocratic regimes}.
\emph{Democratization}, \emph{20}, 13--38.
\url{https://doi.org/10.1080/13510347.2013.738860}

\bibitem[\citeproctext]{ref-Inglehart1997-gw}
Inglehart, Ronald. (1997). \emph{{Modernization and postmodernization:
Cultural, economic, and political change in 43 societies}}. Princeton
University Press.

\bibitem[\citeproctext]{ref-Kuran1995-up}
Kuran, Timur. (1995). \emph{{Private truths, public lies: The social
consequences of preference falsification}} (2nd ed.). Harvard University
Press.

\bibitem[\citeproctext]{ref-Levitsky2010-bc}
Levitsky, Steven, \& A., Lucan. (2010). \emph{{Problems of international
politics: Competitive authoritarianism: Hybrid regimes after the cold
war}}. Cambridge University Press.

\bibitem[\citeproctext]{ref-Lukes2005-bc}
Lukes, Steven. (2005). \emph{{Power: A Radical View}} (2nd ed.).
Palgrave MacMillan.

\bibitem[\citeproctext]{ref-Magaloni2010-wl}
Magaloni, Beatriz, \& Kricheli, Ruth. (2010). {Political order and
one-party rule}. \emph{Annual Review of Political Science (Palo Alto,
Calif.)}, \emph{13}, 123--143.
\url{https://doi.org/10.1146/annurev.polisci.031908.220529}

\bibitem[\citeproctext]{ref-Mattes2007-ej}
Mattes, Robert, \& Bratton, Michael. (2007). {Learning about {Democracy}
in {Africa}: Awareness, {Performance}, and {Experience}}. \emph{American
Journal of Political Science}, \emph{51}, 192--217.

\bibitem[\citeproctext]{ref-Morgenbesser2016-vf}
Morgenbesser, Lee. (2016). \emph{{Behind the Facade: Elections under
Authoritarianism in Southeast Asia}}. State University of New York
Press.

\bibitem[\citeproctext]{ref-Morgenbesser2017-so}
Morgenbesser, Lee. (2017). {The autocratic mandate: elections,
legitimacy and regime stability in {Singapore}}. \emph{The Pacific
Review}, \emph{30}, 205--231.
\url{https://doi.org/10.1080/09512748.2016.1201134}

\bibitem[\citeproctext]{ref-Morgenbesser2020-kz}
Morgenbesser, Lee. (2020). \emph{{The rise of sophisticated
authoritarianism in Southeast Asia}}. Cambridge University Press.

\bibitem[\citeproctext]{ref-Morgenbesser2019-mb}
Morgenbesser, L, \& Pepinsky, T B. (2019). {Elections as causes of
democratization: Southeast Asia in comparative perspective}.
\emph{Comparative Political Studies}.

\bibitem[\citeproctext]{ref-Noren-Nilsson2015-yi}
Noren-Nilsson, Astrid. (2015). {Cambodia at a Crossroads. The Narratives
of Cambodia National Rescue Party Supporters after the 2013 Elections}.
\emph{Internationales Asienforum}, \emph{46}, 261--278.
\url{https://doi.org/10.11588/iaf.2015.46.3726}

\bibitem[\citeproctext]{ref-Noren-Nilsson2019-ck}
Norén-Nilsson, Astrid. (2019). {Kem Ley and Cambodian Citizenship Today:
Grass-Roots Mobilisation, Electoral Politics and Individuals}.
\emph{Journal of Current Southeast Asian Affairs}, \emph{38}, 77--97.
\url{https://doi.org/10.1177/1868103419846009}

\bibitem[\citeproctext]{ref-Noren-Nilsson2021-is}
Norén-Nilsson, Astrid. (2021a). {Fresh News, innovative news:
popularizing Cambodia's authoritarian turn}. \emph{Critical Asian
Studies}, \emph{53}, 89--108.
\url{https://doi.org/10.1080/14672715.2020.1837637}

\bibitem[\citeproctext]{ref-Noren-Nilsson2021-sq}
Norén-Nilsson, Astrid. (2021b). {Youth Mobilization, Power Reproduction
and Cambodia's Authoritarian Turn}. \emph{Contemporary Southeast Asia: A
Journal of International and Strategic Affairs}, \emph{43}, 265--292.

\bibitem[\citeproctext]{ref-Noren-Nilsson2020-ev}
Norén-Nilsson, Astrid, \& Eng, Netra. (2020). {Pathways to leadership
within and beyond Cambodian civil society: Elite status and
boundary-crossing}. \emph{Politics and Governance}, \emph{8}, 109--119.
\url{https://doi.org/10.17645/pag.v8i3.3020}

\bibitem[\citeproctext]{ref-Przeworski2000-ed}
Przeworski, Adam, Alvarez, Michael E, Cheibub, Jose Antonio, \& Limongi,
Fernando. (2000). \emph{{Democracy and Development: Political
Institutions and Well-Being in the World, 1950-{1990}}}. Cambridge
University Press.

\bibitem[\citeproctext]{ref-Schedler2002-eg}
Schedler, Andreas. (2002). {Elections Without Democracy: The Menu of
Manipulation}. \emph{Journal of Democracy}, \emph{13}, 36--50.
\url{https://doi.org/10.1353/jod.2002.0031}

\bibitem[\citeproctext]{ref-Schedler2013-qn}
Schedler, Andreas. (2013). \emph{{The politics of uncertainty:
Sustaining and subverting electoral authoritarianism}}. Oxford
University Press.

\bibitem[\citeproctext]{ref-Schedler2007-pv}
Schedler, A, \& Sarsfield, R. (2007). {Democrats with Adjectives:
Linking Direct and Indirect Measures of Democratic Support}.
\emph{European Journal of Political Research}, \emph{46}, 637--659.

\bibitem[\citeproctext]{ref-Strangio2014-vi}
Strangio, Sebastian. (2014). \emph{{Hun Sen's Cambodia}}. Yale
University Press.

\bibitem[\citeproctext]{ref-Strangio2020-gw}
Strangio, Sebastian. (2020). \emph{{In the Dragon's Shadow: Southeast
Asia in the Chinese Century}}. Yale University Press.

\bibitem[\citeproctext]{ref-Un2012-im}
Un, Kheang. (2012). {A Thin Veneer of Change}. \emph{Asian Survey},
\emph{52}, 202--209. \url{https://doi.org/10.1525/as.2012.52.1.202}

\bibitem[\citeproctext]{ref-Wedeen1998-qz}
Wedeen, Lisa. (1998). {Acting {``As If''}: Symbolic Politics and Social
Control in Syria}. \emph{Comparative Studies in Society and History},
\emph{40}, 503--523. \url{https://doi.org/10.1017/S0010417598001388}

\bibitem[\citeproctext]{ref-Wedeen2000-bg}
Wedeen, Lisa. (2000). \emph{{Ambiguities of domination: Politics,
rhetoric, and symbols in contemporary Syria}} (2nd ed.). University of
Chicago Press.

\end{CSLReferences}




\end{document}
