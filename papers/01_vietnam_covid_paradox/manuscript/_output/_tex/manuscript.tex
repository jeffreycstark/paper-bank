% Options for packages loaded elsewhere
% Options for packages loaded elsewhere
\PassOptionsToPackage{unicode}{hyperref}
\PassOptionsToPackage{hyphens}{url}
\PassOptionsToPackage{dvipsnames,svgnames,x11names}{xcolor}
%
\documentclass[
  12pt,
]{article}
\usepackage{xcolor}
\usepackage[margin=1in]{geometry}
\usepackage{amsmath,amssymb}
\setcounter{secnumdepth}{5}
\usepackage{iftex}
\ifPDFTeX
  \usepackage[T1]{fontenc}
  \usepackage[utf8]{inputenc}
  \usepackage{textcomp} % provide euro and other symbols
\else % if luatex or xetex
  \usepackage{unicode-math} % this also loads fontspec
  \defaultfontfeatures{Scale=MatchLowercase}
  \defaultfontfeatures[\rmfamily]{Ligatures=TeX,Scale=1}
\fi
\usepackage{lmodern}
\ifPDFTeX\else
  % xetex/luatex font selection
  \setmainfont[]{Times New Roman}
\fi
% Use upquote if available, for straight quotes in verbatim environments
\IfFileExists{upquote.sty}{\usepackage{upquote}}{}
\IfFileExists{microtype.sty}{% use microtype if available
  \usepackage[]{microtype}
  \UseMicrotypeSet[protrusion]{basicmath} % disable protrusion for tt fonts
}{}
\usepackage{setspace}
\makeatletter
\@ifundefined{KOMAClassName}{% if non-KOMA class
  \IfFileExists{parskip.sty}{%
    \usepackage{parskip}
  }{% else
    \setlength{\parindent}{0pt}
    \setlength{\parskip}{6pt plus 2pt minus 1pt}}
}{% if KOMA class
  \KOMAoptions{parskip=half}}
\makeatother
% Make \paragraph and \subparagraph free-standing
\makeatletter
\ifx\paragraph\undefined\else
  \let\oldparagraph\paragraph
  \renewcommand{\paragraph}{
    \@ifstar
      \xxxParagraphStar
      \xxxParagraphNoStar
  }
  \newcommand{\xxxParagraphStar}[1]{\oldparagraph*{#1}\mbox{}}
  \newcommand{\xxxParagraphNoStar}[1]{\oldparagraph{#1}\mbox{}}
\fi
\ifx\subparagraph\undefined\else
  \let\oldsubparagraph\subparagraph
  \renewcommand{\subparagraph}{
    \@ifstar
      \xxxSubParagraphStar
      \xxxSubParagraphNoStar
  }
  \newcommand{\xxxSubParagraphStar}[1]{\oldsubparagraph*{#1}\mbox{}}
  \newcommand{\xxxSubParagraphNoStar}[1]{\oldsubparagraph{#1}\mbox{}}
\fi
\makeatother


\usepackage{longtable,booktabs,array}
\usepackage{calc} % for calculating minipage widths
% Correct order of tables after \paragraph or \subparagraph
\usepackage{etoolbox}
\makeatletter
\patchcmd\longtable{\par}{\if@noskipsec\mbox{}\fi\par}{}{}
\makeatother
% Allow footnotes in longtable head/foot
\IfFileExists{footnotehyper.sty}{\usepackage{footnotehyper}}{\usepackage{footnote}}
\makesavenoteenv{longtable}
\usepackage{graphicx}
\makeatletter
\newsavebox\pandoc@box
\newcommand*\pandocbounded[1]{% scales image to fit in text height/width
  \sbox\pandoc@box{#1}%
  \Gscale@div\@tempa{\textheight}{\dimexpr\ht\pandoc@box+\dp\pandoc@box\relax}%
  \Gscale@div\@tempb{\linewidth}{\wd\pandoc@box}%
  \ifdim\@tempb\p@<\@tempa\p@\let\@tempa\@tempb\fi% select the smaller of both
  \ifdim\@tempa\p@<\p@\scalebox{\@tempa}{\usebox\pandoc@box}%
  \else\usebox{\pandoc@box}%
  \fi%
}
% Set default figure placement to htbp
\def\fps@figure{htbp}
\makeatother


% definitions for citeproc citations
\NewDocumentCommand\citeproctext{}{}
\NewDocumentCommand\citeproc{mm}{%
  \begingroup\def\citeproctext{#2}\cite{#1}\endgroup}
\makeatletter
 % allow citations to break across lines
 \let\@cite@ofmt\@firstofone
 % avoid brackets around text for \cite:
 \def\@biblabel#1{}
 \def\@cite#1#2{{#1\if@tempswa , #2\fi}}
\makeatother
\newlength{\cslhangindent}
\setlength{\cslhangindent}{1.5em}
\newlength{\csllabelwidth}
\setlength{\csllabelwidth}{3em}
\newenvironment{CSLReferences}[2] % #1 hanging-indent, #2 entry-spacing
 {\begin{list}{}{%
  \setlength{\itemindent}{0pt}
  \setlength{\leftmargin}{0pt}
  \setlength{\parsep}{0pt}
  % turn on hanging indent if param 1 is 1
  \ifodd #1
   \setlength{\leftmargin}{\cslhangindent}
   \setlength{\itemindent}{-1\cslhangindent}
  \fi
  % set entry spacing
  \setlength{\itemsep}{#2\baselineskip}}}
 {\end{list}}
\usepackage{calc}
\newcommand{\CSLBlock}[1]{\hfill\break\parbox[t]{\linewidth}{\strut\ignorespaces#1\strut}}
\newcommand{\CSLLeftMargin}[1]{\parbox[t]{\csllabelwidth}{\strut#1\strut}}
\newcommand{\CSLRightInline}[1]{\parbox[t]{\linewidth - \csllabelwidth}{\strut#1\strut}}
\newcommand{\CSLIndent}[1]{\hspace{\cslhangindent}#1}



\setlength{\emergencystretch}{3em} % prevent overfull lines

\providecommand{\tightlist}{%
  \setlength{\itemsep}{0pt}\setlength{\parskip}{0pt}}



 


\usepackage{booktabs}
\usepackage{caption}
\usepackage{longtable}
\usepackage{colortbl}
\usepackage{array}
\usepackage{anyfontsize}
\usepackage{multirow}
\usepackage{fontspec}
\usepackage{newunicodechar}
\newunicodechar{‐}{-}
\defaultfontfeatures{Ligatures=TeX,Mapping=tex-text}
\usepackage{booktabs}
\usepackage{longtable}
\usepackage{pdflscape}
\usepackage{setspace}
\doublespacing
\usepackage{float}
\usepackage{placeins}
\usepackage{setspace}
\raggedright
\hyphenpenalty=10000
\exhyphenpenalty=10000
\makeatletter
\@ifpackageloaded{caption}{}{\usepackage{caption}}
\AtBeginDocument{%
\ifdefined\contentsname
  \renewcommand*\contentsname{Table of contents}
\else
  \newcommand\contentsname{Table of contents}
\fi
\ifdefined\listfigurename
  \renewcommand*\listfigurename{List of Figures}
\else
  \newcommand\listfigurename{List of Figures}
\fi
\ifdefined\listtablename
  \renewcommand*\listtablename{List of Tables}
\else
  \newcommand\listtablename{List of Tables}
\fi
\ifdefined\figurename
  \renewcommand*\figurename{Figure}
\else
  \newcommand\figurename{Figure}
\fi
\ifdefined\tablename
  \renewcommand*\tablename{Table}
\else
  \newcommand\tablename{Table}
\fi
}
\@ifpackageloaded{float}{}{\usepackage{float}}
\floatstyle{ruled}
\@ifundefined{c@chapter}{\newfloat{codelisting}{h}{lop}}{\newfloat{codelisting}{h}{lop}[chapter]}
\floatname{codelisting}{Listing}
\newcommand*\listoflistings{\listof{codelisting}{List of Listings}}
\makeatother
\makeatletter
\makeatother
\makeatletter
\@ifpackageloaded{caption}{}{\usepackage{caption}}
\@ifpackageloaded{subcaption}{}{\usepackage{subcaption}}
\makeatother
\usepackage{bookmark}
\IfFileExists{xurl.sty}{\usepackage{xurl}}{} % add URL line breaks if available
\urlstyle{same}
\hypersetup{
  pdftitle={Trust in Information and Pandemic Approval in Southeast Asia: Evidence from the Vietnam Paradox},
  pdfauthor={Jeffrey Stark},
  pdfkeywords={COVID-19, government approval, information
trust, Southeast Asia, pandemic politics},
  colorlinks=true,
  linkcolor={blue},
  filecolor={Maroon},
  citecolor={Blue},
  urlcolor={Blue},
  pdfcreator={LaTeX via pandoc}}


\title{Trust in Information and Pandemic Approval in Southeast Asia:
Evidence from the Vietnam Paradox}
\author{Jeffrey Stark}
\date{2026-01-08}
\begin{document}
\maketitle
\begin{abstract}
Performance legitimacy theories predict citizens punish governments for
poor crisis outcomes, yet Vietnam is an anomaly: despite Southeast
Asia's highest self-reported COVID-19 infection rate (65.9\%), Vietnam
reports the region's highest approval (97.5\%) for pandemic handling.
Using Asian Barometer Wave 6 data from Cambodia, Thailand, and Vietnam
(N = 3,685), we show trust in government COVID-19 information strongly
predicts approval (β = 0.31--0.61), while infection has null effects. In
Thailand's pluralistic information environment, infection reduces
approval only among citizens who distrust official information; in
Vietnam and Cambodia, near-universal trust fully decouples outcomes from
evaluations. Results are robust across specifications (E-values
2.7--3.7). We extend work on authoritarian ``trust paradoxes'' by
identifying an information-based mechanism: when official narratives are
deemed credible, citizens attribute negative experiences to forces
beyond government control, allowing narrative credibility to substitute
for performance outcomes as a basis for regime support.
\end{abstract}


\setstretch{2}
\section{Introduction}\label{introduction}

Performance legitimacy theories rest on a foundational assumption:
citizens evaluate governments based on outcomes. The COVID-19 pandemic
provided a natural test---governments that protected citizens from
infection should enjoy higher approval than those that failed.

Vietnam upends this expectation. In the Asian Barometer Survey conducted
during 2021--2022, 65.9\% of Vietnamese respondents reported that they
or family members had contracted COVID-19---the highest self-reported
infection rate in Southeast Asia and more than seven times Cambodia's
8.7\%. Yet 97.5\% rated their government's pandemic response as ``well''
or ``very well''---also the highest in the region. The worst outcomes
coincided with the best evaluations.

This pattern echoes broader puzzles in the study of institutional trust
across Asia. Baniamin (2025) demonstrates that lower-performing
countries like Vietnam show higher trust in government than
better-performing democracies like Japan and South Korea, attributing
this to authoritarian cultural orientation. We propose a complementary
but distinct mechanism: trust in \emph{crisis-specific} government
information may function as a cognitive filter, shaping how citizens
interpret their experiences independently of diffuse cultural values.

Drawing on Asian Barometer Wave 6 data from Cambodia, Thailand, and
Vietnam (N = 3,685), we find that trust in government COVID-19
information dominates personal experience as a predictor of approval.
Moving from lowest to highest information trust corresponds to
approximately 1.5 points on a 4-point approval scale; whether a citizen
contracted COVID-19 corresponds to less than 0.08 points---substantively
negligible.

Our analysis leverages cross-national variation to test competing
accounts. If performance drives legitimacy, Cambodia (successful disease
control) should show higher approval than Vietnam (mass infection). If
information trust drives legitimacy, Vietnam and Cambodia (both
high-trust environments) should show similar approval despite divergent
outcomes---while Thailand (low trust) should show lower approval
regardless of performance. We frame Vietnam and Cambodia as ``ceiling
cases'' where near-universal trust appears to insulate governments from
experiential accountability. Thailand provides a crucial contrast: with
more pluralistic information and lower baseline trust, we observe that
infection affects approval only among citizens who distrust official
information.

This study offers three contributions:

\begin{enumerate}
\def\labelenumi{\arabic{enumi}.}
\item
  We document a robust empirical pattern wherein information trust
  dominates experiential predictors of pandemic approval---extending
  recent work on trust paradoxes in authoritarian contexts (Baniamin,
  2025; Pernia, 2022) to crisis-specific evaluations.
\item
  We propose an information-credibility heuristic: how citizens
  interpret negative experiences may matter more than whether they
  experience them, consistent with theories of informational autocracy
  (Guriev and Treisman, 2019; Morgenbesser, 2020).
\item
  We demonstrate that the Vietnam paradox survives controls for
  institutional trust, democratic attitudes, and demographics---ruling
  out obvious confounds while acknowledging our cross-sectional design
  cannot establish causality.
\end{enumerate}

\section{Theoretical Framework}\label{theoretical-framework}

\subsection{Performance Legitimacy and Its
Limits}\label{performance-legitimacy-and-its-limits}

Why do citizens support their governments? The dominant answer points to
performance: governments earn legitimacy by delivering results (Easton,
1979; Lipset, 1959; Rothstein, 2009). This rationalist approach expects
institutional trust to reflect institutional performance---when
governments deliver, citizens reward them with support (Bouckaert and
Walle, 2003; Hetherington and Nelson, 2003; Miller and Listhaug, 1990).
Applied to pandemics, the logic is straightforward: governments that
protect citizens should enjoy higher approval than those that fail.

Yet cross-national evidence increasingly challenges this expectation.
Survey data from Asia reveal that lower-performing countries often
report higher institutional trust than better-performing democracies
(Baniamin, 2025; Wong et al., 2011; Zhao and Hu, 2017). Baniamin (2025)
documents this paradox across fourteen East and Southeast Asian
countries using Asian Barometer data, finding that Vietnam reports the
highest trust in government despite middling performance indicators,
while Japan and South Korea---objectively better governed---report
substantially lower trust. He attributes this pattern to authoritarian
cultural orientation (ACO): citizens socialized into hierarchical
deference evaluate authorities less critically, inflating trust measures
independently of actual performance.

Vietnam's COVID-19 experience presents an even starker test. The country
recorded Southeast Asia's highest infection rates yet maintained the
region's highest approval ratings for pandemic handling. If performance
drives legitimacy, Vietnam should show \emph{lower} approval than
Cambodia, which successfully controlled the virus. Something other than
objective outcomes appears to sustain government support when
performance visibly fails.

\subsection{Information Credibility and Crisis
Evaluation}\label{information-credibility-and-crisis-evaluation}

We propose a complementary mechanism that operates alongside---but is
analytically distinct from---diffuse cultural orientations. Citizens
rarely possess direct knowledge of government competence; instead, they
rely on informational signals and cognitive shortcuts (Lupia, 1994;
Popkin, 1994; Sniderman et al., 1991). During crises, when uncertainty
is high and stakes are salient, these shortcuts matter even more.

We formalize this as an \emph{information-credibility heuristic}:
citizens use their trust in official crisis information as a mental
shortcut for evaluating government performance. This mechanism resonates
with Guriev and Treisman's (2019, 2020) concept of ``informational
autocracy,'' which argues that modern authoritarian regimes maintain
power less through overt repression than through control over
information flows and narrative dominance. Morgenbesser (2020) situates
such informational strategies within a broader ``menu of autocratic
innovation''---a repertoire of techniques through which regimes
cultivate the pretence of accountability without permitting its actual
practice.

The information-credibility heuristic matters because negative pandemic
experiences are attributionally ambiguous. Infection can be interpreted
as personal misfortune, an unavoidable feature of a global pandemic, or
preventable policy failure. When official information is perceived as
credible, citizens may accept government explanations and assign blame
externally---attenuating the extent to which negative experiences
translate into disapproval. When official information is distrusted, the
same experiences may convert more readily into political blame. This
attributional logic draws on foundational work in political psychology.
Iyengar (1994) demonstrates that how events are framed---as isolated
incidents or systemic patterns---shapes whether citizens assign
responsibility to individuals or institutions. Applying this to pandemic
contexts, Healy and Malhotra (2013) show that voters often fail to
distinguish between outcomes governments can control and those driven by
external forces, punishing incumbents for natural disasters and economic
shocks beyond their control. Our framework suggests that trusted
official information may help citizens make precisely this distinction:
when the government's narrative is credible, citizens can attribute
infection to the virus itself rather than to policy failure.

The mechanism operates asymmetrically across information environments.
In pluralistic settings with competing narratives, citizens receive
conflicting attributional frames---opposition voices may emphasize
government culpability while official sources emphasize external
factors. Citizens must adjudicate between frames, and those who distrust
official information will weight critical narratives more heavily. In
restricted information environments, the official frame faces less
competition. Citizens who trust this dominant narrative have fewer
alternative attributions available, potentially explaining why even
severe negative experiences fail to translate into disapproval. This
pattern is consistent with broader research on authoritarian information
environments documenting how state-controlled media shape citizens'
interpretations even when alternative sources exist (Rozenas and Zhukov,
2019; Stockmann, 2012).

Importantly, our proposed mechanism differs from Baniamin's (2025)
authoritarian cultural orientation in two ways. First, we focus on
\emph{domain-specific} trust in crisis information rather than diffuse
deference to authority. Second, we propose that information trust
actively mediates the relationship between experiences and evaluations,
rather than simply inflating all government assessments uniformly. If
our mechanism is correct, the effect of COVID information trust should
remain substantial even after controlling for institutional trust and
regime attitudes---proxies for the generalized deference Baniamin
identifies.

This framework yields two testable hypotheses:

\begin{quote}
\textbf{H1 (Information Dominance):} Trust in official COVID-19
information will be positively associated with government pandemic
approval, net of institutional trust, democratic attitudes, and
demographic controls.
\end{quote}

\begin{quote}
\textbf{H2 (Conditional Accountability):} The relationship between
personal COVID-19 infection and approval will be conditional on
information trust---negative among citizens who distrust official
information but attenuated or null among citizens who trust official
information.
\end{quote}

H1 tests whether crisis-specific information trust predicts approval
independently of diffuse regime orientations. H2 tests whether trusted
narratives insulate governments from experiential accountability---the
core claim of our information-credibility framework.

\subsection{Alternative Explanations}\label{alternative-explanations}

Several alternative mechanisms merit consideration.

\emph{Authoritarian cultural orientation} may explain high approval in
Cambodia and Vietnam independently of information trust (Dalton and Ong,
2005; Ma and Yang, 2014; Shi, 2001). Hierarchical societies socialize
citizens to defer to authority, potentially inflating approval
regardless of information environment. However, if ACO alone explains
the Vietnam paradox, controlling for institutional trust and
authoritarian acceptance should substantially reduce the information
trust coefficient---which we test directly.

\emph{Rally-around-the-flag effects} predict that external threats
generate surges in government support (Baker and Oneal, 2001; Baum,
2002; Mueller, 1970). Yet rally theory expects support to strengthen
when governments \emph{succeed} at managing threats, not when protection
visibly fails. Vietnam's high infection rates combined with high
approval inverts the expected pattern.

\emph{Prior regime legitimacy} from economic development may explain
baseline differences in support (Malesky and London, 2014). Vietnam's
rapid growth since \emph{Đổi Mới} has generated substantial performance
legitimacy. However, this cannot explain why \emph{COVID-specific}
infection and economic hardship show such weak associations with
\emph{COVID-specific} approval within Vietnam.

\emph{Social desirability bias} presents a serious concern that cannot
be fully addressed with observational data. Respondents in authoritarian
settings may report approval they do not genuinely feel. However, Asian
Barometer employs standardized anonymity protocols across all countries.
More tellingly, Vietnamese respondents openly reported the region's
highest infection rates and substantial disagreement with the statement
that ``people are free to speak without fear''---responses inconsistent
with uniform fear driving all survey answers. We return to this issue in
the Discussion.

Finally, Thailand provides a crucial comparative test. If cultural
factors common to Southeast Asia explain high approval in Vietnam and
Cambodia, Thailand---with similar regional context---should show
comparable patterns. Instead, Thailand's more pluralistic information
environment and lower baseline trust allow us to observe whether the
accountability mechanism (H2) operates differently across information
contexts.

\section{Data and Methods}\label{data-and-methods}

\subsection{Data Source and Sampling}\label{data-source-and-sampling}

We analyse Asian Barometer Survey (ABS) Wave 6 data, conducted between
late 2021 and fall 2022 across Cambodia, Thailand, and Vietnam. The ABS
employs multi-stage stratified random sampling to achieve nationally
representative samples of adult citizens (Chu, 2008).

Our analytic sample includes 3,685 respondents: Cambodia (n = 1,242),
Vietnam (n = 1,237), and Thailand (n = 1,206). Response rates ranged
from 68\% (Thailand) to 82\% (Cambodia). All surveys were conducted
face-to-face using standardised questionnaires. Listwise deletion yields
an analytic sample that varies by specification. Models using the core
variables (infection, approval, information trust, democracy
satisfaction, institutional trust) retain 3,241 respondents (88\%).
Adding demographic and economic controls reduces N due to item-level
missingness (variable completeness ranges 93--99\%). We report
model-specific sample sizes and show that results are substantively
unchanged under multiple imputation (Online Appendix C).

Case selection maximises theoretical leverage. All three are Southeast
Asian nations with similar cultural contexts and geographic proximity,
yet they vary on theoretically relevant dimensions: regime type
(Cambodia and Vietnam are authoritarian; Thailand is a hybrid regime
with greater information pluralism); pandemic outcomes (infection rates
varied dramatically); and information environments (Freedom House, 2025;
Reporters Without Borders, 2023).

Data collection timing varied: Cambodia (primarily December 2021),
Thailand (April--May 2022), and Vietnam (August--September 2022). This
temporal variation paradoxically strengthens our core finding: Vietnam
was surveyed during objectively worse conditions yet shows higher
approval.

\subsection{Case Context: Pandemic
Trajectories}\label{case-context-pandemic-trajectories}

The three countries experienced COVID-19 along strikingly different
trajectories, making cross-national comparison theoretically
informative.

\textbf{Vietnam} initially achieved remarkable success. Through early
2021, the country reported fewer than 3,000 total cases despite sharing
a border with China and maintaining substantial cross-border trade
(World Health Organization, 2021). The government implemented aggressive
contact tracing, centralised quarantine facilities, and strict mobility
restrictions, earning international recognition for its containment
strategy (La et al., 2020; Pollack et al., 2020). This changed
dramatically with the Delta variant's arrival in April 2021. By early
2023, Vietnam had recorded over 11.5 million confirmed cases (WHO,
2023)---transforming from regional success story to the country with
Southeast Asia's highest per-capita infection rate. Our survey,
conducted in August--September 2022, captured citizens evaluating this
trajectory (World Health Organization, 2023). Critically, our survey
captured citizens evaluating a government whose early success had given
way to mass infection.

\textbf{Cambodia} maintained low infection rates throughout. The country
reported fewer than 140,000 total cases by late 2022, achieving one of
the region's lowest per-capita rates. The government leveraged its
authoritarian capacity to enforce strict lockdowns, particularly in
Phnom Penh during the February 2021 outbreak, and achieved high
vaccination coverage through centralised distribution. When surveyed in
December 2021, Cambodian respondents were evaluating a government that
had---by objective epidemiological measures---protected them from the
pandemic.

\textbf{Thailand} occupied a middle position epidemiologically but
diverged sharply in its political context. The country experienced
moderate infection waves, with approximately 4.4 million cumulative
cases by survey time. More consequentially, Thailand's pandemic response
became deeply politicized. The government faced sustained protests
throughout 2020--2021, with demonstrators explicitly linking pandemic
mismanagement to broader critiques of the military-backed
administration. Unlike Vietnam and Cambodia, Thai citizens had access to
critical media coverage and opposition voices challenging official
narratives (Kongkirati and Kanchoochat, 2018). The survey (April--May
2022) thus captured a population with both moderate pandemic exposure
and pluralistic information access.

These divergent trajectories create analytical leverage. Vietnam and
Cambodia share authoritarian governance and restricted information
environments but experienced opposite pandemic outcomes. Thailand shares
Southeast Asian cultural context but differs in information pluralism.
If performance drives approval, Cambodia should outperform Vietnam. If
information environment matters, Vietnam and Cambodia should converge
despite divergent outcomes---while Thailand should diverge despite
moderate performance.

\subsection{Measurement Strategy}\label{measurement-strategy}

\textbf{Dependent variable.} Government pandemic approval: ``How well or
badly do you think the government handled the pandemic?'' (1 = Very
badly to 4 = Very well; mean = 3.29, SD = 0.87).

\textbf{Primary independent variables.} \emph{Personal COVID-19
infection}: binary indicator based on ``Have you or your family members
previously contracted the Covid-19 virus?'' (overall rate 39.8\%;
Cambodia 8.7\%, Thailand 39.9\%, Vietnam 65.9\%). \emph{COVID-19
economic impact severity}: four-point scale (mean = 2.31, SD = 0.94).
\emph{Trust in government COVID information}: ``How much do you trust
the Covid-19 related information provided by the government?'' (mean =
3.21, SD = 0.79).

\textbf{Control variables.} Institutional trust (9-item index, α =
0.947), democracy satisfaction, authoritarian acceptance (α = 0.849),
and demographic controls (age, gender, education, urban residence).

\subsection{Measurement Validity}\label{measurement-validity}

A potential concern is that trust in government COVID information and
government pandemic approval may be tautological. The correlation
between trust and approval in Vietnam is moderate (r = 0.51), indicating
shared variance of 26\%---substantial but leaving 74\% unexplained. VIF
scores range from 1 to 1.63, well below conventional thresholds,
indicating these constructs are empirically distinguishable.

\subsection{Analytic Strategy}\label{analytic-strategy}

Our analysis proceeds in four stages: descriptive analysis documenting
the Vietnam paradox; bivariate OLS regression testing associations
between COVID impacts and approval within each country; multivariate OLS
regression controlling for potential confounders; and robustness checks
(Online Appendix). We estimate models separately by country and use
heteroskedasticity-consistent (HC3) standard errors.

\section{Results}\label{results}

\subsection{The Aggregate Paradox}\label{the-aggregate-paradox}

The numbers are stark. Vietnam's infection rate (65.9\%) was more than
seven times Cambodia's (8.7\%) and substantially exceeded Thailand's
(39.9\%). Yet Vietnam recorded the region's highest government approval
(97.5\%), outpacing Cambodia (93.6\%) and dramatically exceeding
Thailand (37.5\%). Table 1 summarizes this paradox.

{[}INSERT TABLE 1 ABOUT HERE{]}

\subsection{Infection Does Not Matter}\label{infection-does-not-matter}

Did Vietnamese citizens who contracted COVID-19 evaluate their
government more harshly? The answer is no---personal experience showed
essentially no relationship with evaluations.

In Cambodia, the association between infection and approval was
effectively zero (β = -0.014, p = 0.72). Vietnam showed a slight
\emph{positive} coefficient (β = 0.005, p = 0.8)---infected respondents
were marginally more approving. Thailand exhibited a statistically
significant but substantively trivial negative relationship (β = -0.072,
p \textless{} 0.05). Whether a citizen contracted COVID-19 was
essentially irrelevant to how they evaluated the government.

Economic hardship showed inconsistent effects. Thailand displayed the
expected negative pattern (β = -0.101, p \textless{} 0.001); Vietnam and
Cambodia showed no significant relationship.

Information trust dominated. Cambodia (β = 0.549, p \textless{} 0.001),
Vietnam (β = 0.423, p \textless{} 0.001), and Thailand (β = 0.725, p
\textless{} 0.001) all showed substantial coefficients, with trust alone
explaining 26--45\% of variance in approval (Figure 1). These bivariate
patterns are consistent with H1 but could reflect confounding with
generalized regime support. We turn to multivariate tests.

{[}INSERT FIGURE 1 ABOUT HERE{]}

\subsection{Multivariate Results}\label{multivariate-results}

H1 predicts that information trust will predict pandemic approval net of
institutional trust, democratic attitudes, and demographic controls.
Table 2 presents multivariate OLS models testing this expectation.

The trust coefficient survives intact. Across all countries and
specifications, information trust remains the dominant predictor (β =
0.31 to 0.61, all p \textless{} 0.001), while infection status remains
non-significant. Critically, controlling for institutional trust---a
proxy for the diffuse regime support that Baniamin (2025) attributes to
authoritarian cultural orientation---reduces the COVID information trust
coefficient by only 16--26\%, meaning it retains approximately
three-quarters of its bivariate strength. This suggests the
information-credibility mechanism operates largely independently of
generalized deference.

A respondent moving from lowest to highest information trust gains
approximately 1.5 points on the 4-point approval scale---nearly half the
entire range. By contrast, infection status corresponds to less than 0.1
points. The predictive dominance of information trust over personal
experience strongly supports H1.

{[}INSERT TABLE 2 ABOUT HERE{]}

\subsection{Mechanism Test: Conditional
Accountability}\label{mechanism-test-conditional-accountability}

H2 predicts that infection should reduce approval primarily among
citizens who distrust official information---the accountability
mechanism should operate only when trusted narratives do not insulate
the government from blame. Thailand provides the clearest test.

In Thailand, where baseline approval (37.5\%) and trust (33.8\%) are
both low, we observe a significant negative interaction between
infection and trust (see Online Appendix Table A3). Among low-trust Thai
respondents, personal infection is associated with substantially lower
approval---the expected accountability response. Among high-trust
respondents, infection shows no relationship with approval. Trusted
narratives appear to fully insulate the government from negative
pandemic experiences, consistent with H2.

Vietnam and Cambodia present null interaction effects---but this
\emph{supports} rather than contradicts H2. These represent ``ceiling
cases'' where near-universal trust (91.9\% and 86.6\%) has already
insulated governments from experiential accountability across virtually
the entire population. The information-credibility heuristic operates as
the universal baseline rather than a conditional moderator. With
insufficient variation in trust, we cannot observe the low-trust
accountability mechanism that Thailand reveals. The null interaction
reflects not the absence of information effects but their completeness.

\subsection{Robustness}\label{robustness}

We subjected findings to extensive sensitivity testing (full results in
Online Appendix).

\emph{Construct validity}: COVID information trust predicts pandemic
approval 2--4 times more strongly than non-pandemic outcomes (democracy
satisfaction, institutional trust), confirming domain-specificity rather
than generalized regime loyalty. This addresses concerns that our
mechanism merely captures diffuse support.

\emph{Alternative specifications}: Ordinal logistic models treating
approval as ordered categorical yield identical conclusions. Results are
robust to robust regression, quantile regression, and survey-weighted
estimation.

\emph{Sensitivity analysis}: E-values range from 2.70 to 3.71, meaning
an unmeasured confounder would need to nearly triple the risk of both
high trust and high approval---simultaneously and independently of all
observed covariates---to fully explain away the findings. For context,
confounders of this magnitude are rare in observational research; most
known confounders in political behaviour studies have substantially
smaller effects.

\emph{Social desirability}: 40.3\% of Vietnamese respondents openly
disagreed that ``people are free to speak without fear''---yet among
these respondents, 97.5\% still approved of pandemic handling, virtually
identical to those who feel free to speak (97.3\%). Vietnamese
respondents also openly reported the region's highest infection rates.
These off-diagonal responses are difficult to reconcile with uniform
fear driving survey responses.

\section{Discussion}\label{discussion}

\subsection{The Information-Credibility
Heuristic}\label{the-information-credibility-heuristic}

Our findings support both hypotheses. Trust in government COVID-19
information is substantially more strongly associated with pandemic
approval than personal infection or economic hardship---across all three
countries, all specifications, and robust to extensive sensitivity
testing (H1). In Thailand, infection reduces approval among low-trust
citizens but not among high-trust citizens, demonstrating that trusted
narratives can insulate governments from experiential accountability
(H2). The Vietnam Paradox is real: the region's highest infection rate
coincided with its highest approval.

The information-credibility heuristic offers a compelling
interpretation: trust in official communication functions as a cognitive
filter through which citizens interpret their experiences. When official
information is deemed credible, negative events are attributed to forces
beyond government control---the virus itself, global circumstances, or
personal misfortune. When distrusted, the same events are read as
government failures. This explains why infection showed null
associations with approval in Vietnam and Cambodia: the \emph{meaning}
of pandemic experiences was shaped by nearly universally trusted
official narratives.

Thailand and Vietnam represent opposite poles of this framework. In
Thailand's more pluralistic information environment, the accountability
mechanism operates as performance legitimacy theories predict---but only
among citizens who distrust official information. In Vietnam,
population-wide adoption of official frames appears to have decoupled
personal experience from political evaluation entirely. This poses a
fundamental challenge to performance legitimacy theories: if the state
can maintain a trusted narrative, objective failure need not translate
into political consequences.

\subsection{Limitations and Alternative
Interpretations}\label{limitations-and-alternative-interpretations}

Our cross-sectional design cannot establish causal direction. Trust may
shape approval, approval may shape trust, or both may reflect a common
orientation toward the regime. We measure \emph{trust} in government
information, not information control, propaganda effectiveness, or
censorship. High trust in Vietnam could reflect successful narrative
management---or genuinely effective communication, cultural deference,
or social desirability bias. Throughout, we distinguish between our
measured construct (individual-level trust in official COVID-19
information) and unmeasured contextual factors (information
environments, media systems, censorship regimes) that might explain
cross-national variation in that trust.

Several patterns favour the information-credibility interpretation over
alternatives. First, and most critically for the reverse causality
concern, COVID-specific trust remains the dominant predictor after
controlling for institutional trust and democracy
satisfaction---measures that should capture generalized regime support.
If high trust in COVID information merely reflected a ``halo effect''
from pre-existing regime loyalty, adding these controls should
substantially attenuate the trust coefficient. Instead, it declines by
only 16--26\%, retaining approximately three-quarters of its original
strength, and the effect size remains substantively large (β
\textgreater{} 0.31 across all countries). This pattern is difficult to
reconcile with the view that information trust is simply a proxy for
diffuse regime support; rather, it suggests that crisis-specific trust
operates through a distinct pathway. Second, VIF diagnostics confirm
trust and approval are empirically distinguishable (VIF = 1--1.63), with
shared variance of only 26\%. Third, E-value analysis shows unmeasured
confounders would need risk ratios of 2.7--3.7 to explain away
findings---exceeding plausible confounding scenarios. Fourth, and most
importantly, the Thailand interaction results demonstrate that infection
affects approval \emph{differently} depending on trust levels. This
conditional pattern is difficult to explain if trust merely correlates
with approval rather than functioning as a cognitive filter.

A related alternative is that Vietnam's high approval reflects a
``legacy effect''---citizens rewarding the government's successful 2020
containment rather than evaluating current conditions. On this account,
early success earned a reservoir of legitimacy that citizens drew upon
even as infections surged. Our information-credibility heuristic differs
in a crucial respect: it posits that trust in official information
\emph{actively shapes} how citizens interpret ongoing experiences, not
merely that past performance generates lasting goodwill. The distinction
matters empirically. If legacy effects drove approval, we would expect
institutional trust (which captures accumulated regime legitimacy) to
dominate COVID-specific information trust once both are in the model.
Instead, Table 2 shows the opposite: COVID information trust remains the
stronger predictor even controlling for institutional trust. Citizens
appear to be filtering \emph{current} experiences through \emph{current}
trust in official narratives, not simply coasting on stored legitimacy
from 2020.

How do our findings relate to Baniamin's (2025) authoritarian cultural
orientation (ACO) framework? Using Asian Barometer Wave 4 data across
fourteen countries, Baniamin finds that ACO---measured as unquestioning
obedience to authority---contributes to higher institutional trust
independently of actual governance performance. Our findings are
complementary rather than contradictory. ACO likely explains \emph{why}
citizens in Vietnam and Cambodia trust official COVID-19 information so
readily in the first place: societies with stronger hierarchical
orientations may be predisposed to accept official narratives. But our
analysis suggests that once established, this trust actively shapes how
citizens interpret their experiences---functioning as a mediating
mechanism rather than simply inflating all government assessments
uniformly. Future research might test whether ACO moderates the
relationship between information trust and approval, or whether
information trust mediates the ACO-approval relationship.

Alternative accounts cannot be ruled out. Vietnam's high trust could
reflect genuinely effective crisis communication rather than
manipulation---the government's early COVID-19 response was
internationally praised (Pollack et al., 2020), and citizens may have
accurately perceived competent management even as infections eventually
surged. Cultural variation in deference to authority may contribute
independently of our measured trust variable (Kerkvliet, 2015; Ma and
Yang, 2014; Thayer, 2011). Survey timing strengthens rather than weakens
our findings: Vietnam was surveyed during the Omicron wave when
infections were peaking---respondents were evaluating performance while
actively experiencing the outbreak, not recalling distant memories.

Social desirability bias deserves particular attention. Respondents in
authoritarian settings may report approval they do not genuinely feel.
We cannot definitively rule this out with observational data. However,
several patterns suggest social desirability does not fully explain our
findings. Vietnamese respondents openly reported the region's highest
infection rates---information that could embarrass the government. More
tellingly, 40.3\% of Vietnamese respondents openly disagreed that
``people are free to speak without fear,'' yet among these same
respondents who perceive speech constraints, 97.5\% still approved of
pandemic handling---virtually identical to approval among those who feel
free to speak. If fear uniformly inflates all pro-government responses,
we would expect those who perceive constraints to show \emph{lower}
approval, or at least to avoid criticizing speech freedoms in the first
place. The off-diagonal pattern---critical of speech freedoms but
approving of pandemic handling---suggests respondents distinguish
between domains rather than uniformly deferring.

\subsection{Implications}\label{implications}

For rally-around-the-flag theory, our findings suggest health crises do
not automatically generate government support (Hegewald and Schraff,
2024; Kritzinger et al., 2021). The rally mechanism appears conditional
on information environment characteristics: where citizens trust
official narratives, support may emerge regardless of outcomes; where
they distrust official narratives, the expected accountability pattern
operates.

For theories of retrospective voting and democratic accountability more
broadly, our findings highlight the informational preconditions for
outcome-based evaluation. The canonical retrospective voting model
assumes citizens can observe outcomes and attribute responsibility
(Fiorina, 1978; Healy and Malhotra, 2013). Our results suggest this
assumption may fail when official narratives successfully frame outcomes
as beyond government control. The Thailand results are instructive:
accountability functioned among citizens who distrusted official
information, suggesting the mechanism itself operates---but only when
citizens have access to, and confidence in, alternative attributional
frames. Democratic accountability may thus depend not only on electoral
institutions but on information environments that permit critical
evaluation of official narratives.

For public health communication, these findings present a double-edged
implication. On one hand, maintaining public trust in official health
information appears consequential for social cohesion during
crises---citizens who trust official sources may be more resilient to
the psychological and political disruptions of negative health outcomes.
On the other hand, this same dynamic could insulate governments from
accountability for genuine policy failures, reducing incentives for
effective crisis management. The normative valence depends on whether
high trust reflects genuinely competent communication or successful
narrative control despite poor performance---a distinction our
cross-sectional data cannot adjudicate.

For performance legitimacy theory, the disconnect between outcomes and
approval in Vietnam and Cambodia suggests legitimacy may derive from
sources other than policy effectiveness---at least in certain
information environments. Our findings provide individual-level evidence
consistent with Guriev and Treisman's (2019) informational autocracy
framework: when citizens trust official narratives, objective
performance becomes less consequential for political support. This
extends their macro-level argument about regime survival to micro-level
survey evidence about individual attitudes.

For public health communication, these findings suggest that maintaining
information credibility may be as consequential for public trust and
social cohesion as epidemiological outcomes themselves. Governments that
squander credibility early in a crisis may face accountability pressures
even if they subsequently improve performance; governments that maintain
credibility may retain support even through difficult outcomes.

We caution against normative conclusions about ``authoritarian
advantages'' in crisis management. Thailand's low approval could
indicate democratic accountability functioning appropriately---citizens
punishing a government they perceive as having failed---rather than
democratic vulnerability. The patterns we document are troubling
primarily for theories that assume citizens can and do evaluate
government performance based on outcomes. If narrative credibility can
substitute for actual performance, the informational foundations of
democratic accountability deserve closer scrutiny.

\section{Conclusion}\label{conclusion}

The Vietnam Paradox challenges the core logic of performance-based
legitimacy. Despite recording Southeast Asia's highest COVID-19
infection rate, Vietnam maintained the region's highest approval for
government pandemic handling. Citizens' evaluations were tied far more
closely to trust in official information than to personal health
outcomes---the predictive power of information trust exceeded that of
infection by an order of magnitude.

Our findings support an information-credibility framework. Trust in
government COVID-19 information strongly predicted approval (β = 0.42 to
0.72), explaining 26--45\% of variance, while personal infection showed
null effects across all three countries. Thailand provides more than a
contrast---it validates the underlying theory. Among Thai citizens who
distrusted official information, infection reduced approval exactly as
performance-legitimacy theories predict. The accountability mechanism is
not absent; it is \emph{conditional} on information environment. When
the cognitive filter of trusted narratives is removed, the
performance-legitimacy link functions normally. This transforms our
findings from a regional anomaly into evidence for a general theory:
information trust moderates whether citizens translate personal
experiences into political evaluations. Vietnam and Cambodia represent
ceiling cases where near-universal trust has decoupled outcomes from
evaluations entirely.

These findings complement recent work on trust paradoxes in
authoritarian contexts (Baniamin, 2025) by identifying a domain-specific
mechanism: crisis information trust appears to shape how citizens
interpret negative experiences, functioning as a cognitive filter rather
than simply reflecting diffuse cultural deference. When official
narratives are deemed credible, citizens may attribute suffering to
forces beyond government control rather than to policy failure.

We acknowledge limitations. Our cross-sectional design cannot establish
causality, and high trust could reflect genuinely effective
communication rather than narrative manipulation. The interpretation we
favour awaits confirmation from longitudinal or experimental designs.

But what the evidence establishes is a robust empirical pattern
demanding theoretical attention. If governments can maintain trusted
narratives during crises, objective outcomes may become politically
irrelevant---a troubling prospect for theories grounding democratic
accountability in citizens' ability to evaluate governmental
performance.

\section*{References}\label{references}
\addcontentsline{toc}{section}{References}

\phantomsection\label{refs}
\begin{CSLReferences}{1}{1}
\bibitem[\citeproctext]{ref-Baker2001-kz}
Baker WD and Oneal JR (2001) {Patriotism or opinion leadership?: The
nature and origins of the "rally 'round the flag" effect}. \emph{The
Journal of Conflict Resolution} 45: 661--688.

\bibitem[\citeproctext]{ref-Baniamin2025-rx}
Baniamin HM (2025) {Institutional trust and its connection with
governance quality and performance: Understanding the East and Southeast
Asian paradox}. \emph{Asian Journal of Comparative Politics}. Epub ahead
of print 13 March 2025. DOI:
\href{https://doi.org/10.1177/20578911251326142}{10.1177/20578911251326142}.

\bibitem[\citeproctext]{ref-Baum2002-tc}
Baum MA (2002) \href{https://doi.org/10.1111/1468-2478.00232}{{The
constituent foundations of the rally-round-the-flag phenomenon}}.
\emph{International Studies Quarterly} 46: 263--298.

\bibitem[\citeproctext]{ref-Bouckaert2003-ky}
Bouckaert G and Walle S van de (2003)
\href{https://doi.org/10.1177/0020852303693003}{{Comparing measures of
citizen trust and user satisfaction as indicators of {`good
governance'}: Difficulties in linking trust and satisfaction
indicators}}. \emph{International Review of Administrative Sciences} 69:
329--343.

\bibitem[\citeproctext]{ref-Chu2008-st}
Chu Y-H (2008) \emph{{How East Asians view democracy}} (eds Y-H Chu, AJ
Choffnes, L Diamond, et al.). Columbia University Press.

\bibitem[\citeproctext]{ref-Dalton2005-su}
Dalton RJ and Ong N-NT (2005)
\href{https://doi.org/10.1017/s1468109905001842}{{Authority orientations
and democratic attitudes: A test of the {`Asian values'} hypothesis}}.
\emph{Japanese Journal of Political Science} 6: 211--231.

\bibitem[\citeproctext]{ref-Easton1979-bg}
Easton D (1979) \emph{{A systems analysis of political life}}.
University of Chicago Press.

\bibitem[\citeproctext]{ref-Fiorina1978-wf}
Fiorina MP (1978) \href{https://doi.org/10.2307/2110623}{{Economic
retrospective voting in American national elections: A micro-analysis}}.
\emph{American Journal of Political Science} 22: 426.

\bibitem[\citeproctext]{ref-Freedom-House2025-gx}
Freedom House (2025)
\emph{\href{https://doi.org/10.5040/9798216424611}{{Freedom in the world
2022: The annual survey of political rights and civil liberties}}} (eds
S Repucci, A Slipowitz, S O'Toole, et al.). Rowman \& Littlefield.

\bibitem[\citeproctext]{ref-Guriev2019-xj}
Guriev S and Treisman D (2019)
\href{https://doi.org/10.1257/jep.33.4.100}{{Informational autocrats}}.
\emph{The Journal of Economic Perspectives: A Journal of the American
Economic Association} 33: 100--127.

\bibitem[\citeproctext]{ref-Guriev2020-cu}
Guriev S and Treisman D (2020)
\href{https://doi.org/10.1016/j.jpubeco.2020.104158}{{A theory of
informational autocracy}}. \emph{Journal of Public Economics} 186:
104158.

\bibitem[\citeproctext]{ref-Healy2013-sy}
Healy A and Malhotra N (2013)
\href{https://doi.org/10.1146/annurev-polisci-032211-212920}{{Retrospective
voting reconsidered}}. \emph{Annual Review of Political Science} 16:
285--306.

\bibitem[\citeproctext]{ref-Hegewald2024-ii}
Hegewald S and Schraff D (2024)
\href{https://doi.org/10.1080/17457289.2022.2120886}{{Who rallies around
the flag? Evidence from panel data during the Covid-19 pandemic}}.
\emph{Journal of Elections, Public Opinion and Parties} 34: 158--179.

\bibitem[\citeproctext]{ref-Hetherington2003-oq}
Hetherington MJ and Nelson M (2003)
\href{https://doi.org/10.1017/s1049096503001665}{{Anatomy of a rally
effect: George W. bush and the war on terrorism}}. \emph{PS, Political
Science \& Politics} 36: 37--42.

\bibitem[\citeproctext]{ref-Iyengar1994-fk}
Iyengar S (1994) \emph{{Is Anyone Responsible?: How television frames
political issues}}. University of Chicago Press.

\bibitem[\citeproctext]{ref-Kerkvliet2015-lp}
Kerkvliet BJT (2015)
\href{https://doi.org/10.4324/9781315674735-31}{{Democracy in Thailand:
theory and practice}}. In: Case W (ed.) \emph{{Routledge Handbook of
Southeast Asian Democratization}}. Routledge, pp. 363--381.

\bibitem[\citeproctext]{ref-Kongkirati2018-go}
Kongkirati P and Kanchoochat V (2018)
\href{https://doi.org/10.1017/trn.2018.4}{{The Prayuth regime: Embedded
military and hierarchical capitalism in Thailand}}. \emph{TRaNS Trans
-Regional and -National Studies of Southeast Asia} 6: 279--305.

\bibitem[\citeproctext]{ref-Kritzinger2021-im}
Kritzinger S, Foucault M, Lachat R, et al. (2021)
\href{https://doi.org/10.1080/01402382.2021.1925017}{{{`Rally round the
flag'}: the COVID-19 crisis and trust in the national government}}.
\emph{West European Politics} 44: 1205--1231.

\bibitem[\citeproctext]{ref-La2020-jv}
La V-P, Pham T-H, Ho M-T, et al. (2020)
\href{https://doi.org/10.3390/su12072931}{{Policy response, social media
and science journalism for the sustainability of the public health
system amid the COVID-19 outbreak: The Vietnam lessons}}.
\emph{Sustainability} 12: 2931.

\bibitem[\citeproctext]{ref-Lipset1959-ur}
Lipset SM (1959) \href{https://doi.org/10.2307/1951731}{{Some social
requisites of democracy: Economic development and political
legitimacy}}. \emph{The American Political Science Review} 53: 69--105.

\bibitem[\citeproctext]{ref-Lupia1994-jp}
Lupia A (1994) \href{https://doi.org/10.2307/2944882}{{Shortcuts versus
encyclopedias: Information and voting behavior in California insurance
reform elections}}. \emph{The American Political Science Review} 88:
63--76.

\bibitem[\citeproctext]{ref-Ma2014-jj}
Ma D and Yang F (2014)
\href{https://doi.org/10.1007/s12140-014-9217-z}{{Authoritarian
orientations and political trust in East Asian societies}}. \emph{East
Asia} 31: 323--341.

\bibitem[\citeproctext]{ref-Malesky2014-wz}
Malesky E and London J (2014)
\href{https://doi.org/10.1146/annurev-polisci-041811-150032}{{The
political economy of development in China and Vietnam}}. \emph{Annual
Review of Political Science} 17: 395--419.

\bibitem[\citeproctext]{ref-Miller1990-qm}
Miller AH and Listhaug O (1990)
\href{https://doi.org/10.1017/s0007123400005883}{{Political parties and
confidence in government: A comparison of Norway, Sweden and the United
States}}. \emph{British Journal of Political Science} 20: 357--386.

\bibitem[\citeproctext]{ref-Morgenbesser2020-ya}
Morgenbesser L (2020)
\href{https://doi.org/10.1080/13510347.2020.1746275}{{The menu of
autocratic innovation}}. \emph{Democratization} 27: 1053--1072.

\bibitem[\citeproctext]{ref-Mueller1970-kf}
Mueller JE (1970) \href{https://doi.org/10.2307/1955610}{{Presidential
popularity from Truman to Johnson}}. \emph{The American Political
Science Review} 64: 18--34.

\bibitem[\citeproctext]{ref-Pernia2022-gh}
Pernia RA (2022)
\href{https://doi.org/10.1177/2057891121992118}{{Authoritarian values
and institutional trust: Theoretical considerations and evidence from
the Philippines}}. \emph{Asian Journal of Comparative Politics} 7:
204--232.

\bibitem[\citeproctext]{ref-Pollack2020-xq}
Pollack T, Thwaites G, Rabaa M, et al. (2020) {Emerging COVID-19 success
story: Vietnam's commitment to containment}. \emph{Our World In Data}
2020: 1--15.

\bibitem[\citeproctext]{ref-Popkin1994-yv}
Popkin SL (1994) \emph{{The reasoning voter: Communication and
persuasion in presidential campaigns}}. 2nd ed. University of Chicago
Press.

\bibitem[\citeproctext]{ref-Reporters-Without-Borders2023-ks}
Reporters Without Borders (2023) \emph{{RSF's 2022 World Press Freedom
Index : a new era of polarisation}}. research report. Reporters Without
Borders.

\bibitem[\citeproctext]{ref-Rothstein2009-lr}
Rothstein B (2009)
\href{https://doi.org/10.1177/0002764209338795}{{Creating political
legitimacy: Electoral democracy versus quality of government}}.
\emph{The American Behavioral Scientist} 53: 311--330.

\bibitem[\citeproctext]{ref-Rozenas2019-jk}
Rozenas A and Zhukov YM (2019)
\href{https://doi.org/10.1017/s0003055419000066}{{Mass repression and
political loyalty: Evidence from Stalin's {`terror by hunger'}}}.
\emph{The American Political Science Review} 113: 569--583.

\bibitem[\citeproctext]{ref-Shi2001-ga}
Shi T (2001) \href{https://doi.org/10.2307/422441}{{Cultural values and
political trust: A comparison of the people's republic of China and
Taiwan}}. \emph{Comparative Politics} 33: 401.

\bibitem[\citeproctext]{ref-Sniderman1991-uj}
Sniderman PM, Brody RA and Tetlock PE (1991)
\emph{\href{https://doi.org/10.1017/cbo9780511720468}{{Cambridge studies
in public opinion and political psychology: Reasoning and choice:
Explorations in political psychology: Explorations in political
psychology}}} (eds HE Brady, JE Chubb, PJ Feld, et al.). Cambridge
University Press.

\bibitem[\citeproctext]{ref-Stockmann2012-xq}
Stockmann D (2012)
\emph{\href{https://doi.org/10.1017/cbo9781139087742}{{Communication,
society and politics: Media commercialization and authoritarian rule in
China}}}. Communication, society, and politics. Cambridge University
Press.

\bibitem[\citeproctext]{ref-Thayer2011-te}
Thayer C (2011)
\emph{\href{https://doi.org/10.1057/9781137001474}{{Political legitimacy
in Asia: New leadership challenges}}} (eds J Kane, H-C Loy, and H
Patapan). Palgrave series in asian governance. Palgrave Macmillan.

\bibitem[\citeproctext]{ref-Wong2011-zv}
Wong TK-Y, Wan P-S and Hsiao H-HM (2011)
\href{https://doi.org/10.1177/0192512110378657}{{The bases of political
trust in six Asian societies: Institutional and cultural explanations
compared}}. \emph{International Political Science Review} 32: 263--281.

\bibitem[\citeproctext]{ref-WHO2021-vietnam-sitrep34}
World Health Organization (2021) \emph{Coronavirus disease 2019
({COVID}-19) situation report \#34: {Viet Nam}}. WHO Western Pacific
Region. Available at:
\url{https://www.who.int/docs/default-source/wpro---documents/countries/viet-nam/covid-19/viet-nam-moh-who-covid-19-sitrep_21mar2021.pdf}.

\bibitem[\citeproctext]{ref-WHO2023-vietnam-sitrep107}
World Health Organization (2023) \emph{Coronavirus disease 2019
({COVID}-19) situation report \#107: {Viet Nam}}. WHO Western Pacific
Region. Available at:
\url{https://www.who.int/docs/default-source/wpro---documents/countries/viet-nam/covid-19/viet-nam-moh-who-covid-19-sitrep--107_28feb2023.pdf}.

\bibitem[\citeproctext]{ref-Zhao2017-yw}
Zhao D and Hu W (2017) \href{https://doi.org/10.3917/risa.832.0365}{{Les
déterminants de la confiance du public dans le gouvernement : données
empiriques en provenance de la Chine urbaine}}. \emph{Revue
internationale des sciences administratives} 83: 365--384.

\end{CSLReferences}

\newpage{}

\section*{Tables and Figures}\label{tables-and-figures}
\addcontentsline{toc}{section}{Tables and Figures}

\begin{table}

\caption{\label{tbl-paradox}The Vietnam Paradox---COVID-19 Infection
Rates and Government Approval by Country}

\centering{

\caption*{
{\fontsize{9}{11}\selectfont  The Vietnam Paradox\fontsize{8}{9}\selectfont }
} 
\fontsize{8.0pt}{9.0pt}\selectfont
\begin{tabular*}{1\linewidth}{@{\extracolsep{\fill}}>{\centering\arraybackslash}p{\dimexpr 75.00pt -2\tabcolsep-1.5\arrayrulewidth}>{\raggedleft\arraybackslash}p{\dimexpr 45.00pt -2\tabcolsep-1.5\arrayrulewidth}>{\raggedleft\arraybackslash}p{\dimexpr 90.00pt -2\tabcolsep-1.5\arrayrulewidth}>{\raggedleft\arraybackslash}p{\dimexpr 90.00pt -2\tabcolsep-1.5\arrayrulewidth}>{\raggedleft\arraybackslash}p{\dimexpr 97.50pt -2\tabcolsep-1.5\arrayrulewidth}}
\toprule
Country & N & Infection Rate (\%) & Approval Rate (\%) & Trust COVID Info (\%) \\ 
\midrule\addlinespace[2.5pt]
Thailand & 1,206.0 & 39.9 & 37.5 & 33.8 \\ 
Vietnam & 1,237.0 & 65.9 & 97.5 & 91.9 \\ 
Cambodia & 1,242.0 & 8.7 & 93.6 & 86.6 \\ 
\bottomrule
\end{tabular*}
\begin{minipage}{\linewidth}
Source: Asian Barometer Wave 6 (2021-2022). Infection = \% reporting personal/family COVID infection. Approval = \% rating handling as 'well'/'very well'. Trust = \% reporting 'quite a lot'/'great deal' of trust.\\
\end{minipage}

}

\end{table}%

\newpage{}

\phantomsection\label{cell-fig-coefplot}
\begin{figure}[H]

\centering{

\pandocbounded{\includegraphics[keepaspectratio]{manuscript_files/figure-pdf/fig-coefplot-1.pdf}}

}

\caption{\label{fig-coefplot}Coefficient Plot with 95\% Confidence
Intervals---Bivariate Associations Between COVID-19 Impacts and
Government Approval}

\end{figure}%

\newpage{}

\begin{table}

\caption{\label{tbl-multivariate}Multivariate OLS
Regression---Determinants of Government Pandemic Approval}

\centering{

\caption*{
{\fontsize{8}{10}\selectfont  Multivariate OLS Regression Results\fontsize{7}{8}\selectfont } \\ 
{\fontsize{14}{17}\selectfont  DV: Government Pandemic Handling (1--4)\fontsize{7}{8}\selectfont }
} 
\fontsize{7.0pt}{8.0pt}\selectfont
\begin{tabular*}{1\linewidth}{@{\extracolsep{\fill}}lrrrrrr}
\toprule
 & \multicolumn{2}{c}{Cambodia} & \multicolumn{2}{c}{Vietnam} & \multicolumn{2}{c}{Thailand} \\ 
\cmidrule(lr){2-3} \cmidrule(lr){4-5} \cmidrule(lr){6-7}
Variable & Cambodia (1) & Cambodia (2) & Vietnam (1) & Vietnam (2) & Thailand (1) & Thailand (2) \\ 
\midrule\addlinespace[2.5pt]
COVID Infected & 0.003 (0.032) & -0.029 (0.033) & 0.022 (0.018) & 0.029 (0.020) & -0.038 (0.023) & -0.033 (0.025) \\ 
Trust COVID Info & 0.497*** (0.025) & 0.418*** (0.028) & 0.325*** (0.024) & 0.314*** (0.025) & 0.628*** (0.028) & 0.606*** (0.031) \\ 
Institutional Trust & 0.088** (0.027) & 0.091** (0.028) & 0.102*** (0.028) & 0.113*** (0.029) & 0.046 (0.027) & 0.013 (0.029) \\ 
Democracy Satisfaction & 0.070** (0.022) & 0.080*** (0.023) & 0.125*** (0.024) & 0.111*** (0.025) & 0.159*** (0.028) & 0.128*** (0.031) \\ 
Authoritarian Acceptance & --- & 0.042* (0.020) & --- & -0.008 (0.024) & --- & 0.152*** (0.026) \\ 
COVID Economic Impact & --- & 0.037 (0.025) & --- & -0.027 (0.019) & --- & -0.080* (0.032) \\ 
Income Quintile & --- & -0.027 (0.041) & --- & 0.010 (0.024) & --- & 0.041 (0.032) \\ 
Economic Anxiety & --- & -0.003 (0.029) & --- & 0.008 (0.021) & --- & -0.131*** (0.033) \\ 
Age & --- & -0.015 (0.025) & --- & 0.012 (0.021) & --- & 0.077* (0.035) \\ 
Male & --- & 0.015 (0.040) & --- & -0.031 (0.037) & --- & -0.019 (0.049) \\ 
Education (Secondary) & --- & -0.012 (0.045) & --- & -0.075 (0.065) & --- & -0.091 (0.062) \\ 
Education (University) & --- & -0.008 (0.086) & --- & -0.162* (0.076) & --- & 0.018 (0.086) \\ 
Urban & --- & 0.047* (0.023) & --- & 0.003 (0.019) & --- & 0.066* (0.028) \\ 
Observations & 1037 & 913 & 1153 & 1065 & 1051 & 850 \\ 
R\texttwosuperior & 0.360 & 0.353 & 0.303 & 0.309 & 0.473 & 0.541 \\ 
\bottomrule
\end{tabular*}
\begin{minipage}{\linewidth}
*** p < 0.001, ** p < 0.01, * p < 0.05. SE in parentheses. (1) Core; (2) Full with controls.\\
\end{minipage}

}

\end{table}%




\end{document}
