% Options for packages loaded elsewhere
% Options for packages loaded elsewhere
\PassOptionsToPackage{unicode}{hyperref}
\PassOptionsToPackage{hyphens}{url}
\PassOptionsToPackage{dvipsnames,svgnames,x11names}{xcolor}
%
\documentclass[
  12pt,
]{article}
\usepackage{xcolor}
\usepackage[margin=1in]{geometry}
\usepackage{amsmath,amssymb}
\setcounter{secnumdepth}{5}
\usepackage{iftex}
\ifPDFTeX
  \usepackage[T1]{fontenc}
  \usepackage[utf8]{inputenc}
  \usepackage{textcomp} % provide euro and other symbols
\else % if luatex or xetex
  \usepackage{unicode-math} % this also loads fontspec
  \defaultfontfeatures{Scale=MatchLowercase}
  \defaultfontfeatures[\rmfamily]{Ligatures=TeX,Scale=1}
\fi
\usepackage{lmodern}
\ifPDFTeX\else
  % xetex/luatex font selection
  \setmainfont[]{Times New Roman}
\fi
% Use upquote if available, for straight quotes in verbatim environments
\IfFileExists{upquote.sty}{\usepackage{upquote}}{}
\IfFileExists{microtype.sty}{% use microtype if available
  \usepackage[]{microtype}
  \UseMicrotypeSet[protrusion]{basicmath} % disable protrusion for tt fonts
}{}
\usepackage{setspace}
\makeatletter
\@ifundefined{KOMAClassName}{% if non-KOMA class
  \IfFileExists{parskip.sty}{%
    \usepackage{parskip}
  }{% else
    \setlength{\parindent}{0pt}
    \setlength{\parskip}{6pt plus 2pt minus 1pt}}
}{% if KOMA class
  \KOMAoptions{parskip=half}}
\makeatother
% Make \paragraph and \subparagraph free-standing
\makeatletter
\ifx\paragraph\undefined\else
  \let\oldparagraph\paragraph
  \renewcommand{\paragraph}{
    \@ifstar
      \xxxParagraphStar
      \xxxParagraphNoStar
  }
  \newcommand{\xxxParagraphStar}[1]{\oldparagraph*{#1}\mbox{}}
  \newcommand{\xxxParagraphNoStar}[1]{\oldparagraph{#1}\mbox{}}
\fi
\ifx\subparagraph\undefined\else
  \let\oldsubparagraph\subparagraph
  \renewcommand{\subparagraph}{
    \@ifstar
      \xxxSubParagraphStar
      \xxxSubParagraphNoStar
  }
  \newcommand{\xxxSubParagraphStar}[1]{\oldsubparagraph*{#1}\mbox{}}
  \newcommand{\xxxSubParagraphNoStar}[1]{\oldsubparagraph{#1}\mbox{}}
\fi
\makeatother


\usepackage{longtable,booktabs,array}
\usepackage{calc} % for calculating minipage widths
% Correct order of tables after \paragraph or \subparagraph
\usepackage{etoolbox}
\makeatletter
\patchcmd\longtable{\par}{\if@noskipsec\mbox{}\fi\par}{}{}
\makeatother
% Allow footnotes in longtable head/foot
\IfFileExists{footnotehyper.sty}{\usepackage{footnotehyper}}{\usepackage{footnote}}
\makesavenoteenv{longtable}
\usepackage{graphicx}
\makeatletter
\newsavebox\pandoc@box
\newcommand*\pandocbounded[1]{% scales image to fit in text height/width
  \sbox\pandoc@box{#1}%
  \Gscale@div\@tempa{\textheight}{\dimexpr\ht\pandoc@box+\dp\pandoc@box\relax}%
  \Gscale@div\@tempb{\linewidth}{\wd\pandoc@box}%
  \ifdim\@tempb\p@<\@tempa\p@\let\@tempa\@tempb\fi% select the smaller of both
  \ifdim\@tempa\p@<\p@\scalebox{\@tempa}{\usebox\pandoc@box}%
  \else\usebox{\pandoc@box}%
  \fi%
}
% Set default figure placement to htbp
\def\fps@figure{htbp}
\makeatother





\setlength{\emergencystretch}{3em} % prevent overfull lines

\providecommand{\tightlist}{%
  \setlength{\itemsep}{0pt}\setlength{\parskip}{0pt}}



 


\usepackage{booktabs}
\usepackage{caption}
\usepackage{longtable}
\usepackage{colortbl}
\usepackage{array}
\usepackage{anyfontsize}
\usepackage{multirow}
\usepackage{booktabs}
\usepackage{longtable}
\usepackage{pdflscape}
\doublespacing
\makeatletter
\@ifpackageloaded{caption}{}{\usepackage{caption}}
\AtBeginDocument{%
\ifdefined\contentsname
  \renewcommand*\contentsname{Table of contents}
\else
  \newcommand\contentsname{Table of contents}
\fi
\ifdefined\listfigurename
  \renewcommand*\listfigurename{List of Figures}
\else
  \newcommand\listfigurename{List of Figures}
\fi
\ifdefined\listtablename
  \renewcommand*\listtablename{List of Tables}
\else
  \newcommand\listtablename{List of Tables}
\fi
\ifdefined\figurename
  \renewcommand*\figurename{Figure}
\else
  \newcommand\figurename{Figure}
\fi
\ifdefined\tablename
  \renewcommand*\tablename{Table}
\else
  \newcommand\tablename{Table}
\fi
}
\@ifpackageloaded{float}{}{\usepackage{float}}
\floatstyle{ruled}
\@ifundefined{c@chapter}{\newfloat{codelisting}{h}{lop}}{\newfloat{codelisting}{h}{lop}[chapter]}
\floatname{codelisting}{Listing}
\newcommand*\listoflistings{\listof{codelisting}{List of Listings}}
\makeatother
\makeatletter
\makeatother
\makeatletter
\@ifpackageloaded{caption}{}{\usepackage{caption}}
\@ifpackageloaded{subcaption}{}{\usepackage{subcaption}}
\makeatother
\usepackage{bookmark}
\IfFileExists{xurl.sty}{\usepackage{xurl}}{} % add URL line breaks if available
\urlstyle{same}
\hypersetup{
  pdftitle={Trust in Information and Pandemic Approval in Southeast Asia: Evidence from the Vietnam Paradox},
  pdfauthor={Jeffrey Stark},
  pdfkeywords={COVID-19, government approval, information
trust, Southeast Asia, pandemic politics},
  colorlinks=true,
  linkcolor={blue},
  filecolor={Maroon},
  citecolor={Blue},
  urlcolor={Blue},
  pdfcreator={LaTeX via pandoc}}


\title{Trust in Information and Pandemic Approval in Southeast Asia:
Evidence from the Vietnam Paradox}
\author{Jeffrey Stark}
\date{2025-12-24}
\begin{document}
\maketitle
\begin{abstract}
This study examines a puzzling pattern in Southeast Asian pandemic
politics: Vietnam maintained the highest government approval ratings
(97.5\%) despite experiencing the region's highest COVID
self/family-reported infection rates (65.9\%). Using Asian Barometer
Survey Wave 6 data from Cambodia, Thailand, and Vietnam (N=3,679), we
find that trust in government COVID-19 information is strongly
associated with pandemic approval (β = 0.51--0.67, p \textless{} .001),
while personal infection experience shows no significant relationship
with approval in any country (\textbar β\textbar{} \textless{} 0.08).
Economic hardship effects are weak and inconsistent across cases. These
patterns hold in multivariate models controlling for institutional
trust, democratic attitudes, and demographics. The findings suggest that
citizens' evaluations of pandemic governance may depend less on
experienced outcomes health outcomes as reported in the survey rather
than on their trust in government information---a pattern with
potentially important implications for understanding how regimes
maintain legitimacy during crises. However, our cross-sectional design
cannot establish causality, and we discuss alternative interpretations
including cultural factors, social desirability bias, and the
distinction between measuring trust versus information control
mechanisms.
\end{abstract}


\setstretch{2}
\section{Introduction}\label{introduction}

Vietnam presents a puzzle for theories of crisis governance. Despite
experiencing Southeast Asia's highest COVID-19 infection rate---with
65.9\% of survey respondents reporting personal infection---Vietnam
maintained the region's highest government approval ratings at 97.5\%.
This pattern contradicts both rally-around-the-flag theories, which
predict health threats generate diffuse support, and performance
legitimacy accounts, which expect objective outcomes to drive
evaluations. How can we explain approval that appears disconnected from
performance?

We propose that trust in government information may help account for
this apparent paradox. Using Asian Barometer Survey data from Cambodia,
Thailand, and Vietnam, we find that trust in government COVID-19
information is strongly associated with pandemic approval across all
three countries, while personal infection experience and economic
hardship show weak or null relationships. These findings are consistent
with an information-environment framework suggesting that citizens'
perceptions of government communication shape their evaluations
independently of---or perhaps instead of---objective outcomes.

However, we emphasize at the outset what our data can and cannot show.
Our cross-sectional survey design establishes associations, not causal
mechanisms. We measure \emph{trust} in government information, not
information control, media manipulation, or propaganda effectiveness.
High trust in authoritarian contexts could reflect successful narrative
management, but it could equally reflect effective and accurate
communication, cultural norms of deference, or social desirability bias
in survey responses. We present our theoretical interpretation as one
plausible account consistent with the data, while acknowledging
alternatives that our design cannot rule out.

The empirical patterns are striking nonetheless. When the Asian
Barometer Wave 6 surveyed citizens across Southeast Asia in late
2022--2023, Vietnam recorded the highest government approval rating for
pandemic handling: 97.5\% of respondents rated the government's
performance as ``fairly well'' or ``very well.'' This exceeded even
Cambodia's approval (93.6\%), despite Cambodia's relative success in
limiting infections to just 8.7\%. Thailand, with moderate infection
rates (40.1\%), recorded far lower approval ratings (37.7\%). Vietnam's
paradox is stark: the worst health outcomes corresponded to the highest
government approval.

This finding raises questions about fundamental assumptions in crisis
politics and government accountability. Rally-around-the-flag theories
predict that citizens unite behind leaders during external threats
(Mueller 1970; Baker \& Oneal 2001), but such rallies might be expected
to strengthen when governments successfully protect citizens, not when
they fail to do so. Economic voting theories suggest that citizens
punish incumbents for poor crisis management (Lewis-Beck \& Stegmaier
2000; Duch \& Stevenson 2008), making Vietnam's high approval puzzling.
Several existing explanations prove inadequate: performance-based
accounts cannot explain why Cambodia (successful disease control) and
Vietnam (high infections) achieved similar approval levels while
Thailand (moderate outcomes) showed much lower approval; authoritarian
repression theories fail because Vietnam did not hide pandemic
data---infection rates were publicly acknowledged and widely
experienced; cultural explanations attributing Asian deference to
authority cannot account for Thailand's low approval despite similar
cultural contexts.

We argue that trust in government information may help explain the
Vietnam paradox---though we acknowledge our data cannot definitively
establish this mechanism. In our framework, what matters for approval
may be not how many people suffer but rather whether citizens trust how
the regime communicates about that suffering. Vietnam maintained 91.9\%
public trust in government COVID information despite mass infections,
while Thailand's more pluralistic information environment corresponded
to only 34\% trust despite better outcomes. This trust
differential---not infection differentials---is strongly associated with
the approval patterns we observe.

Contribution and scope. This article makes a deliberately modest
contribution. Using nationally representative Asian Barometer Survey
Wave 6 data from Cambodia, Thailand, and Vietnam, we document a robust
empirical pattern: trust in official government COVID-19 information is
a much stronger correlate of pandemic-handling approval than
respondents' self/family-reported infection exposure or reported
economic hardship. We do not claim to identify propaganda, censorship,
or information control mechanisms, and we do not treat the ABS infection
item as an epidemiological infection rate. Rather, we show that within
each country, experiential indicators that should anchor ``performance''
accounts are weakly related to approval, while informational trust is
strongly related---an empirical pattern that any theory of crisis
accountability must explain.

Interpretive claim and falsifiable implication. Our interpretation is
consistent with an ``information-credibility heuristic'' account: when
citizens perceive official COVID-19 information as credible, they may
evaluate crisis governance through official frames that attenuate the
political consequences of negative experiences. This interpretation
yields a falsifiable within-country implication that we test: the
relationship between infection exposure and approval should be stronger
(more negative) among low-trust respondents than among high-trust
respondents (i.e., a trust × infection interaction), even if the average
infection--approval association is small.

Roadmap. We proceed as follows. The next section clarifies how
informational trust could shape crisis evaluations and distinguishes our
measured construct (trust in official COVID-19 information) from
unmeasured information-environment mechanisms (media exposure,
censorship, propaganda). We then describe the ABS Wave 6 data,
measurement strategy, and the main threats to inference---especially
cross-country fieldwork timing and cross-sectional endogeneity. The
Results section documents the ``Vietnam paradox'' in descriptive terms,
estimates bivariate and multivariate models within each country, and
reports robustness checks (including ordinal models and alternative
specifications). The Discussion considers alternative
interpretations---reverse causality, generalized regime support, and
social desirability bias---and outlines research designs needed to
adjudicate among them.

\section{Theoretical Framework}\label{theoretical-framework}

\subsection{Performance Legitimacy and Its
Limits}\label{performance-legitimacy-and-its-limits}

Conventional theories of political legitimacy emphasize performance as
the foundation of government support (Lipset 1959; Easton 1965).
Citizens evaluate governments based on outcomes---economic growth,
security provision, service delivery---and withdraw support when
performance falters. This ``performance legitimacy'' framework predicts
that COVID-19 outcomes should directly shape pandemic approval:
governments that controlled infection and minimized economic disruption
should enjoy higher support than those presiding over health
catastrophes and economic collapse.

The Vietnam case appears to contradict this expectation. Vietnam
experienced among the highest infection rates in Southeast Asia, yet
maintained the highest approval. This disconnect between objective
performance and subjective evaluation suggests that factors beyond
outcomes may shape crisis-era legitimacy.

\subsection{Information Credibility and Crisis
Evaluation}\label{information-credibility-and-crisis-evaluation}

We advance an information-focused account of crisis evaluation that is
compatible with, but narrower than, broad claims about ``information
control.'' At the individual level, citizens often rely on cognitive
shortcuts when evaluating complex policy domains, especially during
crises when direct knowledge is limited and events are fast-moving. One
such shortcut is perceived credibility of official communication: if
citizens trust that the government's COVID-19 information is accurate
and reliable, they may use official narratives as a primary lens for
attributing responsibility, interpreting uncertainty, and judging
competence.

This ``information-credibility heuristic'' matters because many negative
pandemic experiences are attributionally ambiguous. Infection exposure,
for example, can be interpreted as personal misfortune, an unavoidable
feature of a global pandemic, or a preventable policy failure. Economic
hardship can be interpreted as government negligence, a necessary
tradeoff, or a global shock beyond any incumbent's control. When
official information is perceived as credible, citizens may be more
likely to accept government explanations, assign blame externally, or
view adverse outcomes as unavoidable---attenuating the extent to which
negative experiences translate into disapproval. When official
information is not perceived as credible, negative experiences may more
readily convert into political blame and lower approval.

Importantly, this account does not require that governments manipulate
information, censor criticism, or dominate media exposure---though such
macro-level conditions could plausibly shape perceived credibility. Our
study does not measure media pluralism, information exposure, propaganda
intensity, or censorship. We measure a single, individual-level
attitude: self-reported trust in the government's COVID-19 information.
For that reason, we treat ``information environment'' as a contextual
interpretation rather than an observed mechanism.

This framework yields two empirical expectations that are assessable
with our cross-sectional survey design. First, trust in official
COVID-19 information should be strongly associated with approval of
pandemic handling, net of institutional trust and other political
attitudes that may confound the relationship. Second---and more
discriminating---the relationship between negative experiences
(infection exposure and economic hardship) and approval should be
conditional on informational trust: experiential shocks should matter
more among low-trust respondents than among high-trust respondents. We
emphasize that even if these patterns hold, the data cannot establish
causal ordering; trust may shape approval, approval may shape trust, or
both may reflect a deeper orientation toward the regime. Nevertheless,
observing the predicted conditionality would strengthen the plausibility
of an information-credibility account relative to interpretations in
which experiential outcomes directly drive approval in a uniform way.

\subsection{Conceptualizing Information Trust as Distinct from
Performance
Approval}\label{conceptualizing-information-trust-as-distinct-from-performance-approval}

Our theoretical argument requires distinguishing between two
conceptually related but potentially distinct attitudes: \emph{trust in
official information} and \emph{performance approval}. Trust in
government COVID information captures citizens' assessment of whether
official sources provide reliable information about the pandemic.
Government approval represents an evaluative judgment about policy
performance. While these attitudes may be interconnected, they
potentially operate at different levels of political cognition.

We theorize information trust as potentially shaping how citizens
interpret objective conditions when forming performance evaluations.
However, we acknowledge that the correlation between trust and approval
could alternatively indicate that both measures tap the same underlying
construct of regime support. Our empirical analysis includes validity
checks to assess whether these constructs are distinguishable.

If our empirical analysis instead revealed that information trust and
approval were empirically indistinguishable---measuring the same
underlying attitude of regime loyalty---this would suggest our
theoretical framework is misspecified, and that what we interpret as
``information environment effects'' actually reflects generalized regime
support.

\subsection{Alternative Explanations}\label{alternative-explanations}

Several alternative explanations merit consideration.

\textbf{Cultural Deference.} Perhaps cultural norms emphasizing respect
for authority explain high approval in Cambodia and Vietnam (Dalton \&
Ong 2005). However, this cannot easily explain Thailand's low approval
(37.7\%) despite similar cultural context, nor why personal infection
failed to reduce approval even among those directly affected.

\textbf{Prior Regime Legitimacy.} Perhaps Vietnam's approval reflects
pre-existing regime legitimacy from economic development or historical
legacy rather than pandemic-specific factors (Malesky \& London 2014).
However, this cannot explain why infection and economic hardship showed
such weak associations with approval---high prior legitimacy might
establish baseline approval, but we would still expect negative
experiences to reduce it somewhat.

\textbf{Social Desirability Bias.} Perhaps respondents in authoritarian
contexts report high approval due to fear of reprisal for expressing
dissatisfaction (Kuran 1997). This is a serious concern that we cannot
fully address. However, Asian Barometer uses internationally
standardized anonymity protocols, and if fear drove responses uniformly,
we should observe uniformly high responses across all questions---yet
democracy satisfaction shows much lower levels than government approval.

\textbf{Economic Compensation.} Perhaps governments provided sufficient
economic relief that material hardship did not translate into political
punishment (Bol et al.~2021). However, our economic hardship measures
capture realized suffering after any government support, and this cannot
explain why infection---which governments cannot compensate---also
failed to reduce approval.

We cannot definitively rule out these alternatives with our data. Our
contribution is documenting empirical patterns and offering one
theoretical interpretation; alternative interpretations remain
plausible.

\subsection{Summary: From Theory to
Empirics}\label{summary-from-theory-to-empirics}

Our information-environment framework offers one interpretation of the
Vietnam paradox, suggesting that trust in government COVID-19
information may mediate the relationship between objective outcomes and
political evaluations. When citizens trust government information, they
may interpret their experiences---including infection and economic
hardship---through officially sanctioned frames that preserve positive
evaluations. In this account, what matters for approval is not what
happened but how citizens understand what happened.

This framework generates testable expectations, though we note that our
cross-sectional data can only assess whether patterns are
\emph{consistent with} this interpretation, not whether the
interpretation is correct:

\begin{enumerate}
\def\labelenumi{\arabic{enumi}.}
\tightlist
\item
  Trust in government COVID-19 information should be more strongly
  associated with approval than personal infection or economic impact.
\item
  Personal infection should show negligible direct associations with
  approval.
\item
  Aggregate-level patterns (Vietnam's high approval despite high
  infections) should correspond to high levels of information trust
  rather than individual-level rally effects.
\end{enumerate}

\section{Data and Methods}\label{data-and-methods}

\subsection{Data Source and Sampling}\label{data-source-and-sampling}

We analyze data from the Asian Barometer Survey (ABS) Wave 6, conducted
between late 2022 and early 2023 across three Southeast Asian countries:
Cambodia, Thailand, and Vietnam. The Asian Barometer is a collaborative
regional survey research program monitoring political attitudes, values,
and behaviors in East and Southeast Asia (Chu et al.~2008). Wave 6
included extensive modules on COVID-19 experiences, government pandemic
responses, and political attitudes, making it well-suited for examining
pandemic politics.

The ABS employs multi-stage stratified random sampling to achieve
nationally representative samples of adult citizens (ages 18+) in each
country. The sampling design uses probability proportional to size (PPS)
methods at each stage: primary sampling units (provinces/regions),
secondary sampling units (districts/municipalities), tertiary sampling
units (villages/neighborhoods), and finally households and individual
respondents. Within-household selection follows Kish grid methods to
ensure randomization. This rigorous sampling design produces
high-quality data suitable for population-level inference.

Our analytic sample includes 3,679 respondents across three countries:
Cambodia (n = 1,242), Vietnam (n = 1,237), and Thailand (n = 1,200).
Response rates ranged from 68\% (Thailand) to 82\% (Cambodia),
comparable to other high-quality surveys in the region. All surveys were
conducted face-to-face by trained enumerators using standardized
questionnaires translated into local languages (Khmer, Vietnamese, Thai)
and back-translated to ensure equivalence.

\textbf{Case Selection Rationale.} We selected these three countries to
maximize theoretical leverage through controlled comparison. All three
are Southeast Asian nations with similar cultural contexts, geographic
proximity, and exposure to COVID-19 during the same time period. Yet
they vary on theoretically relevant dimensions: regime type (Cambodia
and Vietnam are authoritarian; Thailand is a hybrid regime with greater
information pluralism); pandemic outcomes (infection rates varied
dramatically); and information environments (Cambodia and Vietnam
maintain more extensive state influence over media than Thailand).

\textbf{Interview Timing.} Data collection timing varied across
countries: Cambodia (primarily December 2021), Thailand (primarily
April-May 2022), and Vietnam (primarily August-September 2022). This
temporal variation means respondents in each country were surveyed at
different pandemic phases. While this introduces potential confounds, it
paradoxically strengthens our core finding in one sense: Vietnam was
surveyed during objectively worse conditions yet shows higher
approval---the opposite of what we would expect if objective conditions
drove evaluations. Within Vietnam, we find no significant variation in
approval across interview months (p = 0.296), indicating that timing
variation within-country does not explain our results.

\subsection{Measurement Strategy}\label{measurement-strategy}

Our analysis examines relationships between three types of COVID-19
impacts (infection, economic hardship, trust in information) and
government pandemic approval. All variables are described below with
exact question wording provided in Online Appendix A.

\textbf{Dependent Variable: Government Pandemic Approval.} Our primary
outcome is a single-item measure: ``Overall, how would you rate the way
the government has handled the COVID-19 crisis?'' Response options: 1 =
Very badly, 2 = Fairly badly, 3 = Fairly well, 4 = Very well. We treat
this as a continuous variable (mean = 3.29, SD = 0.87).

\textbf{Primary Independent Variables}

\begin{itemize}
\item
  \textbf{Personal COVID-19 Infection.} Binary indicator (0 = no, 1 =
  yes) based on: ``Have you or anyone in your immediate family been
  infected with COVID-19?'' Overall infection rate: 39.8\% (Cambodia
  8.7\%, Thailand 40.1\%, Vietnam 65.9\%).
\item
  \textbf{COVID-19 Economic Impact Severity.} Four-point severity scale
  (mean = 2.31, SD = 0.94).
\item
  \textbf{Trust in Government COVID Information.} Single item: ``How
  much do you trust the information about COVID-19 provided by the
  government?'' (mean = 3.21, SD = 0.79).
\end{itemize}

\textbf{Control Variables.} We include institutional trust (9-item
index, α = 0.947), democracy satisfaction, authoritarian acceptance (α =
NA), and demographic controls (age, gender, education, urban residence).

\subsection{Measurement Scope and Interpretive
Limits}\label{measurement-scope-and-interpretive-limits}

Before proceeding to analysis, we note an important distinction between
what our measures capture and the theoretical mechanisms they are
intended to assess. Our key independent variable---trust in government
COVID-19 information---measures respondents' \emph{stated trust} in
government communication. This is not equivalent to, and should not be
conflated with, ``information control,'' ``narrative management,'' or
``propaganda effectiveness''---terms that appear in our theoretical
framework.

High trust could arise through multiple pathways: effective and accurate
government communication, successful manipulation or censorship,
cultural norms of deference to authority, social desirability bias in
survey responses, or pre-existing regime legitimacy independent of
pandemic performance. Our data cannot distinguish among these
mechanisms. When we interpret associations between information trust and
approval through an ``information environment'' lens, we are offering
one theoretically motivated interpretation, not establishing that
information control mechanisms actually operate.

Similarly, our cross-sectional design establishes associations at a
single time point, not causal relationships. We cannot determine whether
trust causes approval, approval causes trust, or both reflect a common
underlying orientation toward the regime. Longitudinal or experimental
designs would be needed to address causality.

We flag these limitations here, and return to them in the Discussion, to
ensure readers interpret our findings appropriately. Our contribution is
documenting robust empirical patterns---the strong association between
information trust and approval relative to experiential
predictors---that are \emph{consistent with} an information-environment
framework. Whether that framework correctly explains the patterns is a
question our data cannot answer.

\subsection{Measurement Validity}\label{measurement-validity}

A potential methodological concern is that our independent variable
(trust in government COVID information) and dependent variable
(government pandemic approval) may be tautological---measuring the same
underlying construct rather than capturing distinct attitudes. We
address this concern through three approaches.

First, we examine the bivariate relationship. The correlation between
trust and approval in Vietnam is moderate (r = 0.51), indicating shared
variance of 26\%---substantial but leaving 74\% independent.

Second, we examine cross-tabulation. In Vietnam, 32.9\% of respondents
who trust government COVID information nonetheless disapprove of
pandemic response. This ``off-diagonal'' pattern suggests these
variables capture somewhat distinct constructs.

Third, we assess multicollinearity using variance inflation factors
(VIF). Across all countries, VIF scores range from 1 to 1.63, below the
conventional threshold of 5.0.

These diagnostics suggest that trust and approval are empirically
distinguishable, though we acknowledge conceptual overlap remains a
concern.

\subsection{Analytic Strategy}\label{analytic-strategy}

Our analysis proceeds in four stages: descriptive analysis documenting
the Vietnam paradox; bivariate OLS regression testing associations
between COVID impacts and approval within each country; multivariate OLS
regression controlling for potential confounders; and extensive
robustness checks (see Online Appendix for alternative estimators,
samples, and specifications). We estimate models separately by country
because our framework predicts similar within-country patterns but
different baseline levels between countries.

For key coefficients, we calculate bootstrap confidence intervals (2,000
resamples) to assess uncertainty. We also compute effect sizes (Cohen's
d) for between-country comparisons.

\subsection{Limitations and Scope
Conditions}\label{limitations-and-scope-conditions}

\textbf{Cross-Sectional Design.} We cannot establish causal ordering.
While our framework posits trust as shaping approval, the reverse is
equally plausible from these data alone.

\textbf{Self-Reported Measures.} All variables rely on self-reports,
with potential social desirability bias especially concerning in
authoritarian contexts. Asian Barometer's anonymity protocols minimize
but cannot eliminate this concern.

\textbf{Three-Country Comparison.} Our findings may not generalize
beyond Southeast Asia or the COVID-19 pandemic context.

\textbf{Timing Differences.} Non-overlapping interview periods across
countries introduce potential confounds, though patterns are consistent
with our framework despite this variation.

\section{Results}\label{results}

\subsection{Descriptive Overview: The Vietnam
Paradox}\label{descriptive-overview-the-vietnam-paradox}

Table 1 presents the core puzzle. Vietnam experienced the highest
COVID-19 infection rate (65.9\%), more than seven times higher than
Cambodia (8.7\%) and substantially higher than Thailand (40.1\%). Yet
Vietnam also recorded the highest government approval rating (97.5\%),
exceeding Cambodia (93.6\%) and far surpassing Thailand (37.7\%). This
pattern contradicts expectations that poor pandemic outcomes should
reduce government approval.

\begin{table}

\caption{\label{tbl-paradox}The Vietnam Paradox: COVID Outcomes and
Government Approval}

\centering{

\caption*{
{\fontsize{20}{25}\selectfont  The Vietnam Paradox\fontsize{7}{8}\selectfont } \\ 
{\fontsize{14}{17}\selectfont  COVID-19 infection rates vs. government approval by country\fontsize{7}{8}\selectfont }
} 
\fontsize{7.0pt}{8.0pt}\selectfont
\begin{tabular*}{1\linewidth}{@{\extracolsep{\fill}}lrrrrrrrrrr}
\toprule
country\_name & n & covid\_infection\_rate & govt\_approval\_rate & mean\_gov\_handling & trust\_covid\_rate & mean\_trust\_covid & mean\_institutional\_trust & dem\_satisfaction\_pct & mean\_dem\_sat & emergency\_support \\ 
\midrule\addlinespace[2.5pt]
Cambodia & 1,242.0 & 8.7 & 93.6 & 3.2 & 86.6 & 3.1 & 0.2 & 81.6 & 0.2 & 86.5 \\ 
Thailand & 1,200.0 & 40.1 & 37.7 & 2.2 & 34.0 & 2.2 & -1.0 & 30.9 & -0.7 & 60.7 \\ 
Vietnam & 1,237.0 & 65.9 & 97.5 & 3.6 & 91.9 & 3.4 & 0.7 & 88.5 & 0.5 & 94.0 \\ 
\bottomrule
\end{tabular*}

}

\end{table}%

\textsubscript{Source:
\href{https://starkjeffrey.github.io/AsianBarometer_ResearchHub/manuscript_revised.qmd.html}{Article
Notebook}}

\subsection{Individual-Level
Relationships}\label{individual-level-relationships}

\textbf{Personal Infection and Approval.} Contrary to
rally-around-the-flag expectations, personal COVID-19 infection showed
negligible associations with government approval across all three
countries (Table 2). In Cambodia, the coefficient was effectively zero
(β = -0.014, p = 0.72). Vietnam exhibited a slight positive coefficient
(β = 0.005, p = 0.8). Thailand showed a weak negative coefficient (β =
-0.072, p \textless{} 0.05), but the effect remained small.

\begin{table}

\caption{\label{tbl-infection-regression}Bivariate OLS Regression:
COVID-19 Infection → Government Approval}

\centering{

\fontsize{7.0pt}{8.0pt}\selectfont
\begin{tabular*}{1\linewidth}{@{\extracolsep{\fill}}lrrrrr}
\toprule
Country & Coefficient & p-value & 95\% CI Lower & 95\% CI Upper & N \\ 
\midrule\addlinespace[2.5pt]
Cambodia & -0.014 & 0.7209 & -0.092 & 0.063 & 1,037 \\ 
Thailand & -0.072 & 0.0209 & -0.134 & -0.011 & 1,051 \\ 
Vietnam & 0.005 & 0.7969 & -0.036 & 0.047 & 1,153 \\ 
\bottomrule
\end{tabular*}

}

\end{table}%

\textsubscript{Source:
\href{https://starkjeffrey.github.io/AsianBarometer_ResearchHub/manuscript_revised.qmd.html}{Article
Notebook}}

\textbf{Economic Hardship and Approval.} Economic impacts showed more
variation but remained modest predictors (Table 3). Thailand exhibited
the expected negative relationship ({[}Results not yet available{]}).
Vietnam showed essentially no relationship ({[}Results not yet
available{]}). Cambodia exhibited a positive coefficient ({[}Results not
yet available{]}).

\textbf{Trust in COVID Information and Approval.} In contrast to the
weak effects of infection and economic hardship, trust in government
COVID information showed strong positive associations with approval
across all three countries (Table 4). Cambodia (β = 0.549, p \textless{}
0.001), Vietnam (β = 0.423, p \textless{} 0.001), and Thailand (β =
0.725, p \textless{} 0.001) all showed coefficients exceeding 0.50,
explaining 26--45\% of variance in government approval.

\begin{table}

\caption{\label{tbl-trust-regression}Bivariate OLS Regression: Trust in
COVID Info → Government Approval}

\centering{

\fontsize{7.0pt}{8.0pt}\selectfont
\begin{tabular*}{1\linewidth}{@{\extracolsep{\fill}}lrrrrr}
\toprule
Country & Coefficient & p-value & 95\% CI Lower & 95\% CI Upper & N \\ 
\midrule\addlinespace[2.5pt]
Cambodia & 0.549 & 0.0000 & 0.502 & 0.595 & 1,037 \\ 
Thailand & 0.725 & 0.0000 & 0.676 & 0.774 & 1,051 \\ 
Vietnam & 0.423 & 0.0000 & 0.381 & 0.465 & 1,153 \\ 
\bottomrule
\end{tabular*}

}

\end{table}%

\textsubscript{Source:
\href{https://starkjeffrey.github.io/AsianBarometer_ResearchHub/manuscript_revised.qmd.html}{Article
Notebook}}

\textbf{Comparative Magnitudes.} Figure 1 compares the three impact
types. Trust effects were 10--77 times larger than infection effects
across countries.

\phantomsection\label{cell-fig-comparative-effects}
\begin{figure}[H]

\centering{

\pandocbounded{\includegraphics[keepaspectratio]{manuscript_revised_files/figure-pdf/fig-comparative-effects-1.pdf}}

}

\caption{\label{fig-comparative-effects}Comparative Regression
Coefficients: COVID Impact Types → Government Approval}

\end{figure}%

\textsubscript{Source:
\href{https://starkjeffrey.github.io/AsianBarometer_ResearchHub/manuscript_revised.qmd.html}{Article
Notebook}}

\subsection{Multivariate Results}\label{multivariate-results}

Table 5 presents multivariate OLS models controlling for institutional
trust, democracy satisfaction, and demographics. Across specifications
and countries, trust in government COVID information remains the
dominant predictor (β = 0.31 to 0.62, all p \textless{} 0.001). COVID
infection status remains non-significant in all models.

\begin{table}

\caption{\label{tbl-multivariate-ols}Multivariate OLS Regression:
Determinants of Government Pandemic Approval}

\centering{

\caption*{
{\fontsize{20}{25}\selectfont  Multivariate OLS Regression Results\fontsize{7}{8}\selectfont } \\ 
{\fontsize{14}{17}\selectfont  DV: Government Pandemic Handling (1-4 scale)\fontsize{7}{8}\selectfont }
} 
\fontsize{7.0pt}{8.0pt}\selectfont
\begin{tabular*}{1\linewidth}{@{\extracolsep{\fill}}llrlrlr}
\toprule
 & \multicolumn{2}{c}{Cambodia} & \multicolumn{2}{c}{Vietnam} & \multicolumn{2}{c}{Thailand} \\ 
\cmidrule(lr){2-3} \cmidrule(lr){4-5} \cmidrule(lr){6-7}
Variable & Cambodia (M1) & Cambodia (M2) & Vietnam (M1) & Vietnam (M2) & Thailand (M1) & Thailand (M2) \\ 
\midrule\addlinespace[2.5pt]
COVID Infected & 0.003 & -0.006 & 0.022 & 0.032 & -0.038 & -0.035 \\ 
Trust COVID Info & 0.497*** & 0.480*** & 0.325*** & 0.314*** & 0.628*** & 0.616*** \\ 
Institutional Trust & 0.088*** & 0.090*** & 0.102*** & 0.113*** & 0.046*** & 0.025*** \\ 
Democracy Satisfaction & 0.070 & 0.064 & 0.125 & 0.112 & 0.159 & 0.140 \\ 
Authoritarianism & \textemdash & 0.013 & \textemdash & -0.012 & \textemdash & 0.142 \\ 
N & 1037 & 963 & 1153 & 1094 & 1051 & 956 \\ 
R\texttwosuperior & 0.360 & 0.356 & 0.303 & 0.309 & 0.473 & 0.517 \\ 
\bottomrule
\end{tabular*}
\begin{minipage}{\linewidth}
*** p < 0.001. M1 = Core model; M2 = Full model with demographics.\\
\end{minipage}

}

\end{table}%

\textsubscript{Source:
\href{https://starkjeffrey.github.io/AsianBarometer_ResearchHub/manuscript_revised.qmd.html}{Article
Notebook}}

\subsection{Summary of Key Findings}\label{summary-of-key-findings}

Four patterns emerge from these analyses.

First, personal COVID-19 infection showed negligible associations with
government approval (\textbar β\textbar{} \textless{} 0.07) across all
three countries. Vietnamese respondents who personally contracted
COVID-19 were no more critical of government handling than uninfected
respondents.

Second, economic hardship showed weak and inconsistent associations (β =
-0.1 to +0.02). While Thailand showed the expected negative
relationship, Cambodia's positive coefficient and Vietnam's null effect
suggest economic considerations were not consistently linked to
approval.

Third, trust in government COVID information was strongly associated
with approval across all three countries (β = 0.42--0.72), with effect
sizes 10--77 times larger than infection or economic impacts.

Fourth, the aggregate paradox corresponds to information trust patterns.
Vietnam's high approval despite high infections coincides with Vietnam
showing the highest levels of trust in government COVID information
(91.9\%).

\section{Discussion}\label{discussion}

\subsection{Summary of Findings}\label{summary-of-findings}

Our analysis documents a striking empirical pattern: trust in government
COVID-19 information is more strongly associated with pandemic approval
than personal infection or economic hardship across three Southeast
Asian countries. This pattern holds in multivariate models controlling
for institutional trust, democratic attitudes, and demographics.
Vietnam's paradox---the highest infection rate corresponding to the
highest approval---coincides with Vietnam also showing the highest
levels of information trust.

\subsection{Interpretation: An Information-Environment
Account}\label{interpretation-an-information-environment-account}

These patterns are consistent with an information-environment framework
in which citizens' trust in government communication shapes pandemic
evaluations independently of objective outcomes. If citizens trust what
the government tells them about COVID-19, they may interpret even
negative personal experiences in ways that do not translate to
government disapproval. The collective nature of Vietnam's paradox
supports this interpretation: Vietnam's high approval reflects not
infected individuals rallying to support the government
(individual-level rally effects), but rather a population-wide pattern
of high information trust despite widespread infection.

This framework, \emph{if correct}, has potential implications for
understanding regime resilience during crises. Governments that maintain
trusted communication channels may be able to sustain legitimacy even
when objective performance is poor, while governments facing information
pluralism may be held more accountable for outcomes. The comparison
between Vietnam and Thailand illustrates this potential dynamic---though
we emphasize that our data establish association, not causation.

\subsection{Alternative
Interpretations}\label{alternative-interpretations}

However, our data are consistent with several alternative accounts that
we cannot rule out.

\textbf{Effective governance, not manipulation.} Vietnam's high trust
could reflect genuinely effective pandemic communication rather than
propaganda or narrative control. Vietnam's early pandemic response
(2020-2021) was internationally praised; residual trust from that period
could persist. High trust might indicate that Vietnamese citizens
accurately perceived competent communication, not that they were
manipulated.

\textbf{Cultural and institutional factors.} Cross-national differences
in trust and approval may reflect cultural variation in deference to
authority, Confucian traditions of state-society relations, or
differences in institutional legitimacy predating the pandemic. Our
cross-sectional design cannot distinguish pandemic-specific effects from
baseline country differences.

\textbf{Social desirability bias.} Survey respondents in authoritarian
contexts may overreport trust and approval due to fear or internalized
norms. If this bias is stronger in Vietnam and Cambodia than Thailand,
it could account for observed patterns without any
information-environment mechanism.

\textbf{Reverse causality.} Citizens who approve of the government may
trust its information, rather than trust causing approval. Our data
cannot establish temporal ordering.

\textbf{Measurement conflation.} Trust in government COVID information
and approval of pandemic handling may tap the same underlying attitude
of regime support rather than distinct constructs. Although our validity
checks suggest partial distinctiveness, conceptual overlap remains a
concern.

\subsection{What We Can and Cannot
Claim}\label{what-we-can-and-cannot-claim}

Given these limitations, we state clearly what our analysis supports:

\textbf{Supported claims:}

\begin{itemize}
\tightlist
\item
  Trust in government COVID-19 information is more strongly associated
  with pandemic approval than personal infection or economic hardship
  across three Southeast Asian countries.
\item
  Personal infection shows negligible associations with approval,
  inconsistent with simple rally-around-the-flag predictions.
\item
  Vietnam's high approval despite high infections coincides with high
  information trust.
\item
  These patterns are robust across model specifications and robustness
  checks.
\end{itemize}

\textbf{Unsupported claims (that our data cannot establish):}

\begin{itemize}
\tightlist
\item
  That information ``control'' or ``manipulation'' causes these
  patterns.
\item
  That authoritarian regimes successfully ``decouple'' performance from
  legitimacy through narrative management.
\item
  That trust causes approval rather than the reverse.
\item
  That these patterns would replicate in other countries, regions, or
  crisis types.
\end{itemize}

\subsection{Implications}\label{implications}

If our information-environment interpretation has merit---a substantial
``if''---several tentative implications follow.

For rally-around-the-flag theory, our findings suggest that health
crises do not automatically generate government support. The absence of
infection-approval associations in all three countries indicates that
rally effects during pandemics may be conditional on factors our study
cannot identify.

For performance legitimacy theory, the disconnect between objective
outcomes and approval in Vietnam suggests that legitimacy may derive
from sources other than policy effectiveness. Trust in government
information represents one possible source, though others may also
contribute.

For comparative politics, Vietnam's ability to maintain approval during
a health crisis raises questions about regime resilience---though
whether this reflects information environment advantages, effective
governance, or measurement artifacts remains unclear.

We caution against strong conclusions about ``authoritarian
advantages.'' Thailand's low approval despite moderate infections could
indicate democratic accountability functioning appropriately---citizens
holding their government responsible---rather than democratic
``vulnerability.''

\section{Conclusion}\label{conclusion}

This study examined why Vietnam maintained the highest government
approval ratings in Southeast Asia despite experiencing the region's
highest COVID-19 infection rates. Using Asian Barometer Survey data from
Cambodia, Thailand, and Vietnam, we found that trust in government
COVID-19 information was the strongest predictor of pandemic approval
across all three countries, while personal infection and economic
hardship showed weak or null associations.

These findings are consistent with an information-environment framework
suggesting that citizens evaluate pandemic governance based on their
perceptions of government communication rather than objective health
outcomes. Vietnam's paradox---severe infections coexisting with high
approval---coincides with exceptionally high trust in government
COVID-19 information (91.9\%), suggesting the aggregate pattern
corresponds to collective information trust rather than individual rally
effects.

However, we emphasize the limits of what our analysis can establish. Our
cross-sectional design precludes causal claims. We measure trust in
government information, not information control, propaganda
effectiveness, or media manipulation. High trust in Vietnam could
reflect successful narrative management, but it could equally reflect
effective communication, cultural factors, social desirability bias, or
pre-existing legitimacy. Our three-country comparison provides limited
basis for generalization.

Three contributions emerge from this analysis, each requiring further
investigation. First, we document empirical patterns---the dominance of
information trust over experiential predictors---that warrant
theoretical attention. Second, we highlight the potential importance of
information environments for crisis legitimacy, suggesting a research
agenda examining how government communication shapes citizen
evaluations. Third, we show that the Vietnam paradox is substantial and
does not disappear when controlling for obvious confounds, warranting
explanation.

Future research should pursue several directions. Longitudinal studies
tracking approval as pandemic conditions evolved would help establish
whether trust precedes approval. Experimental designs varying exposure
to different information sources could identify causal effects. Media
content analysis could assess what citizens are actually exposed to.
Expanded comparative analysis would test generalizability.

What we can say with confidence is that, in these three Southeast Asian
countries during the COVID-19 pandemic, citizens' evaluations of
government performance were more closely tied to their trust in
government information than to their personal experiences of infection
or economic hardship. Whether this pattern reflects the power of
information environments to shape political evaluations, or something
else entirely, remains an open question with potentially significant
implications for understanding governance during crises.

\section*{References}\label{references}
\addcontentsline{toc}{section}{References}

\phantomsection\label{refs}




\end{document}
